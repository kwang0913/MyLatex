\chapter{Set Theory, Probability, and Single Experiment}
\section{From Set to Probability (of the single experiment)}
\begin{center}
\begin{tabular}{|c|c|}
    \hline
    \textbf{Set Theory} & \textbf{Probability} \\
    \hline
    \hline
    Element & Outcome \\
    \hline
    Subset  & Event   \\
    \hline
    Universal Set & Sample Space ($\Omega$) \\
    \hline
\end{tabular}
\end{center}
\begin{enumerate}
    \item There are three Set Operations: $A \cup B, A \cap B, A^{\mathsf{c}}$.
    \item A probability $\P{\cdot}$ is a function that maps events in the sample space to real numbers such that $\P{\emptyset}=0$, $\P{\text{Event}}\geq 0$, and $\P{\Omega}=1$, where $\emptyset$ is null set has no element (i.e., event has no outcome).
    \item $\P{A \cup B} = \P{A}+\P{B}-\P{AB}$, where $\P{AB}=\P{A\cap B}$.
    \item Union Bound: $\P{A\cup B}\leq \P{A}+\P{B}$. And $\P{\cup_{i=1}^N A_i}\leq \sum_{i=1}^{N}\P{A_i}$ for more than two sets.
\end{enumerate}
\section{Set Properties and corresponding Probability Properties}
\begin{enumerate}
    \item Mutually Exclusive: $A \cap B = \emptyset \Rightarrow \P{A \cap B}=0$, which implies $\P{A \cup B} = \P{A}+\P{B}$.
    \item Pairwise Mutually Exclusive: $A_i \cap A_j = \emptyset$ for $i\neq j$.
    \item Outcomes are always Pairwise Mutually Exclusive since they are the smallest units (i.e., Elements) in the Set.
    \item Collectively Exhaustive: $\cup_{i=1}^{N}A_i = \Omega \Rightarrow \P{\cup_{i=1}^{N}A_i} = 1$.
    \item Partitions (i.e., Mutually Exclusive \& Collectively Exhaustive): $\P{\cup_{i=1}^N A_i}=\sum_{i=1}^{N}\P{A_i}=1$.
\end{enumerate}
\section{Conditional Probability and Bayes' Theorem}
\begin{enumerate}
    \item $\condP{A}{B} = \P{AB}/\P{B}$.
    \item If $A_i$ are Mutually Exclusive: $\condP{A}{B} = \condP{\cup_{i=1}^{N}A_i}{B} = \sum_{i=1}^{N}\condP{A_i}{B}$.
    \item If $B_i$ are Partitions (Law of Total Number),
    \begin{align}
        \condP{A}{B}
        &= \frac{\P{AB}}{\P{B}}  \tag{Definition of Conditional Probability}\\
        &= \P{AB}  \tag{$B$ is Collectively Exhaustive so $\P{B}=1$} \\
        &= \P{A\cdot\bigcup\limits_{i=1}^{N}B_i}  \tag{$B$ is Mutually Exclusive} \\
        &= \sum_{i=1}^{N}\P{AB_i}  \tag{$B$ is Partition} \\
        &= \sum_{i=1}^{N}\condP{A}{B_i}\P{B_i}. \tag{Definition of Conditional Probability}
    \end{align}
\item Bayes' Theorem:{
    \begin{equation*}
        \condP{A}{B} = \frac{\condP{B}{A}\P{A}}{\P{B}}.
    \end{equation*}
}
\end{enumerate}
\section{Independent}
    \begin{enumerate}
        \item $\P{A\cap B}=\P{A}\P{B}$.
        \item $\P{A \cup B} = \P{A}+\P{B}-\P{A}\P{B}$.
        \item $\condP{A}{B} = \frac{\P{AB}}{\P{B}}=\frac{\P{A}\P{B}}{\P{B}}=\P{A}$.
        \item If $A$ and $B$ are independent then $A^{\mathsf{c}}$ and $B$ are independent and so on.{
            \begin{align*}
                \P{B} = \P{(A\cup A^{\mathsf{c}})\cap B}
                &= \P{AB}+\P{A^{\mathsf{c}}B} && \text{($A$ and $A^\mathsf{c}$ are partitions)}\\
                \Rightarrow \P{A^{\mathsf{c}}B}
                &= \P{B}-\P{AB} \\
                &= \P{B}-\P{A}\P{B} \\
                &= \P{B}\left(1-\P{A}\right) \\
                &= \P{B}\P{A^{\mathsf{c}}}.
            \end{align*}
        }
    \end{enumerate}