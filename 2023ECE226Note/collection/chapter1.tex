\chapter{Combinatorics, Set Theory, and Probability}

\section{Counting Methods}
% (\textbf{Essentially the outcomes in each experiment (i.e., sample space) are equiprobable})
\begin{enumerate}
    \item Basic Principle:~$n_1\times n_2\times \ldots$
    \item Ordered Sampling without Replacement---Permutation (or Arrangement): \[{}_n A_k=\frac{n!}{(n-k)!}.\]
    \item Ordered Sampling with Replacement: \[n^k.\]
    \item Unordered Sampling without Replacement---Combination: \[{}_n C_k=\binom{n}{k}=\frac{n!}{k!(n-k)!}=\binom{n}{n-k}.\]
    \item Unordered Sampling with Replacement: \[\binom{n+k-1}{k}.\]
    \item Combination is Permutation without order. Combination is also called n choose k.
    \item Multiple Combination:{
        \begin{enumerate}
            \item $\binom{n}{k_1,k_2,\ldots,k_m}=\frac{n!}{k_1!k_2!\ldots k_m!}$ where $n=\sum_{i=1}^{m}k_i$.
            \item For the two cases situation, $n=k_1+k_2\Rightarrow \binom{n}{k_1k_2}=\frac{n!}{k_1!k_2!}\iff\binom{n}{k_1}\iff\binom{n}{k_2}$.
        \end{enumerate}
        }
\end{enumerate}

\section{Venn Diagram}

\section{From Set to Probability}
\begin{table}[H]
    \centering
    \begin{tabular}{|c|c|}
        \hline
        \textbf{Set Theory} & \textbf{Probability} \\
        \hline
        \hline
        Element & Outcome \\
        \hline
        Subset  & Event   \\
        \hline
        Universal Set & Sample Space ($\Omega$) \\
        \hline
    \end{tabular}
\end{table}
\begin{enumerate}
    \item There are three Set Operations: $A \cup B, A \cap B, A^{\mathsf{c}}$.
    \item A probability $\P{\cdot}$ is a function that maps events in the sample space to real numbers such that $\P{\emptyset}=0$, $\P{\text{Event}}\geq 0$, and $\P{\Omega}=1$, where $\emptyset$ is null set has no element (\ie, event has no outcome).
    \item $\P{A \cup B} = \P{A}+\P{B}-\P{AB}$, where $\P{AB}=\P{A\cap B}$.
    \item Union Bound: $\P{A\cup B}\leq \P{A}+\P{B}$. And $\P{\cup_{i=1}^N A_i}\leq \sum_{i=1}^{N}\P{A_i}$ for more than two sets.
\end{enumerate}
\begin{axiom}
    [Axioms of Probability]
    The three axioms of probability are:
    \begin{enumerate}
        \item $0\leq \P{A} \leq 1$ for any event $A$.
        \item $\P{\Omega}=1$.
        \item If $A_i$ are Mutually Exclusive, then $\P{\cup_{i=1}^{N}A_i}=\sum_{i=1}^{N}\P{A_i}$.
    \end{enumerate}
\end{axiom}

\section{Set Properties and Corresponding Probability Properties}
\begin{enumerate}
    \item Mutually Exclusive: $A \cap B = \emptyset \Rightarrow \P{A \cap B}=0$, which implies $\P{A \cup B} = \P{A}+\P{B}$.
    % \item Pairwise Mutually Exclusive: $A_i \cap A_j = \emptyset$ for $i\neq j$.
    \item Collectively Exhaustive: $\cup_{i=1}^{N}A_i = \Omega \Rightarrow \P{\cup_{i=1}^{N}A_i} = 1$.
    \item Partitions (\ie, Mutually Exclusive \& Collectively Exhaustive): $\P{\cup_{i=1}^N A_i}=\sum_{i=1}^{N}\P{A_i}=1$.
    \item All Outcomes constitute a partition.
    \item $A$ and $A\complement$ constitute a partition.
    \item If $B_i$ are Collectively Exhaustive, then $\P{A\cap(\cup_{i=1}^N B_i)}=\P{A\Omega}=\P{A}$
    \item For any $B$, $\P{A}=\P{A\cap(B\cup B\complement)}=\P{A\cap B}+\P{A\cap B\complement}$
\end{enumerate}