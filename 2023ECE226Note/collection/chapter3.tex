\chapter{Discrete Random Variables}
\begin{enumerate}
    \item Discrete Random Variables: Assign numerical value to discrete outcomes
    \item Probability Mass Function~(PMF): \[\sum_{x\in X}P_X(x)=1.\]
    \item Families of Discrete Random Variables and their PMF{
        \begin{enumerate}
            \item Bernoulli ($p$): \textbf{E.g., Flip a coin}{
                \[ P_X(x) =
                \begin{cases}
                    1-p & x=0, \\
                    p   & x=1, \\
                    0   & otherwise.
                \end{cases} \]
            }
            \item Binomial ($n, p$): Get \textbf{x} successes in \textbf{n} Bernoulli ($p$) experiments $\iff$ independent trails{
                \[P_X(\textcolor{red}{x}) = {
                    \begin{cases}
                        \binom{n}{\textcolor{red}{x}}p^x(1-p)^{n-x} & x=0,1,\ldots,n \\
                        0 & otherwise.
                    \end{cases}
                }\]
                \textbf{Note:} Bernoulli ($p$) $\iff$ Binomial ($1, p$).
            }
            \item Poisson ($\alpha$): Binomial ($n, p$) with small $p$, large $n$, and $\alpha=np${
                \[ P_X(x) =
                \begin{cases}
                    \dfrac{\alpha^x e^{-\alpha}}{x!}   & x=0,1,\ldots \\
                    0   & otherwise.
                \end{cases} \]
            }
            \item Geometric ($p$): Get the $\bm{1}$\textbf{st} success at the $\bm{x}$\textbf{th} Bernoulli ($p$) experiment {
                \[ P_X(x) =
                \begin{cases}
                    p(1-p)^{x-1} & x=1,2,\ldots \\
                    0   & otherwise.
                \end{cases} \]
            }
            \item Pascal ($k,p$): Get the $\bm{k}$\textbf{th} success at the $\bm{x}$\textbf{th} Bernoulli ($p$) experiment {
                \[P_X(\textcolor{red}{x}) = {
                    \begin{cases}
                        \binom{\textcolor{red}{x}-1}{k-1}p^k(1-p)^{x-k} & x=k,k+1,k+2,\ldots \\
                        0 & otherwise.
                    \end{cases}
                }\]
                \textbf{Note:} Geometric ($p$) $\iff$ Pascal ($1, p$).
            }
            \item Discrete Uniform ($k, l$): outcomes are uniformly distributed on range $(k, l)$ \textbf{E.g., Roll a Die}{
                \[ P_X(x) =
                \begin{cases}
                    1/(l-k+1)   & x=k,k+1,k+2,\ldots,l \\
                    0   & otherwise.
                \end{cases} \]
            }
        \end{enumerate}
    }
    \item Cumulative Distribution Function (CDF): {
        \begin{align*}
            F_X(x)&=P_X[X\leq x]=\sum_{k=0}^{x}P_X(k). \\
            F_X(b)-F_X(a)&=\sum_{k=0}^{b}P_X(k)-\sum_{k=0}^{a}P_X(k)=\sum_{k=a+1}^{b}P_X(k)=P_X(a<X\leq b).
        \end{align*}
        The CDF of Geometric ($p$) is worth to remember
        \begin{align*}
            F_X(x)
            &= P_X[X\leq x] \\
            &= 1-P_X[X > x] \\
            &= 1-\sum_{i=x+1}^{\infty}p(1-p)^{i-1} \\
            &= 1-(1-p)^x\sum_{i=1}^{\infty}p(1-p)^{i-1} \\
            &= 1-(1-p)^x.
        \end{align*}
    }
    \item Average and Expectations{
        \begin{enumerate}
            \item In ordinary language, an \textbf{Average} is a single number taken as representative of a list of numbers.{
            \begin{enumerate}
                \item Mode: The outcome appears the most often in the sample space \[P_X(x_{\textnormal{mode}})\geq P_X(x).\]
                \item Median: The outcome which separates the higher half from  the lower half of a sample space \[P_X[X \leq x_{\textnormal{med}}] \geq 1/2, \qquad \qquad P_X[X \geq x_{\textnormal{med}}] \geq 1/2.\]
                \item (Arithmetic) mean:  The sum of all the outcomes divided by the number of outcomes \[\bar{x} = \frac{1}{n}\sum_{i=1}^{n}x_i.\]
            \end{enumerate}
            }
            \item Expectation: Weighted (Arithmetic) mean{
                \begin{enumerate}
                    \item Definition:{
                        \begin{align}
                            \mu_x = \E{X}&=\sum_{x\in S_X}xP_X(x). \tag{First Moment of $X$}\\
                            \E{X^2}&=\sum_{x\in S_X}x^2P_X(x). \tag{Second Moment of $X$}
                        \end{align}
                    }
                    \item Important Expectations{
                        \begin{enumerate}
                            \item Bernoulli (p): \[\E{X}=0\cdot P_X(0)+1\cdot P_X(1)=0(1-p)+1(p)=p.\]
                            \item Binomial (n, p): \[\E{X}=np.\]
                            \item Poisson ($\alpha$): \[\E{X}=\alpha.\]
                            \item Geometric (p): \[\E{X}=1/p.\]
                            \item Pascal (k, p): \[\E{X}=k/p.\]
                            \item Discrete Uniform (k, l): \[\E{X}=(k+l)/2.\]
                        \end{enumerate}
                    }
                \end{enumerate}
            }
            \item From an engineering perspective, \textbf{Mean (including Expectations, etc.)} is numerically easier to calculate, either using human brain or computers, than Mode and Median, when the sample space is humongous.
            \item In most cases, average, mean and expectation refer to the same concept.
        \end{enumerate}
    }
    \item Derived Random Variable: $Y = g(X)${
        \begin{enumerate}
            \item $P_Y(y) = P[Y=y] = P[Y=g(x)] = P[g^{-1}(Y)=g^{-1}(g(x))] = P[X=x] = P_X(x)$
            \item $\E{Y} = \sum yP_Y(y) = \sum g(x)P_X(x)$
            \item $\E{X-\mu_x}=\sum_{x\in S_X}(x-\mu_x)P_X(x)=\sum_{x\in S_X}xP_X(x)-\mu_x\sum_{x\in S_X}P_X(x)=\E{X}-\mu_x\cdot 1=\mu_x-\mu_x=0$
            \item $\E{aX+b}=a\E{X}+b \Rightarrow \E{b}=\E{0\cdot X+b}=b$
        \end{enumerate}
    }
    \item Variance ($\sigma_x^2$) and Standard Deviation ($\sigma_x$){
        \begin{enumerate}
            \item {
                \begin{align*}
                    \sigma_x^2
                    &= \Var[X] \\
                    &= \E{(X-\mu_x)^2} \\
                    &= \E{X^2-2\mu_x X+\mu_x^2} \\
                    &= \E{X^2}-2\mu_x\E{X}+\E{\mu_x^2} \\
                    &= \E{X^2}-2\mu_x^2+\mu_x^2 \\
                    &= \E{X^2}-\mu_x^2
                \end{align*}
            }
            \item $\Var[X]\geq 0$
            \item $\Var[aX+b]=a^2\Var[X]$
            \item Important Variance:{
                \begin{enumerate}
                    \item Bernoulli (p): \[\Var[X]=p(1-p).\]
                    \item Binomial (n, p): \[\Var[X]=np(1-p).\]
                    \item Poisson ($\alpha$): \[\Var[X]=\alpha.\]
                    \item Geometric (p): \[\Var[X]=(1-p)/p^2.\]
                    \item Pascal (k, p): \[\Var[X]=k(1-p)/p^2.\]
                    \item Discrete Uniform (k, l): \[\Var[X]=(l-k)(l-k+2)/12.\]
                \end{enumerate}
            }
        \end{enumerate}
    }
\end{enumerate}