\documentclass{article}
\usepackage[
    % letterpaper,
    margin=1in,
    % headheight=13.6pt,
]{geometry}

%%%% Page Header and Foot %%%%
\usepackage{fancyhdr}
\fancypagestyle{plain}{
\fancyhf{}
\fancyhead[L]{ECE 14:332:226}
\fancyhead[C]{\textbf{Probability And Random Processes}}
\fancyhead[R]{Summer 2023}
\fancyfoot[L]{}
\fancyfoot[C]{\thepage}
\fancyfoot[R]{}
}
\pagestyle{plain}

%%%% Load Format %%%%
%%%%%%%%%%%%%%%%%%%%%%%%%%%%%%%%%%%%%%%%%%%%%%%%%%%%%%%%%%%%%%%%%%%%%%%%%%%%%%%%%%%% Packages to Load
%% Font Packages
\RequirePackage[english]{babel} %Multilingual support
\RequirePackage[utf8]{inputenc} %utf8 support.
\RequirePackage[T1]{fontenc} %Better Font Encoding for printing
\RequirePackage{libertinus} %Popular Math Font
\RequirePackage{microtype} %Better Font Spacing.
\RequirePackage{lmodern} %Popular Font

%% Math Packages
\RequirePackage{amsfonts,amsmath,amssymb} %Math Fonts and Symbols
\RequirePackage{mathtools} %Define math symbols
\RequirePackage{mleftright}  %Fixes some annoying spacing issues
\RequirePackage{bm} %Boldface math, load after other math font packages

%% Graphics Packages
\RequirePackage{graphicx} %Graphics backend with key-value arguments
\RequirePackage[export]{adjustbox} %More arguments for includegraphics
\RequirePackage[svgnames]{xcolor} %Beamer has this by default
\RequirePackage[most]{tcolorbox} %Customized Box
\RequirePackage{transparent} %Transparent Images
\RequirePackage{subcaption} %Package to use subfigure

%% Table Packages
\RequirePackage{tabularx} %Including tabular with more space option
\RequirePackage{multirow} %Package to use multirow in tables
\RequirePackage{colortbl} %Package to use color in tables
\RequirePackage{array} %Math and more spacing support in tables

%% Theorems etc (also  problems)
\RequirePackage[shortlabels]{enumitem} %Better List
\RequirePackage{hyperref} %Beamer has this by default
\RequirePackage{cleveref} %Must load before amsthm. Conflict With Beamer
\RequirePackage{amsthm, thmtools, thm-restate} %Better Theorems

%% Code Packages
\RequirePackage{listings} %Package to insert code. Better than minted
\RequirePackage{algorithm} %Package to insert algorithm
\RequirePackage{algpseudocode} %Package to insert algorithm
% \RequirePackage{keyval}

%% Display Packages
% \RequirePackage[document]{ragged2e} %Replaced by microtype
\RequirePackage{csquotes} %Package to facilities quotations
\RequirePackage{multicol} %Package to use multicols

%% Other Packages
\RequirePackage{xspace} %Package to add space after macros
\RequirePackage{calc} %Package to calculate in Latex
\RequirePackage{rotating} %Package to rotate objects
% \RequirePackage{pgf, tikz} %Package to draw figures.
\RequirePackage{epstopdf} %Package to convert eps to pdf
\RequirePackage[backend=biber, style=numeric, sorting=none]{biblatex}
%%%% delimiters
\DeclarePairedDelimiter\parens{\lparen}{\rparen}  
\DeclarePairedDelimiter\bracks{\lbrack}{\rbrack}
\DeclarePairedDelimiter\braces{\lbrace}{\rbrace}
\DeclarePairedDelimiter\abs{\lvert}{\rvert}
\DeclarePairedDelimiter\norm{\lVert}{\rVert}
\DeclarePairedDelimiter\angles{\langle}{\rangle}
\DeclarePairedDelimiter\ceil{\lceil}{\rceil}
\DeclarePairedDelimiter\floor{\lfloor}{\rfloor}

%%%% math operators naming
\DeclareMathOperator*{\argmax}{\mathrm{argmax}}
\DeclareMathOperator*{\argmin}{\mathrm{argmin}}
\DeclareMathOperator{\tr}{\mathrm{tr}}
\DeclareMathOperator{\eig}{\mathrm{eig}}
\DeclareMathOperator{\sgn}{\mathrm{sgn}}
% \let\det\relax % "Undefine" \det
% \DeclareMathOperator{\det}{\mathrm{det}} % already defined in mathtools
\DeclareMathOperator{\diag}{\mathrm{diag}}
\DeclareMathOperator{\rank}{\mathrm{rank}}
\DeclareMathOperator{\Vol}{\mathrm{Vol}}   % volume
\DeclareMathOperator{\Surf}{\mathrm{Surf}} % surface area

%%%% Transforms! -- requires mathtools package
\newcommand*{\LapTrans}{\xleftrightarrow{\mathcal{Z}}}
\newcommand*{\ZTrans}{\xleftrightarrow{\mathcal{L}}}
\newcommand*{\CTFS}{\xleftrightarrow{\mathrm{CTFS}}}
\newcommand*{\CTFT}{\xleftrightarrow{\mathrm{CTFT}}}
\newcommand*{\DTFS}{\xleftrightarrow{\mathrm{DTFS}}}
\newcommand*{\DTFT}{\xleftrightarrow{\mathrm{DTFT}}}

%%%% vector font
\let\oldvec\vec
\renewcommand*{\vec}[1]{\mathbf{#1}}
\newcommand*{\trn}{{}^{\mkern-4mu\intercal}}
\newcommand*{\inv}{^{-1}} 

%%%% number systems
\DeclareMathOperator{\R}{\mathbb{R}}
\DeclareMathOperator{\C}{\mathbb{C}}
\DeclareMathOperator{\N}{\mathbb{N}}
\DeclareMathOperator{\Z}{\mathbb{Z}}
\DeclareMathOperator{\F}{\mathbb{F}}
\DeclareMathOperator{\Q}{\mathbb{Q}}

%%%% STATISTICS AND PROBABILITY
\newcommand*{\Var}{\mathop{\mathrm{Var}}}
\newcommand*{\Cov}{\mathop{\mathrm{Cov}}}
\newcommand*{\Corr}{\mathop{\mathrm{Corr}}}
\newcommand*{\MSE}{\mathop{\mathrm{MSE}}}

\newcommand*{\E}[1]{\mathbb{E}\bracks*{#1}}
\newcommand*{\condE}[2]{\mathbb{E}\bracks*{#1 \mid #2}}
\renewcommand*{\P}[1]{\mathbb{P}\parens*{#1}}
\newcommand*{\condP}[2]{\mathbb{P}\parens*{#1 \mid #2}}

\DeclareMathOperator{\Bern}{\mathsf{Bern}}
\DeclareMathOperator{\Unif}{\mathsf{Unif}}
\DeclareMathOperator{\Expv}{\mathsf{Exp}}
\DeclareMathOperator{\Poi}{\mathsf{Poi}}
\DeclareMathOperator{\Gamv}{\mathsf{Gamma}}
\DeclareMathOperator{\Dirv}{\mathsf{Dir}}
\DeclareMathOperator{\Mult}{\mathsf{Mult}}
\DeclareMathOperator{\Beta}{\mathsf{Beta}}
\DeclareMathOperator{\Geomv}{\mathsf{Geom}}
\DeclareMathOperator{\Binomv}{\mathsf{Binom}}
\DeclareMathOperator{\NegBinomv}{\mathsf{NB}}
\DeclareMathOperator{\Lap}{\mathsf{Lap}}
\DeclareMathOperator{\Gaus}{\mathsf{N}}
\DeclareMathOperator{\Weibull}{\mathsf{Weibull}}


\DeclareMathOperator{\iidsim}{\stackrel{\mathrm{i.i.d.}}{\sim}}
\DeclareMathOperator{\diff}{\mathop{}\!\mathrm{d}}

%%%% Special norms and linear algebra stuff
\newcommand*{\subgnorm}[1]{\norm*{#1}_{\psi_2}}
\newcommand*{\subexpnorm}[1]{\norm*{#1}_{\psi_1}}
\newcommand*{\frobnorm}[1]{\norm*{#1}_{\mathrm{F}}}
\newcommand*{\opnorm}[1]{\norm*{#1}_{\mathrm{op}}}
\newcommand*{\Lipnorm}[1]{\norm*{#1}_{\mathrm{Lip}}}

%%%%
\newcommand*{\set}[1]{\braces*{\,#1\,}}
\newcommand*{\ie}{\textnormal{i.e.\ }}
\newcommand*{\eg}{\textnormal{e.g.\ }}
\newcommand*{\etc}{\textnormal{etc.\ }}
\addbibresource{bib.bib}

%%%% Theorem Style %%%%
\declaretheorem[numbered=no, style=definition]{axiom}
\declaretheorem[numberwithin=section,style=definition]{definition}
\declaretheorem[sibling=definition]{theorem, lemma, corollary, proposition, conjecture}
\declaretheorem[numbered=no,style=remark]{remark, claim}

%%%% Document Information %%%%
\title{Project: Bertrand's Paradox}
\author{}
\date{}

\begin{document}
\maketitle

\begin{abstract}
    In the studying of continuous random variable and joint random variable, we are going to learn that the probability of an event is defined as the area under the probability density function (PDF) and the PDF is derived from Cumulated Distribution Function (CDF). Surprisingly, for the same problem, the definition of CDF can also involve randomness. In this project, we will discuss Bertrand's Paradox, which is a problem within the classical interpretation of probability theory. We will also discuss the real-world scenarios where Bertrand's Paradox might take effect.
\end{abstract}

Bertrand's Paradox is a problem within the classical interpretation of probability theory. It was proposed by Joseph Bertrand in 1889 and reveals a paradox within probability theory.

Consider an equilateral triangle inscribed in a circle. Suppose a chord of the circle is chosen at random. What is the probability that the chord is longer than a side of the triangle?

The paradox is that there are different ways to define what is meant by ``chosen at random'', and different methods can give three different answers: $1/2$, $1/3$, or $1/4$.

\begin{enumerate}
    \item Choose a random radius of the circle and then choose a random point on this radius for the midpoint of the chord. In this case, the chord will be longer than the side of the triangle when the midpoint lies in the smaller circle whose radius is half of the original circle's radius. So, the probability is $1/4$.
    \item Choose two random points on the circle's circumference, and use these as the endpoints of the chord. In this case, the chord will be longer than a side of the triangle if and only if the two points lie in the same semicircle. So, the probability is $1/2$.
    \item Choose a random point inside the circle and draw the chord with this as one endpoint and the circle center as the other endpoint. The chord will be longer than a side of the triangle when the chosen point lies within the circle inscribed in the triangle. The area of this smaller circle is $1/3$ of the area of the original circle, so the probability is $1/3$.
\end{enumerate}

Here are two real world scenarios where Bertrand's Paradox might apply:
\begin{enumerate}
    \item \textbf{Uncertainty in Simulation Modeling:} When creating a simulation model for any real-world phenomenon (like climate change, financial markets, \etc), defining what is ``random'' can have significant impacts on the results. Different interpretations of randomness can lead to vastly different simulation outcomes, much like in Bertrand's paradox.
    \item \textbf{Polling and Statistics:} Consider a company that wants to poll its customers about their satisfaction with a product. How the company defines a ``random'' sample of customers can dramatically affect the results. For instance, if they randomly select from their online customers, from customers who have bought recently, or from customers within a certain geographic region, they might get different responses in each case.
\end{enumerate}

The paradox is a classic example of the importance of clear definitions in the application of probability, as well as the potential issues with intuition and ambiguity in the interpretation of ``random''.

Here are three concepts need to be covered in the presentation:
\begin{enumerate}
    \item Explain the Bertrand's Paradox with figures and calculations.
    \item Run a simulation to verify each of the three answers (You may want to work with the team who choose the Monte Carlo).
    \item Maybe you can come up with the forth answer or more?
\end{enumerate}

\end{document}