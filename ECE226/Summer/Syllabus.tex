\documentclass{article}
\usepackage[
    % letterpaper,
    margin=1in,
    % headheight=13.6pt,
]{geometry}

%%%% Page Header and Foot %%%%
\usepackage{fancyhdr}
\fancypagestyle{plain}{
\fancyhf{}
\fancyhead[L]{ECE 14:332:226}
\fancyhead[C]{\textbf{Probability And Random Processes}}
\fancyhead[R]{Summer 2023}
\fancyfoot[L]{}
\fancyfoot[C]{\thepage}
\fancyfoot[R]{}
}
\pagestyle{plain}

%%%% Load Format %%%%
%%%% Page Margin %%%%
\RequirePackage{geometry}

%%%% Page Header and Foot %%%%
\RequirePackage{fancyhdr}
\pagestyle{fancyplain}
\lhead{\lefthead}
\chead{\centerhead}
\rhead{\righthead}
\lfoot{\leftbottom}
\cfoot{\centerbottom}
\rfoot{\rightbottom}

%%%% Additional Packages to Load %%%%
%%%% Main Font
\RequirePackage{lmodern} %Computer Modern, but with many more glyphs
\RequirePackage{microtype} %Better Font Rendering
\RequirePackage{libertinus} %Very popular math font
\RequirePackage[T1]{fontenc} %Better Printing Font
\RequirePackage[utf8]{inputenc} %enable utf8 encoding
\RequirePackage[english]{babel} %English Language

%%%% Math Env
\RequirePackage{amsfonts,amsmath,amssymb} %Math Font Packages
\RequirePackage{mathtools} %Better Math Environments
\RequirePackage{mleftright}  %Fixes some annoying spacing issues
\RequirePackage{bm} %Boldface math, load after other math font packages

%%%% Beautiful Stuff
\RequirePackage{graphicx} %Better Graphics
\RequirePackage[export]{adjustbox} %Better Graphics
\RequirePackage[dvipsnames,svgnames]{xcolor} %Must load before tcolorbox
\RequirePackage[breakable]{tcolorbox} %Customized Box
\tcbuselibrary{skins}
\RequirePackage{transparent} 

%%%% Table and List 
\RequirePackage{tabularx} %Better Table
\RequirePackage{multirow}
\RequirePackage{colortbl}
\RequirePackage{array}

%%%% Theorems etc (also  problems)
\RequirePackage[shortlabels]{enumitem} %Better List
\RequirePackage{hyperref, cleveref} %Fantastic crossrefer
\RequirePackage{amsthm, thmtools} %Better Theorems
\RequirePackage{thm-restate} %Better Theorems

%%%% Code Packages
\RequirePackage{algorithm2e} % Package to create pseudo-code
\RequirePackage{listings} % Package to insert code
% \RequirePackage{keyval}

%%%% Display Packages
\RequirePackage[document]{ragged2e} % Package to justify text
\RequirePackage{csquotes} % Package to facilities quotations
\RequirePackage{multicol} % Package to use multicols

%%%% Other Packages
\RequirePackage{calc}
\RequirePackage{rotating}
\RequirePackage{pgf, tikz}
\RequirePackage{epstopdf}
\RequirePackage[backend=biber, style=numeric, sorting=none]{biblatex}

%%%% A box fill the rest of the page
\newtcolorbox{stretchbox}[1][]{
    height fill,
    sharp corners,
    colback=white,
    % colframe=yellow!50!black,
    #1
    }

%%%% Box Environment
\newtcolorbox{prob}[1]{
% Set box style
    sidebyside,
    sidebyside align=top,
% Dimensions and layout
    width=\textwidth,
    toptitle=2.5pt,
    bottomtitle=2.5pt,
    righthand width=0.20\textwidth,
% Coloring
    colbacktitle=white,
    coltitle=black,
    colback=white,
    colframe=black,
% Title formatting
    title={
        #1 \hfill Grade:\hspace*{0.15\textwidth}
    },
    fonttitle=\large\bfseries
}

%%%% Problem Environment 
\newenvironment{problem}[1]{
    \begin{prob}{#1}
}
{
    \tcblower
    \centering
    \textit{\scriptsize\bfseries Faculty Comments}
    \end{prob}
}

%%%% Draw a line
\newcommand{\myrule}{\rule{1.5in}{0.1mm}}
%%%% delimiters
\DeclarePairedDelimiter\parens{\lparen}{\rparen}
\DeclarePairedDelimiter\bracks{\lbrack}{\rbrack}
\DeclarePairedDelimiter\braces{\lbrace}{\rbrace}
\DeclarePairedDelimiter\abs{\lvert}{\rvert}
\DeclarePairedDelimiter\norm{\lVert}{\rVert}
\DeclarePairedDelimiter\angles{\langle}{\rangle}
\DeclarePairedDelimiter\ceil{\lceil}{\rceil}
\DeclarePairedDelimiter\floor{\lfloor}{\rfloor}

%%%% math operators naming
\DeclareMathOperator*{\argmax}{\textnormal{argmax}}
\DeclareMathOperator*{\argmin}{\textnormal{argmin}}
\DeclareMathOperator{\tr}{\textnormal{tr}}
\DeclareMathOperator{\eig}{\textnormal{eig}}
\DeclareMathOperator{\sgn}{\textnormal{sgn}}
% \let\det\relax % "Undefine" \det
% \DeclareMathOperator{\det}{\textnormal{det}} % already defined in mathtools
\DeclareMathOperator{\diag}{\textnormal{diag}}
\DeclareMathOperator{\rank}{\textnormal{rank}}
\DeclareMathOperator{\Vol}{\textnormal{Vol}}   % volume
\DeclareMathOperator{\Surf}{\textnormal{Surf}} % surface area

%%%% Transforms! -- requires mathtools package
\newcommand*{\LapTrans}{\xleftrightarrow{\mathcal{Z}}}
\newcommand*{\ZTrans}{\xleftrightarrow{\mathcal{L}}}
\newcommand*{\CTFS}{\xleftrightarrow{\textnormal{CTFS}}}
\newcommand*{\CTFT}{\xleftrightarrow{\textnormal{CTFT}}}
\newcommand*{\DTFS}{\xleftrightarrow{\textnormal{DTFS}}}
\newcommand*{\DTFT}{\xleftrightarrow{\textnormal{DTFT}}}

%%%% vector font
\let\oldvec\vec
\renewcommand*{\vec}[1]{\mathbf{#1}}
% \newcommand*{\trn}{\!^{\!\intercal}}
\newcommand*{\trn}{\!^{\mathsf{T}}}
% \newcommand*{\coj}{\!^{\dag}} % Text Mode Symbol, Should not be used
% \newcommand*{\coj}{\!^{\dagger}}
\newcommand*{\coj}{\!^{\mathsf{H}}}
\newcommand*{\inv}{^{-1}}

%%%% number systems
\DeclareMathOperator{\R}{\mathbb{R}}
\DeclareMathOperator{\C}{\mathbb{C}}
\DeclareMathOperator{\N}{\mathbb{N}}
\DeclareMathOperator{\Z}{\mathbb{Z}}
\DeclareMathOperator{\F}{\mathbb{F}}
\DeclareMathOperator{\Q}{\mathbb{Q}}

%%%% STATISTICS AND PROBABILITY
\newcommand*{\Var}{\mathop{\textnormal{Var}}}
\newcommand*{\Cov}{\mathop{\textnormal{Cov}}}
\newcommand*{\Corr}{\mathop{\textnormal{Corr}}}
\newcommand*{\MSE}{\mathop{\textnormal{MSE}}}
\newcommand*{\MSD}{\mathop{\textnormal{MSD}}}
\newcommand*{\NSD}{\mathop{\textnormal{NSD}}}

\newcommand*{\E}[1]{\mathbb{E}\bracks*{#1}}
\newcommand*{\condE}[2]{\mathbb{E}\bracks*{#1 \mid #2}}
\renewcommand*{\P}[1]{\mathbb{P}\parens*{#1}}
\newcommand*{\condP}[2]{\mathbb{P}\parens*{#1 \mid #2}}

\DeclareMathOperator{\Bern}{\mathsf{Bern}}
\DeclareMathOperator{\Unif}{\mathsf{Unif}}
\DeclareMathOperator{\Expv}{\mathsf{Exp}}
\DeclareMathOperator{\Poi}{\mathsf{Poi}}
\DeclareMathOperator{\Gamv}{\mathsf{Gamma}}
\DeclareMathOperator{\Dirv}{\mathsf{Dir}}
\DeclareMathOperator{\Mult}{\mathsf{Mult}}
\DeclareMathOperator{\Beta}{\mathsf{Beta}}
\DeclareMathOperator{\Geomv}{\mathsf{Geom}}
\DeclareMathOperator{\Binomv}{\mathsf{Binom}}
\DeclareMathOperator{\NegBinomv}{\mathsf{NB}}
\DeclareMathOperator{\Lap}{\mathsf{Lap}}
\DeclareMathOperator{\Gaus}{\mathsf{N}}
\DeclareMathOperator{\Weibull}{\mathsf{Weibull}}


\DeclareMathOperator{\iidsim}{\stackrel{\textnormal{i.i.d.}}{\sim}}
\DeclareMathOperator{\diff}{\mathop{}\!\textnormal{d}}

%%%% Special norms and linear algebra stuff
\newcommand*{\subgnorm}[1]{\norm*{#1}_{\psi_2}}
\newcommand*{\subexpnorm}[1]{\norm*{#1}_{\psi_1}}
\newcommand*{\frobnorm}[1]{\norm*{#1}_{\textnormal{F}}}
\newcommand*{\opnorm}[1]{\norm*{#1}_{\textnormal{op}}}
\newcommand*{\Lipnorm}[1]{\norm*{#1}_{\textnormal{Lip}}}

%%%%
\newcommand*{\set}[1]{\braces*{\,#1\,}}
\newcommand*{\ie}{\textnormal{i.e.\ }}
\newcommand*{\eg}{\textnormal{e.g.\ }}
\newcommand*{\etc}{\textnormal{etc.\ }}
\newcommand*{\iid}{\textnormal{i.i.d.\ }}
\addbibresource{bib.bib}

%%%% Theorem Style %%%%
\declaretheorem[numbered=no, style=definition]{axiom}
\declaretheorem[numberwithin=section,style=definition]{definition}
\declaretheorem[sibling=definition]{theorem, lemma, corollary, proposition, conjecture}
\declaretheorem[numbered=no,style=remark]{remark, claim}

%%%% Document Information %%%%
\title{RUTGERS UNIVERSITY, DEPARTMENT OF ECE \\
COURSE SYLLABUS: 14:332:226}
\author{Kailong Wang}
\date{}

\begin{document}
\maketitle
Probability theory studies random phenomena in a formal mathematical way. It is essential for all engineering and scientific disciplines dealing with models that depend on chance.
Probability provides a well-defined way to quantify the uncertainty of a random event. With this framework, we can analyze the behavior of complex systems and \textbf{make informed decisions} (i.e., minimize the negative effect of bad behavior or maximize the positive effect of good behavior).
With a long history development of probability theory, it plays a central role in e.g.,  telecommunications and finance systems. Telecommunications systems strive to provide reliable and secure transmission and storage of information under the uncertainties coming from various types of random noise and adversarial behavior. Finance systems strive to maximize profits in spite of the uncertainties coming from natural and man-made events.
% With the rapid growth of AIGC, probability theory inspired the design of model learning algorithms and architectures. For example, ChatGPT is a large model which outputs result with maximized likelihood of human language habits.
The students will learn the fundamentals of probability that are necessary for several ECE courses and related fields and help them prepare for the career in the industry and academia.

\textbf{Class Time and Place:} Monday$\sim$Thursday 10:30am$\sim$12:20pm, Busch SEC-203.

\textbf{Office Hour:} TBD.

\textbf{Contact:} kw414@scarletmail.rutgers.edu.

\textbf{Prerequisites:} Calculus (Mandatory), Linear Algebra (Recommended).

\textbf{Grading:} {
    \begin{itemize}
        \item Structure: {
            \begin{itemize}
                \item HW Presentation: 20 times$\times$1 credit
                \item Exam: 2 times$\times$50 credits
                \item Project Presentation: 1 times$\times$15 credits
                \item Project appraisal: 1 times$\times$15 credits
            \end{itemize}
        }
        \item Exam Formats: {
            \begin{itemize}
                % \item In total 100 points in each exam. Each will convert to 50 credits in final grade.
                \item 40 credits questions from HW. Fill the blank.
                \item 10 credits questions from anywhere. True or False with punishment.
                % \item 50 points challenge questions. Format unknown.
            \end{itemize}
        }
    \end{itemize}
}

\textbf{Textbook and Materials:} {
    \begin{enumerate}
        \item \textbf{(For Beginner and Engineer)} \fullcite{Yates_Goodman_2015}\label{book:Yates_Goodman_2015}
        \href{https://bcs.wiley.com/he-bcs/Books?action=index&itemId=1118324560&bcsId=8677}{\textbf{Companion Site}}
        \item \textbf{(Beginner Alternative)} \fullcite{94107244}
        % \item \textbf{(Classic Alternative)} \fullcite{91610706}
        \item \textbf{(Engineer Alternative)} \href{https://www.probabilitycourse.com/}{Introduction to Probability, Statistics, and Random Processes} by Pishro-Nik
        % \item \textbf{(Modern Alternative)} \href{https://www.stat.berkeley.edu/~aldous/134/grinstead.pdf}{Introduction to Probability} by Grinstead and Snell
        % \item \textbf{(Optional)} \fullcite{93738109}
        % \item \textbf{(Advanced Choice)} \fullcite{94013254}
        \item \textbf{(Lecture Note)} \href{https://github.com/kwang0913/MyLatex/blob/main/2023ECE226Note/main.pdf}{Probability and Random Processes} by Kailong Wang
        % \item \textbf{(Path to Intermediate)} \fullcite{91978890}
        \item \textbf{(Alternative Note)} \href{https://web.ma.utexas.edu/users/gordanz/lecture_notes_page.html}{Introduction to Mathematical Statistics
        } Chapter $1\sim7$ by Gordanz
    \end{enumerate}
}

\textbf{Extended Reading:}{
    \begin{enumerate}
            \item \textbf{(Publication of Nobel Memorial Prize in Economic Sciences Recipients)} \fullcite{93150815}
            \item \textbf{(Publication of Turing Award Recipients)} \fullcite{98333950}
            \item \textbf{(Read at Your Own Risk!!!)} \fullcite{93574984}
    \end{enumerate}
}

\textbf{Topics Covered By Day (Based on Textbook \cref{book:Yates_Goodman_2015}):} {
    \begin{enumerate}[\textbf{Day} 1]
        \item Course Introduction and Review of Calculus
        \item \textbf{(Chapter 1,2)} Combinatorics, Counting Methods, Set Theory, Axioms of Probability, Venn Diagrams
        \item Examples and Exercises
        \item \textbf{(Chapter 1,2)} Conditional Probability, Bayes Theorem, Independence, Tree Diagrams
        \item Examples and Exercises
        \item \textbf{(Chapter 3)} PMF, Discrete Random Variables, CDF, Expectation
        \item Examples and Exercises
        \item \textbf{(Chapter 4)} CDF, Continuous Random Variables, PDF, Gaussian Random Variable, Delta Function
        \item Examples and Exercises
        \item \textbf{(Chapter 5)} Joint CDF, Joint Random Variables, Joint PMF, Joint PDF, Marginal PMF, Marginal PDF, Joint Expectation, Covariance, Correlation, Linear Independence
        \item Examples and Exercises
        \item Review
        \item Exam 1 (Cover Day2 to Day9)
        \item \textbf{(Chapter 7)} Conditional Probability Models
        \item Examples and Exercises
        \item Examples and Exercises
        \item \textbf{(Chapter 6,9)} Derived Random Variables, Moment Generating Function and Central Limit Theorem
        \item Examples and Exercises
        \item Introduction of Information Theory
        \item Review
        \item Exam 2 (Cover Day10 to Day17)
        \item Applications of Probability Theory in Modern Technology
        \item Final Presentation and Appraisal
    \end{enumerate}
}

\end{document}