\documentclass{article}
\usepackage[
    % letterpaper,
    margin=1in,
    % headheight=13.6pt,
]{geometry}

%%%% Page Header and Foot %%%%
\usepackage{fancyhdr}
\fancypagestyle{plain}{
\fancyhf{}
\fancyhead[L]{ECE 14:332:226}
\fancyhead[C]{\textbf{Probability And Random Processes}}
\fancyhead[R]{Summer 2023}
\fancyfoot[L]{}
\fancyfoot[C]{\thepage}
\fancyfoot[R]{}
}
\pagestyle{plain}

%%%% Load Format %%%%
%%%%%%%%%%%%%%%%%%%%%%%%%%%%%%%%%%%%%%%%%%%%%%%%%%%%%%%%%%%%%%%%%%%%%%%%%%%%%%%%%%%% Packages to Load
%% Font Packages
\RequirePackage[english]{babel} %Multilingual support
\RequirePackage[utf8]{inputenc} %utf8 support.
\RequirePackage[T1]{fontenc} %Better Font Encoding for printing
\RequirePackage{libertinus} %Popular Math Font
\RequirePackage{microtype} %Better Font Spacing.
\RequirePackage{lmodern} %Popular Font

%% Math Packages
\RequirePackage{amsfonts,amsmath,amssymb} %Math Fonts and Symbols
\RequirePackage{mathtools} %Define math symbols
\RequirePackage{mleftright}  %Fixes some annoying spacing issues
\RequirePackage{bm} %Boldface math, load after other math font packages

%% Graphics Packages
\RequirePackage{graphicx} %Graphics backend with key-value arguments
\RequirePackage[export]{adjustbox} %More arguments for includegraphics
\RequirePackage[svgnames]{xcolor} %Beamer has this by default
\RequirePackage[most]{tcolorbox} %Customized Box
\RequirePackage{transparent} %Transparent Images
\RequirePackage{subcaption} %Package to use subfigure

%% Table Packages
\RequirePackage{tabularx} %Including tabular with more space option
\RequirePackage{multirow} %Package to use multirow in tables
\RequirePackage{colortbl} %Package to use color in tables
\RequirePackage{array} %Math and more spacing support in tables

%% Theorems etc (also  problems)
\RequirePackage[shortlabels]{enumitem} %Better List
\RequirePackage{hyperref} %Beamer has this by default
\RequirePackage{cleveref} %Must load before amsthm. Conflict With Beamer
\RequirePackage{amsthm, thmtools, thm-restate} %Better Theorems

%% Code Packages
\RequirePackage{listings} %Package to insert code. Better than minted
\RequirePackage{algorithm} %Package to insert algorithm
\RequirePackage{algpseudocode} %Package to insert algorithm
% \RequirePackage{keyval}

%% Display Packages
% \RequirePackage[document]{ragged2e} %Replaced by microtype
\RequirePackage{csquotes} %Package to facilities quotations
\RequirePackage{multicol} %Package to use multicols

%% Other Packages
\RequirePackage{xspace} %Package to add space after macros
\RequirePackage{calc} %Package to calculate in Latex
\RequirePackage{rotating} %Package to rotate objects
% \RequirePackage{pgf, tikz} %Package to draw figures.
\RequirePackage{epstopdf} %Package to convert eps to pdf
\RequirePackage[backend=biber, style=numeric, sorting=none]{biblatex}
%%%% delimiters
\DeclarePairedDelimiter\parens{\lparen}{\rparen}  
\DeclarePairedDelimiter\bracks{\lbrack}{\rbrack}
\DeclarePairedDelimiter\braces{\lbrace}{\rbrace}
\DeclarePairedDelimiter\abs{\lvert}{\rvert}
\DeclarePairedDelimiter\norm{\lVert}{\rVert}
\DeclarePairedDelimiter\angles{\langle}{\rangle}
\DeclarePairedDelimiter\ceil{\lceil}{\rceil}
\DeclarePairedDelimiter\floor{\lfloor}{\rfloor}

%%%% math operators naming
\DeclareMathOperator*{\argmax}{\mathrm{argmax}}
\DeclareMathOperator*{\argmin}{\mathrm{argmin}}
\DeclareMathOperator{\tr}{\mathrm{tr}}
\DeclareMathOperator{\eig}{\mathrm{eig}}
\DeclareMathOperator{\sgn}{\mathrm{sgn}}
% \let\det\relax % "Undefine" \det
% \DeclareMathOperator{\det}{\mathrm{det}} % already defined in mathtools
\DeclareMathOperator{\diag}{\mathrm{diag}}
\DeclareMathOperator{\rank}{\mathrm{rank}}
\DeclareMathOperator{\Vol}{\mathrm{Vol}}   % volume
\DeclareMathOperator{\Surf}{\mathrm{Surf}} % surface area

%%%% Transforms! -- requires mathtools package
\newcommand*{\LapTrans}{\xleftrightarrow{\mathcal{Z}}}
\newcommand*{\ZTrans}{\xleftrightarrow{\mathcal{L}}}
\newcommand*{\CTFS}{\xleftrightarrow{\mathrm{CTFS}}}
\newcommand*{\CTFT}{\xleftrightarrow{\mathrm{CTFT}}}
\newcommand*{\DTFS}{\xleftrightarrow{\mathrm{DTFS}}}
\newcommand*{\DTFT}{\xleftrightarrow{\mathrm{DTFT}}}

%%%% vector font
\let\oldvec\vec
\renewcommand*{\vec}[1]{\mathbf{#1}}
\newcommand*{\trn}{{}^{\mkern-4mu\intercal}}
\newcommand*{\inv}{^{-1}} 

%%%% number systems
\DeclareMathOperator{\R}{\mathbb{R}}
\DeclareMathOperator{\C}{\mathbb{C}}
\DeclareMathOperator{\N}{\mathbb{N}}
\DeclareMathOperator{\Z}{\mathbb{Z}}
\DeclareMathOperator{\F}{\mathbb{F}}
\DeclareMathOperator{\Q}{\mathbb{Q}}

%%%% STATISTICS AND PROBABILITY
\newcommand*{\Var}{\mathop{\mathrm{Var}}}
\newcommand*{\Cov}{\mathop{\mathrm{Cov}}}
\newcommand*{\Corr}{\mathop{\mathrm{Corr}}}
\newcommand*{\MSE}{\mathop{\mathrm{MSE}}}

\newcommand*{\E}[1]{\mathbb{E}\bracks*{#1}}
\newcommand*{\condE}[2]{\mathbb{E}\bracks*{#1 \mid #2}}
\renewcommand*{\P}[1]{\mathbb{P}\parens*{#1}}
\newcommand*{\condP}[2]{\mathbb{P}\parens*{#1 \mid #2}}

\DeclareMathOperator{\Bern}{\mathsf{Bern}}
\DeclareMathOperator{\Unif}{\mathsf{Unif}}
\DeclareMathOperator{\Expv}{\mathsf{Exp}}
\DeclareMathOperator{\Poi}{\mathsf{Poi}}
\DeclareMathOperator{\Gamv}{\mathsf{Gamma}}
\DeclareMathOperator{\Dirv}{\mathsf{Dir}}
\DeclareMathOperator{\Mult}{\mathsf{Mult}}
\DeclareMathOperator{\Beta}{\mathsf{Beta}}
\DeclareMathOperator{\Geomv}{\mathsf{Geom}}
\DeclareMathOperator{\Binomv}{\mathsf{Binom}}
\DeclareMathOperator{\NegBinomv}{\mathsf{NB}}
\DeclareMathOperator{\Lap}{\mathsf{Lap}}
\DeclareMathOperator{\Gaus}{\mathsf{N}}
\DeclareMathOperator{\Weibull}{\mathsf{Weibull}}


\DeclareMathOperator{\iidsim}{\stackrel{\mathrm{i.i.d.}}{\sim}}
\DeclareMathOperator{\diff}{\mathop{}\!\mathrm{d}}

%%%% Special norms and linear algebra stuff
\newcommand*{\subgnorm}[1]{\norm*{#1}_{\psi_2}}
\newcommand*{\subexpnorm}[1]{\norm*{#1}_{\psi_1}}
\newcommand*{\frobnorm}[1]{\norm*{#1}_{\mathrm{F}}}
\newcommand*{\opnorm}[1]{\norm*{#1}_{\mathrm{op}}}
\newcommand*{\Lipnorm}[1]{\norm*{#1}_{\mathrm{Lip}}}

%%%%
\newcommand*{\set}[1]{\braces*{\,#1\,}}
\newcommand*{\ie}{\textnormal{i.e.\ }}
\newcommand*{\eg}{\textnormal{e.g.\ }}
\newcommand*{\etc}{\textnormal{etc.\ }}
\addbibresource{bib.bib}

%%%% Theorem Style %%%%
\declaretheorem[numbered=no, style=definition]{axiom}
\declaretheorem[numberwithin=section,style=definition]{definition}
\declaretheorem[sibling=definition]{theorem, lemma, corollary, proposition, conjecture}
\declaretheorem[numbered=no,style=remark]{remark, claim}

%%%% Document Information %%%%
\title{Project Topic: Copula}
\author{}
\date{}

\begin{document}
\maketitle
\begin{abstract}
    In this course, we are going to study the joint distribution of random variables. Under the condition of independence, it is easy to achieve the joint distribution by the product of marginal distributions. However, in the case of dependence, it is not easy to achieve the joint distribution. In this project, we are going to study the copula, which is a tool to construct the joint distribution of random variables with given marginal distributions.
\end{abstract}

A copula is a mathematical function that joins or ``couple'' multiple univariate distributions to form a multivariate distribution. The primary use of the copula is to model and understand the \textbf{dependence} structure between different variables in a multivariate distribution. It allows you to specify the marginal distributions (the distributions of each variable individually) and the dependence separately.

Copula has widely been used in many fields, including finance, medicine, and hydrology. For example,

\begin{enumerate}
    \item \textbf{Finance and Risk Management:} In financial risk management, copulas are used to understand and model the correlation structure between different financial assets or market risk factors. For instance, they can be used in portfolio risk management to understand how different assets in a portfolio might jointly react to market shocks. This information is critical in constructing efficient portfolios and understanding systemic risk. After 2008, some people argue that the financial crisis was caused by the misuse of copulas.
    \item \textbf{Medicine and Health Studies:} In medical statistics, copulas are often used to model the dependence between different health-related variables. For instance, they can be used to study the joint distribution of different biomarkers in a patient to understand the overall disease risk better.
\end{enumerate}

It's important to note that while copulas offer flexibility in modeling multivariate distributions, choosing an appropriate copula and correctly estimating its parameters can be challenging. Furthermore, copulas often assume that the dependence structure remains constant over time, which may not hold in many practical applications. As with all statistical models, care must be taken in their application and interpretation.

In this project, the three concepts should be covered:
\begin{enumerate}
    \item The definition of copula and its properties (\eg strength and limitation).
    \item An example of using copula to construct the joint distribution of random variables (require some coding and simulation).
    \item An example of the failure of misusing copula (It would be better if you could show some numerical analysis results).
\end{enumerate}

\end{document}