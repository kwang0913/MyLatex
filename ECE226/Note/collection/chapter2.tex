\chapter{Conditional Probability and Independent}

\section{Conditional Probability}
\begin{theorem}
    [Conditional Probability]
    If $\P{B}>0$, then
    \[\condP{A}{B} = \frac{\P{AB}}{\P{B}}.\]
\end{theorem}
\begin{enumerate}
    \item If $A_i$ are Mutually Exclusive: $\condP{\cup_{i=1}^{N}A_i}{B} = \sum_{i=1}^{N}\condP{A_i}{B}$.
    \item If $A_i$ are Collectively Exhaustive: $\condP{\cup_{i=1}^N A_i}{B}=\condP{\Omega}{B}=\P{\Omega B}/\P{B}=1$.
    \item If $A_i$ are partitions: $\sum_{i=1}^{N}\condP{A_i}{B}=\condP{\cup_{i=1}^{N}A_i}{B}=\condP{\Omega}{B}=\P{\Omega B}/\P{B}=1$.
    \item If $B_i$ are Mutually Exclusive: \[\condP{A}{\cup_{i=1}^{N}B_i} = \P{A\cap(\cup_{i=1}^N B_i)}/\P{\cup_{i=1}^N B_i}=\sum_{i=1}^{N}\P{AB_i}/\sum_{i=1}^N\P{B_i}.\]
    \item If $B_i$ are Collectively Exhaustive: $\condP{A}{\cup_{i=1}^{N}B_i} = \condP{A}{\Omega}=\P{A\Omega}/\P{\Omega}=\P{A}$
    \item If $B_i$ are Partitions \textbf{(Law of Total Number)},
    \begin{align}
        \condP{A}{B}
        &= \sum_{i=1}^{N}\P{AB_i}/\sum_{i=1}^N\P{B_i} \tag{$B_i$ are Mutually Exclusive} \\
        &= \sum_{i=1}^{N}\P{AB_i}  \tag{$B_i$ are Collectively Exhaustive} \\
        &= \sum_{i=1}^{N}\condP{A}{B_i}\P{B_i}. \tag{Definition of Conditional Probability}
    \end{align}
\end{enumerate}

\begin{theorem}
    [$\P{\cdot\mid F}$ is a Probability]
    The conditional probability satisfies the axioms of probability.
    \begin{enumerate}
        \item $0\leq\condP{E}{F}\leq 1$.
        \item $\condP{\Omega}{F}=1$.
        \item If $E_i$ are mutually exclusive, then $\condP{\cup_{i=1}^{N}E_i}{F}=\sum_{i=1}^{N}\condP{E_i}{F}$.
    \end{enumerate}
\end{theorem}

\begin{theorem}
    [Bayes' Theorem]
    If $\P{A}>0$ and $\P{B}>0$, then
    \begin{equation*}
        \condP{A}{B} = \frac{\condP{B}{A}\P{A}}{\P{B}}.
    \end{equation*}
\end{theorem}

\section{Independent}
\begin{definition}
    [Independent]
    $A$ and $B$ are independent if and only if $\P{AB}=\P{A}\P{B}$.
\end{definition}

\begin{theorem}
    If $A$ and $B$ are independent, then
    \begin{enumerate}
        \item $\P{A \cup B} = \P{A}+\P{B}-\P{A}\P{B}$.
        \item $\condP{A}{B} = \frac{\P{AB}}{\P{B}}=\frac{\P{A}\P{B}}{\P{B}}=\P{A}$. \label{item:independent}
        \item If $A$ and $B$ are independent then $A^{\mathsf{c}}$ and $B$ are independent and so on.{
            \begin{proof}
                \begin{align*}
                    \P{B} = \P{(A\cup A^{\mathsf{c}})\cap B}
                    &= \P{AB}+\P{A^{\mathsf{c}}B}  \tag{$A$ and $A^\mathsf{c}$ are partitions}\\
                    \Rightarrow \P{A^{\mathsf{c}}B}
                    &= \P{B}-\P{AB} \\
                    &= \P{B}-\P{A}\P{B}  \tag{$A$ and $B$ are independent}\\
                    &= \P{B}\left(1-\P{A}\right) \\
                    &= \P{B}\P{A^{\mathsf{c}}}.
                \end{align*}
            \end{proof}
        }
    \end{enumerate}
\end{theorem}

\begin{theorem}
    [Independent Trails]
    The Probability of $k_0$ failures and $k_1$ successes in $n=k_0+k_1$ Independent Trails with success rate $p$ is \[\binom{n}{k_0}(1-p)^{k0}p^{k_1}=\binom{n}{k_1}(1-p)^{k0}p^{k_1}.\]
    If $n=k_1+k_2+\cdots+k_m$ and success rates are $p_1, p_2,\ldots,p_m$, where $\sum_{i=1}^{m}p_i=1$, the probability of such independent trails is
    \[\binom{n}{k_1,k_2,\ldots,k_m}p_1^{k_1}p_2^{k_2}\ldots p_m^{k^m}.\]
\end{theorem}

\section{Tree Diagram}