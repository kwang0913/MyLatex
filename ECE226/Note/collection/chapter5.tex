\chapter{Joint Random Variables}

\section{Joint CDF}
\begin{definition}[Joint CDF]
    The joint CDF of random variables $X$ and $Y$ is
    \[F_{X,Y}(x,y) = \P{X\leq x, Y\leq y}.\]
\end{definition}

The joint CDF is a \textbf{complete} probability model for any pair of random variables $X$ and $Y$.

\begin{theorem}
    For any pair of random variables, $X$ and $Y$, the following properties hold:
    \begin{enumerate}
        \item $0 \leq F_{X,Y}(x,y) \leq 1$,
        \item $F_{X,Y}(\infty,\infty)=1$,
        \item $F_{X,Y}(-\infty,y)=F_{X,Y}(x,-\infty)=0$,
        \item $F_{X,Y}(x,y)$ is non-decreasing in $x$ and $y$.
    \end{enumerate}
\end{theorem}

\begin{definition}
    [Marginal CDF]
    \[F_X(x)=F_{X,Y}(x,\infty) \qquad F_Y(y)=F_{X,Y}(\infty,y)\].
\end{definition}

\section{Joint PMF}
\begin{definition}[Joint PMF]
    The joint PMF of random variables $X$ and $Y$ is
    \[P_{X,Y}(x,y) = \P{X=x, Y=y}.\]
\end{definition}

The joint PMF is a \textbf{complete} probability model for any pair of discrete random variables $X$ and $Y$.

\begin{theorem}
    For discrete random variables $X$ and $Y$ and any set $B$ in the $X$, $Y$ plane, the probability of the event
    % $\set{(X,Y)\in B}$
    is
    \[\P{\set{B}}=\sum_{(x,y)\in B}P_{X,Y}(x,y).\]
\end{theorem}

Apparently, the joint PMF is non-negative and sums to one.
\[\sum_{x\in S_X}\sum_{y\in S_Y}P_{X,Y}(x,y)=1.\]

\begin{definition}
    [Marginal PMF]
    For discrete random variables $X$ and $Y$ with joint \textnormal{PMF} $P_{X,Y}(x,y)$,
    \[P_X(x)=\sum_{y\in S_Y}P_{X,Y}(x,y), \qquad P_Y(y)=\sum_{x\in S_X}P_{X,Y}(x,y).\]
\end{definition}
For discrete random variables, the marginal PMF $P_X(x)$ and $P_Y(y)$ are probability models for the individual random variables $X$ and $Y$, but they only provide an \textbf{incomplete} probability model for the pair of random variables $X$ and $Y$.

\section{Joint PDF}
\begin{definition}[Joint PDF]
    The joint CDF of continuous random variables $X$ and $Y$ is a function $f_{X,Y}(x,y)$ with the property
    \[F_{X,Y}(u,v)=\int_{-\infty}^{u}\int_{-\infty}^{v}f_{X,Y}(x,y)\diff x \diff y.\]
\end{definition}

Apparently, we can then derive the joint PDF as follows,
\[f_{X,Y}(x,y)=\frac{\partial^2}{\partial x \partial y}F_{X,Y}(x,y).\]
The joint PDF is a \textbf{complete} probability model for any pair of continuous random variables $X$ and $Y$.

\begin{theorem}
    The probability that the continuous random variables $(X,Y)$ are in any set $A$
    \[\P{\set{A}}=\iint\limits_A f_{X,Y}(x,y)\diff x\diff y.\]
\end{theorem}

The joint PDF is non-negative and integrates to one.
\[\int_{-\infty}^{\infty}\int_{-\infty}^{\infty}f_{X,Y}(x,y)\diff x\diff y=1.\]

\begin{definition}
    [Marginal PDF]\label{def:marginal_pdf}
    For continuous random variables $X$ and $Y$ with joint \textnormal{PDF} $f_{X,Y}(x,y)$,
    \[f_X(x)=\int_{-\infty}^{\infty}f_{X,Y}(x,y)\diff y, \qquad f_Y(y)=\int_{-\infty}^{\infty}f_{X,Y}(x,y)\diff x.\]
\end{definition}
For continuous random variables, the marginal PDFs $f_X(x)$ and $f_Y(y)$ are probability models for the individual random variables $X$ and $Y$, but they only provide an \textbf{incomplete} probability model for the pair of random variables $X$ and $Y$.

\section{Expected Value}
\begin{theorem}[Expected Value of a Function of Two Random Variables]
    The function of two random variables (\ie, $g(X,Y)$) is also a random variable, then the expected value of $g(X,Y)$ is
    \begin{align}
        \E{g(X,Y)} &= \sum_{x\in S_X}\sum_{y\in S_Y}g(x,y)P_{X,Y}(x,y); \tag{Discrete}\\
        \E{g(X,Y)} &= \int_{-\infty}^{\infty}\int_{-\infty}^{\infty}g(x,y)f_{X,Y}(x,y)\diff x\diff y. \tag{Continuous}
    \end{align}
\end{theorem}

\begin{theorem}\label{thm:linearity}
    The expectation of a linear combination of several functions can be easily derived as follows,
    \[\E{\sum_{i=1}^{n}a_i g_i(X,Y)}=\sum_{i=1}^{n}a_i\E{g_i(X,Y)}.\]
\end{theorem}

\begin{corollary}
    [Sum of two random variables]\label{cor:linearity}
    For random variables $X$ and $Y$,
    \[\E{X+Y}=\E{X}+\E{Y}.\]
\end{corollary}
\textbf{Note:} The \cref{cor:linearity} states that the expectation of sum of random variables does not need the joint probability model of all random variables. However, this is not true for the variance.

\begin{corollary}
    [The variance of the sum of two random variables]\label{thm:variance_of_sum}
    \[\Var[aX+bY]=a^2\Var[X]+b^2\Var[Y]+2ab\E{(X-\mu_X)(Y-\mu_Y)}.\]
\end{corollary}

% \textcolor{red}{In the following contents, we will assume random variables (e.g., $X$) as a function of random variables.}

\section{Covariance, Correlation and Independent}
\begin{definition}[Covariance]
    The covariance of two random variables $X$ and $Y$ is
    \[\sigma_{xy}=\Cov[X,Y]=\E{(X-\E{X})(Y-\E{Y})}=\E{XY}-\E{X}\E{Y}.\]
\end{definition}

\begin{definition}[Correlation Coefficient]
    The correlation coefficient of two random variables $X$ and $Y$ is
    \[\rho_{xy}=\Corr[X,Y]=\frac{\Cov[X,Y]}{\sqrt{\Var[X]\Var[Y]}}=\frac{\sigma_{xy}}{\sigma_x\sigma_y}.\]
\end{definition}
\textbf{Note:} Correlation coefficient is a dimensionless quantity. There are other definitions of correlation coefficient, but this is the most common one.

\begin{theorem}
    If $X$ and $Y$ are random variables such that $Y=aX+b$,
    \[\rho_{X,Y}={
        \begin{cases}
            -1, & a<0\\
            0, & a=0\\
            1, & a>0
        \end{cases}
    }\]
    which also implies
    \[-1\leq \rho_{xy}\leq 1.\]
\end{theorem}

\begin{definition}
    [Correlation]
    The correlation of random variables $X$ and $Y$ is
    \[r_{X,Y} = \E{XY}.\]
    This is a different parameter from the correlation coefficient (\textcolor{red}{and widely used in engineer major}).
\end{definition}

\begin{definition}
    [Uncorrelatedness]
    Random variables $X$ and $Y$ are uncorrelated if and only if
    \[\Cov[X,Y]=0.\]
\end{definition}

\begin{definition}[Independence]
    Random variables $X$ and $Y$ are independent if and only if
    \begin{align}
        P_{X,Y}(x,y) &=P_X(x)P_Y(y); \tag{Discrete}\\
        f_{X,Y}(x,y) &=f_X(x)f_Y(y). \tag{Continuous}
    \end{align}
\end{definition}
It's easy to show that if $X$ and $Y$ are independent, then
\[F_{X,Y}(x,y)=\P{X\leq x, Y\leq y}=\P{X\leq x}\P{Y\leq y}=F_X(x)F_Y(y).\]

\begin{theorem}
    If Random variables $X$ and $Y$ are independent
    \begin{enumerate}
        \item $\E{XY}=\E{X}\E{Y}$, so that $\Cov[X, Y]=0$,
        \item $\Var[aX+bY]=a^2\Var[X]+b^2\Var[Y]$.
    \end{enumerate}
\end{theorem}

\begin{definition}
    [Orthogonality]
    If $r_{X,Y}=\E{XY}=0$, then $X$ and $Y$ are orthogonal.
\end{definition}

\begin{theorem}
    The relationship between uncorrelatedness, independence and orthogonality
    \begin{enumerate}
        \item Uncorrelatedness means changing the value of one random variable does not affect the \textbf{mean} of the other random variable.
        \item Independence means changing the value of one random variable does not affect the probability distribution (\ie, mean, variance and other moments as well) of the other random variable.
        \item Uncorrelated is linear independent. But independent includes both linear and nonlinear independent. Thus, uncorrelatedness does not imply independence. \eg, $X\sim \Unif[-1,1]$ and $Y=X^2$.
        \item Orthogonality is a different concept from uncorrelatedness and independence. From the perspective of Linear Algebra, orthogonality is a concept regarding angle between random variables, while independence and uncorrelatedness are concepts regarding the length (or projection) of random variables.
        \item Independence and Uncorrelatedness are preferred for easier calculation of the variance of linear combination of random variables.
        \item Joint Gaussian Random Variables has a preferred property: uncorrelatedness $\iff$ independence. This is the fundamental of a lot of (I would say more than 85\%) stochastic modeling, machine learning, etc.
    \end{enumerate}
\end{theorem}


\section{Bivariate Gaussian Random Variables}
\begin{definition}
    [Bivariate Gaussian Random Variables]
    Random variables $X$ and $Y$ are bivariate Gaussian if and only if their joint PDF is given by the following equation,
    \[f_{X,Y}(x,y)=\frac{1}{2\pi\sigma_x\sigma_y\sqrt{1-\rho^2}}\exp\left\{-\frac{1}{2(1-\rho^2)}\left[\left(\frac{x-\mu_x}{\sigma_x}\right)^2-2\rho\left(\frac{x-\mu_x}{\sigma_x}\right)\left(\frac{y-\mu_y}{\sigma_y}\right)+\left(\frac{y-\mu_y}{\sigma_y}\right)^2\right]\right\}.\]
    Where $\mu_x=\E{X}$, $\mu_y=\E{Y}$, $\sigma_x^2=\Var[X]$, $\sigma_y^2=\Var[Y]$, and $\rho=\Corr[X,Y]$.
\end{definition}

\begin{theorem}
    If $X$ and $Y$ are bivariate Gaussian, then $X$ is the $\mathsf{N}(\mu_X, \sigma_X)$ and $Y$ is the $\mathsf{N}(\mu_Y, \sigma_Y)$.
\end{theorem}

\begin{theorem}
    Bivariate Gaussian Random Variables are uncorrelated if and only if they are independent.
\end{theorem}

\begin{theorem}
    If $X$ and $Y$ are bivariate Gaussian, and $W_1$ and $W_2$ are given by linear independent equations,
    \[W_1=a_1X+b_1Y, \qquad W_2=a_2X+b_2Y,\]
    then $W_1$ and $W_2$ are bivariate Gaussian random variables such that
    \begin{align*}
        \E{W_i} &= a_i\mu_X+b_i\mu_Y, \\
        \Var[W_i] &= a_i^2\sigma_X^2+b_i^2\sigma_Y^2+2a_ib_i\sigma_X\sigma_Y\rho, \\
        \Cov[W_1,W_2] &= a_1a_2\sigma_X^2+b_1b_2\sigma_Y^2+(a_1b_2+a_2b_1)\sigma_X\sigma_Y\rho.
    \end{align*}
\end{theorem}

% \section{Multivariate Probability Model}
% \begin{definition}
%     [Multivariate Joint CDF]
%     The Joint CDF of $n$ random variables $X_1, X_2, \dots, X_n$ is
%     \[F_{X_1,X_2,\dots,X_n}(x_1,x_2,\dots,x_n)=\P{X_1\leq x_1, X_2\leq x_2, \dots, X_n\leq x_n}.\]
% \end{definition}

% \begin{definition}
%     [Multivariate Joint PMF]
%     The Joint PMF of $n$ discrete random variables $X_1, X_2, \dots, X_n$ is
%     \[P_{X_1,X_2,\dots,X_n}(x_1,x_2,\dots,x_n)=\P{X_1=x_1, X_2=x_2, \dots, X_n=x_n}.\]
% \end{definition}

% \begin{definition}
%     [Multivariate Joint PDF]
%     The Joint PDF of $n$ continuous random variables $X_1, X_2, \dots, X_n$ is
%     \[f_{X_1,X_2,\dots,X_n}(x_1,x_2,\dots,x_n) = \frac{\partial^n F_{X_1,X_2,\dots,X_n}(x_1,x_2,\dots,x_n)}{\partial x_1 \partial x_2 \cdots \partial x_n}.\]
% \end{definition}

% \begin{theorem} The Multivariate Random Variables have the following properties:
%     \begin{enumerate}
%         \item $P_{X_1,X_2,\dots,X_n}(x_1,x_2,\dots,x_n)\geq 0$,
%         \item $\sum_{x_1\in S_{X_1}}\sum_{x_2\in S_{X_2}}\cdots\sum_{x_n\in S_{X_n}}P_{X_1,X_2,\dots,X_n}(x_1,x_2,\dots,x_n)=1$,
%         \item $f_{X_1,X_2,\dots,X_n}(x_1,x_2,\dots,x_n)\geq 0$,
%         \item $\int_{-\infty}^{\infty}\int_{-\infty}^{\infty}\cdots\int_{-\infty}^{\infty}f_{X_1,X_2,\dots,X_n}(x_1,x_2,\dots,x_n)\diff x_1\diff x_2\cdots\diff x_n=1$.
%         \item $F_{X_1,X_2,\dots,X_n}(u_1,u_2,\dots,u_n)= \int_{-\infty}^{u_1}\int_{-\infty}^{u_2}\cdots\int_{-\infty}^{u_n}f_{X_1,X_2,\dots,X_n}(x_1,x_2,\dots,x_n)\diff x_1\diff x_2\cdots\diff x_n$.
%     \end{enumerate}
% \end{theorem}

% \begin{theorem}
%     The probability of an event $A$ expressed in terms of the random variables $X_1, X_2, \dots, X_n$ is
%     \begin{align*}
%         &\text{Discrete:}& \P{\set{A}} &= \sum_{(x_1,x_2,\dots,x_n)\in A}P_{X_1,X_2,\dots,X_n}(x_1,x_2,\dots,x_n); \\
%         &\text{Continuous:}& \P{\set{A}} &= \int_{(x_1,x_2,\dots,x_n)\in A}f_{X_1,X_2,\dots,X_n}(x_1,x_2,\dots,x_n)\diff x_1\diff x_2\cdots\diff x_n.
%     \end{align*}
% \end{theorem}

% \begin{theorem}
%     The marginal \textnormal{PMF/PDF} of $X_1, X_2, \dots, X_n$ is
%     \begin{align*}
%         &\text{Discrete:}& P_{X_1}(x_1) &= \sum_{x_2\in S_{X_2}}\cdots\sum_{x_n\in S_{X_n}}P_{X_1,X_2,\dots,X_n}(x_1,x_2,\dots,x_n); \\
%         &\text{Continuous:}& f_{X_1}(x_1) &= \int_{-\infty}^{\infty}\int_{-\infty}^{\infty}\cdots\int_{-\infty}^{\infty}f_{X_1,X_2,\dots,X_n}(x_1,x_2,\dots,x_n)\diff x_2\cdots\diff x_n.
%     \end{align*}
% \end{theorem}

% \begin{theorem}
%     [N Independent Random Variables] If $X_1, X_2, \dots, X_n$ are independent, then
%     \begin{align*}
%         &\text{Discrete:}& P_{X_1,X_2,\dots,X_n}(x_1,x_2,\dots,x_n) &= \prod_{i=1}^{n}P_{X_i}(x_i); \\
%         &\text{Continuous:}& f_{X_1,X_2,\dots,X_n}(x_1,x_2,\dots,x_n) &= \prod_{i=1}^{n}f_{X_i}(x_i).
%     \end{align*}
% \end{theorem}

% \begin{theorem}
%     [N Independent and Identically Distributed Random Variables] If $X_1, X_2, \dots, X_n$ are independent and identically distributed (\iid), then
%     \begin{align*}
%         &\text{Discrete:}& P_{X_1,X_2,\dots,X_n}(x_1,x_2,\dots,x_n) &= \prod_{i=1}^{n}P_{X}(x_i); \\
%         &\text{Continuous:}& f_{X_1,X_2,\dots,X_n}(x_1,x_2,\dots,x_n) &= \prod_{i=1}^{n}f_{X}(x_i).
%     \end{align*}
% \end{theorem}