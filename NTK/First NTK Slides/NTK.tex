\documentclass[10pt]{../Formats/RU}
%%%% delimiters
\DeclarePairedDelimiter\parens{\lparen}{\rparen}
\DeclarePairedDelimiter\bracks{\lbrack}{\rbrack}
\DeclarePairedDelimiter\braces{\lbrace}{\rbrace}
\DeclarePairedDelimiter\abs{\lvert}{\rvert}
\DeclarePairedDelimiter\norm{\lVert}{\rVert}
\DeclarePairedDelimiter\angles{\langle}{\rangle}
\DeclarePairedDelimiter\ceil{\lceil}{\rceil}
\DeclarePairedDelimiter\floor{\lfloor}{\rfloor}

%%%% math operators naming
\DeclareMathOperator*{\argmax}{\textnormal{argmax}}
\DeclareMathOperator*{\argmin}{\textnormal{argmin}}
\DeclareMathOperator{\tr}{\textnormal{tr}}
\DeclareMathOperator{\eig}{\textnormal{eig}}
\DeclareMathOperator{\sgn}{\textnormal{sgn}}
% \let\det\relax % "Undefine" \det
% \DeclareMathOperator{\det}{\textnormal{det}} % already defined in mathtools
\DeclareMathOperator{\diag}{\textnormal{diag}}
\DeclareMathOperator{\rank}{\textnormal{rank}}
\DeclareMathOperator{\Vol}{\textnormal{Vol}}   % volume
\DeclareMathOperator{\Surf}{\textnormal{Surf}} % surface area

%%%% Transforms! -- requires mathtools package
\newcommand*{\LapTrans}{\xleftrightarrow{\mathcal{Z}}}
\newcommand*{\ZTrans}{\xleftrightarrow{\mathcal{L}}}
\newcommand*{\CTFS}{\xleftrightarrow{\textnormal{CTFS}}}
\newcommand*{\CTFT}{\xleftrightarrow{\textnormal{CTFT}}}
\newcommand*{\DTFS}{\xleftrightarrow{\textnormal{DTFS}}}
\newcommand*{\DTFT}{\xleftrightarrow{\textnormal{DTFT}}}

%%%% vector font
\let\oldvec\vec
\renewcommand*{\vec}[1]{\mathbf{#1}}
% \newcommand*{\trn}{\!^{\!\intercal}}
\newcommand*{\trn}{\!^{\mathsf{T}}}
% \newcommand*{\coj}{\!^{\dag}} % Text Mode Symbol, Should not be used
% \newcommand*{\coj}{\!^{\dagger}}
\newcommand*{\coj}{\!^{\mathsf{H}}}
\newcommand*{\inv}{^{-1}}

%%%% number systems
\DeclareMathOperator{\R}{\mathbb{R}}
\DeclareMathOperator{\C}{\mathbb{C}}
\DeclareMathOperator{\N}{\mathbb{N}}
\DeclareMathOperator{\Z}{\mathbb{Z}}
\DeclareMathOperator{\F}{\mathbb{F}}
\DeclareMathOperator{\Q}{\mathbb{Q}}

%%%% STATISTICS AND PROBABILITY
\newcommand*{\Var}{\mathop{\textnormal{Var}}}
\newcommand*{\Cov}{\mathop{\textnormal{Cov}}}
\newcommand*{\Corr}{\mathop{\textnormal{Corr}}}
\newcommand*{\MSE}{\mathop{\textnormal{MSE}}}
\newcommand*{\MSD}{\mathop{\textnormal{MSD}}}
\newcommand*{\NSD}{\mathop{\textnormal{NSD}}}

\newcommand*{\E}[1]{\mathbb{E}\bracks*{#1}}
\newcommand*{\condE}[2]{\mathbb{E}\bracks*{#1 \mid #2}}
\renewcommand*{\P}[1]{\mathbb{P}\parens*{#1}}
\newcommand*{\condP}[2]{\mathbb{P}\parens*{#1 \mid #2}}

\DeclareMathOperator{\Bern}{\mathsf{Bern}}
\DeclareMathOperator{\Unif}{\mathsf{Unif}}
\DeclareMathOperator{\Expv}{\mathsf{Exp}}
\DeclareMathOperator{\Poi}{\mathsf{Poi}}
\DeclareMathOperator{\Gamv}{\mathsf{Gamma}}
\DeclareMathOperator{\Dirv}{\mathsf{Dir}}
\DeclareMathOperator{\Mult}{\mathsf{Mult}}
\DeclareMathOperator{\Beta}{\mathsf{Beta}}
\DeclareMathOperator{\Geomv}{\mathsf{Geom}}
\DeclareMathOperator{\Binomv}{\mathsf{Binom}}
\DeclareMathOperator{\NegBinomv}{\mathsf{NB}}
\DeclareMathOperator{\Lap}{\mathsf{Lap}}
\DeclareMathOperator{\Gaus}{\mathsf{N}}
\DeclareMathOperator{\Weibull}{\mathsf{Weibull}}


\DeclareMathOperator{\iidsim}{\stackrel{\textnormal{i.i.d.}}{\sim}}
\DeclareMathOperator{\diff}{\mathop{}\!\textnormal{d}}

%%%% Special norms and linear algebra stuff
\newcommand*{\subgnorm}[1]{\norm*{#1}_{\psi_2}}
\newcommand*{\subexpnorm}[1]{\norm*{#1}_{\psi_1}}
\newcommand*{\frobnorm}[1]{\norm*{#1}_{\textnormal{F}}}
\newcommand*{\opnorm}[1]{\norm*{#1}_{\textnormal{op}}}
\newcommand*{\Lipnorm}[1]{\norm*{#1}_{\textnormal{Lip}}}

%%%%
\newcommand*{\set}[1]{\braces*{\,#1\,}}
\newcommand*{\ie}{\textnormal{i.e.\ }}
\newcommand*{\eg}{\textnormal{e.g.\ }}
\newcommand*{\etc}{\textnormal{etc.\ }}
\newcommand*{\iid}{\textnormal{i.i.d.\ }}
\graphicspath{{../Formats}}

%Theorem style
\declaretheorem[numbered=no, style=definition]{axiom}
\declaretheorem[numberwithin=section,style=definition]{definition}
\declaretheorem[sibling=definition]{theorem, lemma, corollary, proposition, conjecture}
\declaretheorem[numbered=no,style=remark]{remark, claim}

%Information to be included in the title page:
\title[Neural Tangent Kernel]{Neural Tangent Kernel: Convergence and Generalization in Neural Networks~\footcite{Jacot2020}
}
\author[Kai] % (optional, for multiple authors)
{Kailong Wang\inst{$\dagger$}
%\and Someone Else\inst{2}
}
\institute[Rutgers] % (optional)
{
    \inst{$\dagger$}%
    % Ph.D.\ of ECE\\
    Rutgers University
    % \and
    % \inst{2}%
    % Faculty of Statistics\\
    % Very Famous University
}
\date[\today] % (optional)
{
    % ECE 539 HDP,
    \today
}

%Bib Resource
\addbibresource{../../../Markdown/Paper/My Library.bib}

%Document begins
\begin{document}
\frame{\titlepage}

\begin{frame}
    \frametitle{On the equivalence of Neural Network and Kernel Machine}
    \begin{itemize}
        \item On the equivalence at initialization---Infinity width Multi-Layer Perceptron (MLP) initialize with Gaussian weights and biases is equivalent to a Gaussian Process.~\footcite{Lee2018}
        \item On the equivalence at training---Infinity width MLP with ReLU activation under squared loss trained by gradient descent is equivalent to a Gaussian Process.~\footcite{Jacot2020}
        \item On the equivalence at training---Infinite width MLP with ReLU activation under soft margin loss trained by subgradient descent is equivalent to a support vector machine.~\footcite{Chen2021}
    \end{itemize}
\end{frame}

\begin{frame}
\frametitle{Table of Contents}
\tableofcontents
\end{frame}


\section{Notations}
\begin{frame}
    {The Architecture of a Multi-Layer Perceptron (MLP)}
    \includegraphics[width=\textwidth, center]{simple_nn.png}
\end{frame}

\begin{frame}
    \frametitle{Notations---Parameters Space}
    \begin{itemize}
        \item $L$: the depth of the network (\aka the number of layers). \Eg $0$ is the input layer, $L$ is the output layer.
        \item $N^{(L)}$: the width of the network (\aka the number of neurons in each layer). \Eg $N^{(0)}$ is the dimension of the input, $N^{(L)}$ is the dimension of the output.
        \item $P$: the total number of parameters, which is $\sum_{l=1}^L N^{(l-1)}N^{(l)}$.
        \item The parameters between each layer are denoted as $W^{(l)}\in\R^{N^{(l-1)}\times N^{(l)}}$ and $b^{(l)}\in\R^{N^{(l)}}$, which are initialized as \iid $\Gaus(0,1)$.
    \end{itemize}
\end{frame}

\begin{frame}
    \frametitle{Notations---Function Space}
    \begin{itemize}
        \item $\sigma$: the activation function. \Eg $\sigma(x) = \max\{0, x\}$ is the ReLU.
        \item $\mathcal{F}$: the function space of the MLP. \Eg $\mathcal{F} = \{f:\R^{N^{(0)}}\to\R^{N^{(L)}}\}$.
        \item $\tilde{f}_{\theta}^{(l)}(x):\R^{(l-1)}\rightarrow\R^{(l)}$: the pre-activations
        \item $f_{\theta}^{(l)}(x):\R^{(l)}\rightarrow\R^{(l)}$: the post-activations
        \item Clearly, we have $\theta\in\R^P$ and {
            \begin{align*}
                f_{\theta}^{(0)}(\vec{x}) &= \vec{x} \\
                \tilde{f}_{\theta}^{(l)}(\vec{x}) &= W^{(l)}\trn f_{\theta}^{(l-1)}(\vec{x}) + b^{(l)} \\
                f_{\theta}^{(l)}(\vec{x}) &= \sigma(\tilde{f}_{\theta}^{(l)}(\vec{x}))
            \end{align*}
        }
    \end{itemize}
\end{frame}

\section{Neural Tangent Kernel (NTK)}
\begin{frame}
    \frametitle{Gradient Descent---Parameter Space}
    For a task with input $x\in\R^{n\times N^{(0)}}$ and output $y\in\R^{n\times N^{(L)}}$, the total loss $\mathcal{L}$ is defined with per-sample loss $\ell$
    \[\mathcal{L}(\theta)=\argmin_{\theta} \frac{1}{n}\sum_{i=1}^{n}\ell(\mathcal{F}(\vec{x}_i), y_i).\]
    According to the chain rule, the gradient of the loss is
    \[\triangledown_{\theta}\mathcal{L}(\theta)=\frac{1}{n}\sum_{i=1}^{n}\triangledown_{\theta}\mathcal{F}(\vec{x}_i)\triangledown_{\mathcal{F}}\ell(\mathcal{F},y_i).\]
    If we let the gradient descent update the parameters with infinitisimal step size, then the gradient descent can be approximated as a derivative on the time dimension $t$
    \[\frac{\diff \theta}{\diff t} = -\triangledown_{\theta}\mathcal{L}(\theta)=-\frac{1}{n}\sum_{i=1}^{n}\triangledown_{\theta}\mathcal{F}(\vec{x}_i)\triangledown_{\mathcal{F}}\ell(\mathcal{F},y_i).\]
\end{frame}

\begin{frame}
    \frametitle{Gradient Descent---Function Space}
    Then for any pair $(\vec{x^*},\vec{x}_i)$, the output of the network slowly evolves as
    \[\frac{\diff \mathcal{F}(\vec{x^*})}{\diff t}=\frac{\diff \mathcal{F}(\vec{x^*})}{\diff \theta}\frac{\diff \theta}{\diff t}=-\frac{1}{n}\sum_{i=1}^{n}\triangledown_{\theta}\mathcal{F}(\vec{x^*})\trn\triangledown_{\theta}\mathcal{F}(\vec{x}_i)\triangledown_{\mathcal{F}}\ell(\mathcal{F},y_i).\]
    And the inner product of the gradient defines the Neural Tangent Kernel (NTK), which is also called kernel gradient
    \[K(\vec{x^*},x_i) = \triangledown_{\theta}\mathcal{F}(\vec{x^*})\trn\triangledown_{\theta}\mathcal{F}(\vec{x}_i)\]
    With random features~\footcite{Rahimia} of $p$ (where $p\ll P$) samples of function space $\mathcal{F}_k^{p}$,
    % the paper claims
    these functions define a linear parametrization: $\R^P\rightarrow\mathcal{F}$ and
    the NTK can be approximated as
    \[K(\vec{x^*},\vec{x}_i) \approx \E{f_k^{p}(\vec{x^*})\trn f_k^{p}(\vec{x}_i)}.\]
\end{frame}

\begin{frame}
    \frametitle{The good and the bad}
    Recall the knowledge of Convex Optimization, the gradient descent converges to the global minimum of the convex loss function. For NTK,
    \begin{itemize}
        \item The Good: For the MLP trained by gradient descent, the network output slowly evolves along the (negative) kernel gradient with respect to the neural tangent kernel (NTK) during training.
        \item The Bad: The MLP (not to mention other NN architectures) is usually non-convex and the linear parametrization does not exist. Which leads to unstable NTK at initialization and during training.
    \end{itemize}
    Question: Can we prove the NTK is deterministic and staying constant, for example in an asymptotic manner?

\end{frame}

\section{Bridge the Gap}
\begin{frame}
    \frametitle{NTK becomes deterministic at initialization~\footcite{Lee2018}}
    \begin{alertblock}
        {Theorem: Neural Network as Gaussian Process}
        For a network of depth $L$ at initialization, with a Lipschitz nonlinearity $\sigma$, and in the limit as the layers width $n_1, n_2, \ldots, n_L \rightarrow\infty$, the NTK $\Theta^{(L)}$ converges in probability to a deterministic limiting kernel:
        \[\Theta^{(L)}\rightarrow\Theta^{(L)}_{\infty}\otimes I_{N_L}.\]
        The scalar kernel $\Theta^{(L)}_{\infty}:\R^{N_0}\times\R^{N_0}\rightarrow\R$ is defined recursively as
        \begin{align*}
            \Theta^{(1)}_{\infty}(\vec{x^*},\vec{x}_i) &= \Sigma^{(1)}_{\infty}(\vec{x^*},\vec{x}_i) \\
            \Theta^{(L+1)}_{\infty}(\vec{x^*},\vec{x}_i) &= \Theta^{(L+1)}_{\infty}(\vec{x^*},\vec{x}_i)\dot{\Sigma}^{(L+1)}_{\infty}(\vec{x^*},\vec{x}_i)+\Sigma^{(L+1)}_{\infty}(\vec{x^*},\vec{x}_i) \\
            \dot{\Sigma}^{(L+1)}_{\infty}(\vec{x^*},\vec{x}_i)&=\mathbb{E}_{f\sim\Gaus(0,\Sigma^{(L)})}\bracks*{\dot{\sigma}(f(\vec{x^*}))\dot{\sigma}(f(\vec{x}_i))}
        \end{align*}
    \end{alertblock}
\end{frame}

\begin{frame}
    {The NTK is constant during training}
    Previously, we have shown that (omit the inputs for simplicity)
    \[\frac{\diff \mathcal{F}_{\theta}}{\diff t}=-\alpha\mathbf{K}_{\infty}\triangledown_{f}\mathcal{L}.\]
    To track the evolution of $\theta$ in time, we can consider it as a function of time $t$. With Taylor expansion, we have~\footcite{Lee2020}
    \begin{align*}
        \mathcal{F}_{\theta}(t)&=\mathcal{F}_{\theta}(0)+\triangledown_{\theta}\mathcal{F}_{\theta}(0)\bracks*{\theta(t)-\theta(0)} \\
        \theta(t)-\theta(0)&=-\alpha\triangledown_{\theta}\mathcal{F}_{\theta}\trn\triangledown_{f}\mathcal{L} \\
        \mathcal{F}_{\theta}(t)-\mathcal{F}_{\theta}(0)
        &=-\alpha\triangledown_{\theta}\mathcal{F}_{\theta}(0)\trn\triangledown_{\theta}\mathcal{F}(\mtx{X};\theta(0))\triangledown_f\mathcal{L} \\
        \frac{\diff \mathcal{F}_{\theta}(t)}{\diff t}
        &= -\alpha K(\theta(0))\triangledown_f\mathcal{L} \\
        &=-\alpha\mathbf{K}_{\infty}\triangledown_{f}\mathcal{L}
    \end{align*}
    % This shows the same NTK is preserved during training.~\footcite{Lee2020}
\end{frame}

\begin{frame}
    {The light that is Shed by the NTK}
    \begin{itemize}
        \item With the equivalence of the NTK and the Gaussian Process, Bayesian Inference can be applied to explain the generalization of the neural network.
        \item For the same reason, the computation technique of Bayesian Inference (\eg MCMC) can be applied to the neural network.
        \item Since NTK is the gradient defined on the function space, calculus of variations can be applied to the neural network.
        \item Other analytic tools such as Spectrum Analysis or Random Matrix can be applied to find tighter bounds.
    \end{itemize}
\end{frame}

\begin{frame}
    {The Undiscovered Area}
    \begin{itemize}
        \item Empirical results shows that the Neural Network optimized by NTK does not achieve State-of-the-Art (SOTA) performance.
        \item The condition of the equivalence of NTK and Gaussian Process is too strong, which might not be practical.
        \item The NTK of other architectures (\eg CNN, RNN) is not well studied (Has been majorly solved in 2023).
    \end{itemize}
\end{frame}
% \begin{frame}
%     \frametitle{Reference}
%     \printbibliography
% \end{frame}

\end{document}