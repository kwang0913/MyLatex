\documentclass{article}
\usepackage[
    % letterpaper,
    margin=1in,
    % headheight=13.6pt,
]{geometry}

%%%% Page Header and Foot %%%%
\usepackage{fancyhdr}
\fancypagestyle{plain}{
\fancyhf{}
\fancyhead[L]{}
\fancyhead[C]{}
\fancyhead[R]{}
\fancyfoot[L]{}
\fancyfoot[C]{\thepage}
\fancyfoot[R]{}
}
\pagestyle{plain}

%%%% Load Format %%%%
\input{../../Formats/format.tex}
%%%% delimiters
\DeclarePairedDelimiter\parens{\lparen}{\rparen}
\DeclarePairedDelimiter\bracks{\lbrack}{\rbrack}
\DeclarePairedDelimiter\braces{\lbrace}{\rbrace}
\DeclarePairedDelimiter\abs{\lvert}{\rvert}
\DeclarePairedDelimiter\norm{\lVert}{\rVert}
\DeclarePairedDelimiter\angles{\langle}{\rangle}
\DeclarePairedDelimiter\ceil{\lceil}{\rceil}
\DeclarePairedDelimiter\floor{\lfloor}{\rfloor}

%%%% math operators naming
\DeclareMathOperator*{\argmax}{\textnormal{argmax}}
\DeclareMathOperator*{\argmin}{\textnormal{argmin}}
\DeclareMathOperator{\tr}{\textnormal{tr}}
\DeclareMathOperator{\eig}{\textnormal{eig}}
\DeclareMathOperator{\sgn}{\textnormal{sgn}}
% \let\det\relax % "Undefine" \det
% \DeclareMathOperator{\det}{\textnormal{det}} % already defined in mathtools
\DeclareMathOperator{\diag}{\textnormal{diag}}
\DeclareMathOperator{\rank}{\textnormal{rank}}
\DeclareMathOperator{\Vol}{\textnormal{Vol}}   % volume
\DeclareMathOperator{\Surf}{\textnormal{Surf}} % surface area

%%%% Transforms! -- requires mathtools package
\newcommand*{\LapTrans}{\xleftrightarrow{\mathcal{Z}}}
\newcommand*{\ZTrans}{\xleftrightarrow{\mathcal{L}}}
\newcommand*{\CTFS}{\xleftrightarrow{\textnormal{CTFS}}}
\newcommand*{\CTFT}{\xleftrightarrow{\textnormal{CTFT}}}
\newcommand*{\DTFS}{\xleftrightarrow{\textnormal{DTFS}}}
\newcommand*{\DTFT}{\xleftrightarrow{\textnormal{DTFT}}}

%%%% vector font
\let\oldvec\vec
\renewcommand*{\vec}[1]{\mathbf{#1}}
% \newcommand*{\trn}{\!^{\!\intercal}}
\newcommand*{\trn}{\!^{\mathsf{T}}}
% \newcommand*{\coj}{\!^{\dag}} % Text Mode Symbol, Should not be used
% \newcommand*{\coj}{\!^{\dagger}}
\newcommand*{\coj}{\!^{\mathsf{H}}}
\newcommand*{\inv}{^{-1}}

%%%% number systems
\DeclareMathOperator{\R}{\mathbb{R}}
\DeclareMathOperator{\C}{\mathbb{C}}
\DeclareMathOperator{\N}{\mathbb{N}}
\DeclareMathOperator{\Z}{\mathbb{Z}}
\DeclareMathOperator{\F}{\mathbb{F}}
\DeclareMathOperator{\Q}{\mathbb{Q}}

%%%% STATISTICS AND PROBABILITY
\newcommand*{\Var}{\mathop{\textnormal{Var}}}
\newcommand*{\Cov}{\mathop{\textnormal{Cov}}}
\newcommand*{\Corr}{\mathop{\textnormal{Corr}}}
\newcommand*{\MSE}{\mathop{\textnormal{MSE}}}
\newcommand*{\MSD}{\mathop{\textnormal{MSD}}}
\newcommand*{\NSD}{\mathop{\textnormal{NSD}}}

\newcommand*{\E}[1]{\mathbb{E}\bracks*{#1}}
\newcommand*{\condE}[2]{\mathbb{E}\bracks*{#1 \mid #2}}
\renewcommand*{\P}[1]{\mathbb{P}\parens*{#1}}
\newcommand*{\condP}[2]{\mathbb{P}\parens*{#1 \mid #2}}

\DeclareMathOperator{\Bern}{\mathsf{Bern}}
\DeclareMathOperator{\Unif}{\mathsf{Unif}}
\DeclareMathOperator{\Expv}{\mathsf{Exp}}
\DeclareMathOperator{\Poi}{\mathsf{Poi}}
\DeclareMathOperator{\Gamv}{\mathsf{Gamma}}
\DeclareMathOperator{\Dirv}{\mathsf{Dir}}
\DeclareMathOperator{\Mult}{\mathsf{Mult}}
\DeclareMathOperator{\Beta}{\mathsf{Beta}}
\DeclareMathOperator{\Geomv}{\mathsf{Geom}}
\DeclareMathOperator{\Binomv}{\mathsf{Binom}}
\DeclareMathOperator{\NegBinomv}{\mathsf{NB}}
\DeclareMathOperator{\Lap}{\mathsf{Lap}}
\DeclareMathOperator{\Gaus}{\mathsf{N}}
\DeclareMathOperator{\Weibull}{\mathsf{Weibull}}


\DeclareMathOperator{\iidsim}{\stackrel{\textnormal{i.i.d.}}{\sim}}
\DeclareMathOperator{\diff}{\mathop{}\!\textnormal{d}}

%%%% Special norms and linear algebra stuff
\newcommand*{\subgnorm}[1]{\norm*{#1}_{\psi_2}}
\newcommand*{\subexpnorm}[1]{\norm*{#1}_{\psi_1}}
\newcommand*{\frobnorm}[1]{\norm*{#1}_{\textnormal{F}}}
\newcommand*{\opnorm}[1]{\norm*{#1}_{\textnormal{op}}}
\newcommand*{\Lipnorm}[1]{\norm*{#1}_{\textnormal{Lip}}}

%%%%
\newcommand*{\set}[1]{\braces*{\,#1\,}}
\newcommand*{\ie}{\textnormal{i.e.\ }}
\newcommand*{\eg}{\textnormal{e.g.\ }}
\newcommand*{\etc}{\textnormal{etc.\ }}
\newcommand*{\iid}{\textnormal{i.i.d.\ }}
% \allowpagebreaks
%%%% Document Information %%%%
\title{Literature Review}
\author{Kailong Wang}
\date{}

\addbibresource{../../../Markdown/Paper/My Library.bib}

%%%% Start Document %%%%
\begin{document}
\maketitle

\section{Parameters}
\begin{itemize}
    \item $p_i$: Number of Parameters
    \item $m$: Number of Samples
\end{itemize}

\section{Naming}
\begin{itemize}
    \item HDP: High Dimensional Probability
    \item NTK: Neural Tangent Kernel
    \item MLP: Multi-Layer Perceptron, \aka Fully Connected (FC) Neural Network
    \item ResNet: Residual Neural Network, MLP with skip connections
\end{itemize}

\section{Convergence and Generalization of Wide Neural Network}

\begin{table}[H]
\centering
\begin{tabular}{|p{0.1\textwidth}|p{0.6\textwidth}|p{0.3\textwidth}|}
\hline
Paper & Key Result & Condition Setup \\
\hline
\cite{Allen-Zhu}        & Convergence in Polynomial Time with Polynomial Size of Data & Two or Three Layer MLP with SGD \\
\hline
\cite{Bietti}& ReLU networks is not Lipschitz but holds weaker H\"{o}lder smoothness           & CNN or MLP with ReLU activation \\
\hline
% \cite{Cao}& Generalization Bound of wide and deep Network derived with NTK &  \\
% \hline
\cite{Nitanda}        & Tighter and less conditioned bound. & 2 layer MLP under less over-parameterized condition, ReLU, SGD \\
\hline
\cite{Arora2019}        & Analyze CNN with NTK and an algorithm designed for CNN inspired by NTK. & MLP, CNN, ReLU, SGD \\
\hline
\end{tabular}
\end{table}

\section{Convergence and Generalization of Deep Neural Network}

\begin{table}[H]
\centering
\begin{tabular}{|p{0.1\textwidth}|p{0.6\textwidth}|p{0.3\textwidth}|}
\hline
Paper & Key Result & Condition Setup \\
\hline
\cite{Hayoua,Hayou,Littwin2020,Huang2020}        & NTK does not work under finite width infinity depth NN architecture. However, NTK does work under ResNet architecture. Initialization matters more for deep network. & MLP, ResNet, ReLU, SGD \\
\hline
\cite{Huang}        & An ordinary differential equations point of view & CNN, ReLU, SGD \\
\hline
\cite{Lee}        &  & MLP, CNN, ReLU, SGD \\
\hline
\end{tabular}
\end{table}

\section{Convergence and Generalization of Wide Deep Neural Network}

\begin{table}[H]
\centering
\begin{tabular}{|p{0.1\textwidth}|p{0.6\textwidth}|p{0.3\textwidth}|}
\hline
Paper & Key Result & Condition Setup \\
\hline
\cite{Adlam}        & Triple Descent Phenomenon & Single Layer MLP, $p_1 \ll m \ll p_2 \ll m^2 \ll p_3$ \\
\hline
\cite{Hanin2020,Cao}        & When both depth and width are finite, the convergence and generalization is determined by the ratio of $p/m$ &  \\
\hline
\end{tabular}
\end{table}

\section{Spectral Analysis of NTK}

\begin{table}[H]
\centering
\begin{tabular}{|p{0.1\textwidth}|p{0.6\textwidth}|p{0.3\textwidth}|}
\hline
Paper & Key Result & Condition Setup \\
\hline
\cite{Belfer}        & The Spectrum of ResNTK shows stable frequencies while FC-NTK has spike frequencies & The eigenfunctions of ResNTK are (scaled) spherical harmonics and that its eigenvalues decay with frequency $k$ at the rate of $k^{-p}$. \\
\hline
\cite{Bordelon}            & NTK fail to learn high spectral modes unless the sample size $p$ is sufficiently large  &\\
\hline
\cite{Karp}        & NTK fail to filter out high frequency noise and thus fail to reconstruct the signal, while CNN shows robust performance under such a case. & MLP, CNN, ReLU, SGD \\
\hline
\cite{Tancik2020,Yang2022}        & RFF, NTK and Neural Value Approximation & one line code to help RFF and NTK capture high frequency feature \\
\hline
\cite{Karakida2020}        & FIM, eigenspace, etc &  \\
\hline
\end{tabular}
\end{table}

A common agreement is that the NTK fail to capture high frequency feature at the initial setup.

\section{Researches inspired by NTK}

\begin{table}[H]
\centering
\begin{tabular}{|p{0.1\textwidth}|p{0.6\textwidth}|p{0.3\textwidth}|}
\hline
Paper & Key Result & Condition Setup \\
\hline
\cite{Cai,Jacot2020b}        & A second order optimization method to train NN inspired by NTK &  \\
\hline
\cite{Damian}            & In the transfer learning setup, the model convergence does not depends on input dimension but the feature dimension.  &\\
\hline
\cite{Deng}            & Using NTK analyzes the generalization bound of robust optimization &\\
\hline
\cite{Lisicki,Zhou,Kassraie2022,Jia2022}        & NTK enables Gaussian Process and Bayesian Inference in designing Neural Network based bandit algorithm. & contextual bandit algorithm, reinforcement learning \\
\hline
\cite{Long}        & How the intermediate parameter looks like after train the Neural Network with NTK. &  \\
\hline
\cite{Lou}        & Explains the feature representation capability with NTK. & MLP, ReLU, SGD \\
\hline
\cite{Tirer}        & Explain the superior performance of ResNet over MLP by analyzing NTK. & MLP, ResNet, ReLU, SGD \\
\hline
\cite{Wei}        & NTK under $L_2$ regularizer. & MLP, ReLU, SGD \\
\hline
\cite{Wu}        & NTK for Domain Adaptation. & \\
\hline
\cite{Goumiri2020}        & Reinforcement with NTK and Gaussian Process & \\
\hline
\cite{Chen2021a,Geifman2020}        & NTK is similar to the Laplace Kernel &  \\
\hline
\end{tabular}
\end{table}

\section{Other Kernel}

\begin{table}[H]
\centering
\begin{tabular}{|p{0.1\textwidth}|p{0.6\textwidth}|p{0.3\textwidth}|}
\hline
Paper & Key Result & Condition Setup \\
\hline
\cite{Han}        & A new type of random features as a kernel function &  \\
\hline
\cite{Shankar}            & Composition Kernel  &\\
\hline
\cite{Woodruff}        & Another sampling based kernel function &  \\
\hline
\end{tabular}
\end{table}

\section{Proofs of NTK}

\begin{table}[H]
\centering
\begin{tabular}{|p{0.1\textwidth}|p{0.6\textwidth}|p{0.3\textwidth}|}
\hline
Paper & Key Result & Condition Setup \\
\hline
\cite{Jacot2020}        & The first proof, a function space perspective &  \\
\hline
\cite{Simon}            & Reverse Proof of NTK &\\
\hline
\cite{Lee2020}        & Parameter Space Perspective &  \\
\hline
\cite{Xu}        & Random Kernel Function converges to NTK in expectation with high Probability. & MLP, ReLU, SGD \\
\hline
\cite{Chizat2020,Geiger2020,Ghorbani}        & Under infinity width architecture, NN's parameters almost remaining unchanging during training. This is called lazy training. &  \\
\hline
\end{tabular}
\end{table}

\clearpage
\printbibliography
\end{document}
