\section*{Chapter 2: Sequential Experiments}
\begin{enumerate}
    \item Tree Diagrams
    \item Counting Methods (\textbf{Essentially the outcomes in each experiment (i.e., sample space) are equiprobable})
    \begin{enumerate}
        \item Multiplication:~$n\times k_1\times k_2\times \ldots$
        \item Sampling without Replacement
        \begin{enumerate}
            \item Permutation:~$\frac{n!}{(n-k)!}$.
            \item Combination:~$\binom{n}{k}=\frac{n!}{k!(n-k)!}=\binom{n}{n-k}$.
            \item Combination is Permutation without order. Combination is also called n choose k.
        \end{enumerate}
        \item Sampling with Replacement:~$n^k$
        \item Multiple Combination:{
            \begin{enumerate}
                \item $\binom{n}{k_1,k_2,\ldots,k_m}=\frac{n!}{k_1!k_2!\ldots k_m!}$ where $n=\sum_{i=1}^{m}k_i$.
                \item For the two cases situation, $n=k_1+k_2\Rightarrow \binom{n}{k_1k_2}=\frac{n!}{k_1!k_2!}\iff\binom{n}{k_1}\iff\binom{n}{k_2}$.
            \end{enumerate}
            }
    \end{enumerate}
    \item Independent Trails (\textbf{Essentially the outcomes in each sample space are not necessarily equiprobable})
    \begin{enumerate}
        \item \textit{Theorem 2.8:} The Probability of $k_0$ failures and $k_1$ successes in $n=k_0+k_1$ Independent Trails with success rate $p$ is \[\P{k_0, k_1}=\binom{n}{k_0}(1-p)^{k0}p^{k_1}=\binom{n}{k_1}(1-p)^{k0}p^{k_1}.\]
        \item \textit{Theorem 2.9:} $n=k_1+k_2+\ldots+k_m$ and success rates are $p_1, p_2,\ldots,p_m$, where $\sum_{i=1}^{m}p_i=1$ has 
        \[\P{k_1,k_2,\ldots,k_m}=\binom{n}{k_1,k_2,\ldots,k_m}p_1^{k_1}p_2^{k_2}\ldots p_m^{k^m}.\]
        % \item If the outcomes in each sample space is equiprobable, the question can be solved by Sampling with Replacement.
    \end{enumerate}
\end{enumerate}