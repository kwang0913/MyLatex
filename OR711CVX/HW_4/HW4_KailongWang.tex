\documentclass{article}
\usepackage[
    letterpaper,
    top=1in,
    bottom=1in,
    inner=0.75in,
    outer=0.75in,
    % margin=1in,
]{geometry}

%%%% Page Header and Foot %%%%
\usepackage{fancyhdr}
\fancypagestyle{plain}{
\fancyhf{}
\fancyhead[L]{266:711:685}
\fancyhead[C]{\textbf{Convex Analysis and Optimization}}
\fancyhead[R]{Rutgers University}
\fancyfoot[L]{}
\fancyfoot[C]{\thepage}
\fancyfoot[R]{}
}
\pagestyle{plain}

%%%% Load Format %%%%
%%%%%%%%%%%%%%%%%%%%%%%%%%%%%%%%%%%%%%%%%%%%%%%%%%%%%%%%%%%%%%%%%%%%%%%%%%%%%%%%%%%% Packages to Load
%% Font Packages
\RequirePackage[english]{babel} %Multilingual support
\RequirePackage[utf8]{inputenc} %utf8 support.
\RequirePackage[T1]{fontenc} %Better Font Encoding for printing
\RequirePackage{libertinus} %Popular Math Font
\RequirePackage{microtype} %Better Font Spacing.
\RequirePackage{lmodern} %Popular Font

%% Math Packages
\RequirePackage{amsfonts,amsmath,amssymb} %Math Fonts and Symbols
\RequirePackage{mathtools} %Define math symbols
\RequirePackage{mleftright}  %Fixes some annoying spacing issues
\RequirePackage{bm} %Boldface math, load after other math font packages

%% Graphics Packages
\RequirePackage{graphicx} %Graphics backend with key-value arguments
\RequirePackage[export]{adjustbox} %More arguments for includegraphics
\RequirePackage[svgnames]{xcolor} %Beamer has this by default
\RequirePackage[most]{tcolorbox} %Customized Box
\RequirePackage{transparent} %Transparent Images
\RequirePackage{subcaption} %Package to use subfigure

%% Table Packages
\RequirePackage{tabularx} %Including tabular with more space option
\RequirePackage{multirow} %Package to use multirow in tables
\RequirePackage{colortbl} %Package to use color in tables
\RequirePackage{array} %Math and more spacing support in tables

%% Theorems etc (also  problems)
\RequirePackage[shortlabels]{enumitem} %Better List
\RequirePackage{hyperref} %Beamer has this by default
\RequirePackage{cleveref} %Must load before amsthm. Conflict With Beamer
\RequirePackage{amsthm, thmtools, thm-restate} %Better Theorems

%% Code Packages
\RequirePackage{listings} %Package to insert code. Better than minted
\RequirePackage{algorithm} %Package to insert algorithm
\RequirePackage{algpseudocode} %Package to insert algorithm
% \RequirePackage{keyval}

%% Display Packages
% \RequirePackage[document]{ragged2e} %Replaced by microtype
\RequirePackage{csquotes} %Package to facilities quotations
\RequirePackage{multicol} %Package to use multicols

%% Other Packages
\RequirePackage{xspace} %Package to add space after macros
\RequirePackage{calc} %Package to calculate in Latex
\RequirePackage{rotating} %Package to rotate objects
% \RequirePackage{pgf, tikz} %Package to draw figures.
\RequirePackage{epstopdf} %Package to convert eps to pdf
\RequirePackage[backend=biber, style=numeric, sorting=none]{biblatex}
%%%% delimiters
\DeclarePairedDelimiter\parens{\lparen}{\rparen}  
\DeclarePairedDelimiter\bracks{\lbrack}{\rbrack}
\DeclarePairedDelimiter\braces{\lbrace}{\rbrace}
\DeclarePairedDelimiter\abs{\lvert}{\rvert}
\DeclarePairedDelimiter\norm{\lVert}{\rVert}
\DeclarePairedDelimiter\angles{\langle}{\rangle}
\DeclarePairedDelimiter\ceil{\lceil}{\rceil}
\DeclarePairedDelimiter\floor{\lfloor}{\rfloor}

%%%% math operators naming
\DeclareMathOperator*{\argmax}{\mathrm{argmax}}
\DeclareMathOperator*{\argmin}{\mathrm{argmin}}
\DeclareMathOperator{\tr}{\mathrm{tr}}
\DeclareMathOperator{\eig}{\mathrm{eig}}
\DeclareMathOperator{\sgn}{\mathrm{sgn}}
% \let\det\relax % "Undefine" \det
% \DeclareMathOperator{\det}{\mathrm{det}} % already defined in mathtools
\DeclareMathOperator{\diag}{\mathrm{diag}}
\DeclareMathOperator{\rank}{\mathrm{rank}}
\DeclareMathOperator{\Vol}{\mathrm{Vol}}   % volume
\DeclareMathOperator{\Surf}{\mathrm{Surf}} % surface area

%%%% Transforms! -- requires mathtools package
\newcommand*{\LapTrans}{\xleftrightarrow{\mathcal{Z}}}
\newcommand*{\ZTrans}{\xleftrightarrow{\mathcal{L}}}
\newcommand*{\CTFS}{\xleftrightarrow{\mathrm{CTFS}}}
\newcommand*{\CTFT}{\xleftrightarrow{\mathrm{CTFT}}}
\newcommand*{\DTFS}{\xleftrightarrow{\mathrm{DTFS}}}
\newcommand*{\DTFT}{\xleftrightarrow{\mathrm{DTFT}}}

%%%% vector font
\let\oldvec\vec
\renewcommand*{\vec}[1]{\mathbf{#1}}
\newcommand*{\trn}{{}^{\mkern-4mu\intercal}}
\newcommand*{\inv}{^{-1}} 

%%%% number systems
\DeclareMathOperator{\R}{\mathbb{R}}
\DeclareMathOperator{\C}{\mathbb{C}}
\DeclareMathOperator{\N}{\mathbb{N}}
\DeclareMathOperator{\Z}{\mathbb{Z}}
\DeclareMathOperator{\F}{\mathbb{F}}
\DeclareMathOperator{\Q}{\mathbb{Q}}

%%%% STATISTICS AND PROBABILITY
\newcommand*{\Var}{\mathop{\mathrm{Var}}}
\newcommand*{\Cov}{\mathop{\mathrm{Cov}}}
\newcommand*{\Corr}{\mathop{\mathrm{Corr}}}
\newcommand*{\MSE}{\mathop{\mathrm{MSE}}}

\newcommand*{\E}[1]{\mathbb{E}\bracks*{#1}}
\newcommand*{\condE}[2]{\mathbb{E}\bracks*{#1 \mid #2}}
\renewcommand*{\P}[1]{\mathbb{P}\parens*{#1}}
\newcommand*{\condP}[2]{\mathbb{P}\parens*{#1 \mid #2}}

\DeclareMathOperator{\Bern}{\mathsf{Bern}}
\DeclareMathOperator{\Unif}{\mathsf{Unif}}
\DeclareMathOperator{\Expv}{\mathsf{Exp}}
\DeclareMathOperator{\Poi}{\mathsf{Poi}}
\DeclareMathOperator{\Gamv}{\mathsf{Gamma}}
\DeclareMathOperator{\Dirv}{\mathsf{Dir}}
\DeclareMathOperator{\Mult}{\mathsf{Mult}}
\DeclareMathOperator{\Beta}{\mathsf{Beta}}
\DeclareMathOperator{\Geomv}{\mathsf{Geom}}
\DeclareMathOperator{\Binomv}{\mathsf{Binom}}
\DeclareMathOperator{\NegBinomv}{\mathsf{NB}}
\DeclareMathOperator{\Lap}{\mathsf{Lap}}
\DeclareMathOperator{\Gaus}{\mathsf{N}}
\DeclareMathOperator{\Weibull}{\mathsf{Weibull}}


\DeclareMathOperator{\iidsim}{\stackrel{\mathrm{i.i.d.}}{\sim}}
\DeclareMathOperator{\diff}{\mathop{}\!\mathrm{d}}

%%%% Special norms and linear algebra stuff
\newcommand*{\subgnorm}[1]{\norm*{#1}_{\psi_2}}
\newcommand*{\subexpnorm}[1]{\norm*{#1}_{\psi_1}}
\newcommand*{\frobnorm}[1]{\norm*{#1}_{\mathrm{F}}}
\newcommand*{\opnorm}[1]{\norm*{#1}_{\mathrm{op}}}
\newcommand*{\Lipnorm}[1]{\norm*{#1}_{\mathrm{Lip}}}

%%%%
\newcommand*{\set}[1]{\braces*{\,#1\,}}
\newcommand*{\ie}{\textnormal{i.e.\ }}
\newcommand*{\eg}{\textnormal{e.g.\ }}
\newcommand*{\etc}{\textnormal{etc.\ }}
\allowdisplaybreaks

%%%% Theorem Style %%%%
\declaretheorem[numbered=no, style=plain]{axiom, lemma}
\declaretheorem[numberwithin=section,style=definition]{definition}
\declaretheorem[sibling=definition]{theorem, corollary, proposition, conjecture}
\declaretheorem[numbered=no,style=remark]{remark, claim}

%%%% Document Information %%%%
\title{Homework 4}
\author{Kailong Wang}
\date{\today}

%%%% Start Document %%%%
\begin{document}
\maketitle

\begin{problem}
    {Q1}
    Recall That $N_C$ denotes the normal cone map of the set $C$. Show that if $U$ is a linear subspace of $\R^n$, then $N_U(x)=U^{\bot}$ for all $x\in U$, where $U^{\bot}$ denotes the subspace orthogonal to $U$ (by definition, $N_U(x)=\emptyset$ if $x\notin U$).
\end{problem}

\begin{solution}
    {Solution}
    \begin{proof}
        To show that $N_U(x)=U^{\bot}$, we need to show that $N_U(x)\subseteq U^{\bot}$ and $U^{\bot}\subseteq N_U(x)$.
        \begin{itemize}
            \item $N_U(x)\subseteq U^{\bot}$: {
                    Let $y\in N_U(x)$, then we have $y\trn(x-u)\leq 0$ for all $u\in U$. Since $U$ is a linear subspace, we have $0\in U$. Thus, $y\trn(x-0)\leq 0$, which implies $y\trn x\leq 0$. Since $y\trn x\leq 0$ for all $y\in N_U(x)$, we have $x\in U^{\bot}$. Thus, $N_U(x)\subseteq U^{\bot}$.
            }
            \item $U^{\bot}\subseteq N_U(x)$: {
                    Let $y\in U^{\bot}$, then we have $y\trn u=0$ for all $u\in U$. Since $U$ is a linear subspace, we have $x-u\in U$. Thus, $y\trn(x-u)\leq 0$ for all $u\in U$. Thus, $y\in N_U(x)$. Thus, $U^{\bot}\subseteq N_U(x)$.
            }
        \end{itemize}
    \end{proof}
\end{solution}

\begin{problem}
    {Q2}
    In the proof of the existence of subgradients and of the Rockafellar-Moreau theorem, we used portions of the following result: for a proper convex function $f:\R^n\rightarrow \R\cup\set{+\infty}$, one has \[\ri\epi f=\set{(x,z)\mid x\in\ri\dom f, z>f(x)}.\] In this problem, we will prove this result, using the prolongation principle. Let $R$ denote the set on the right-hand side of the above equation. Note that you can use some form of the prolongation principle in each of the three parts of this question.
    \begin{enumerate}[(a)]
        \item Show that for any $x\in\dom f$, then $(x,f(x))$ cannot be in $\ri\epi f$.
        \item Show that a point $(x,z)\in\epi f$ that has $x\notin \ri\dom f$ cannot be in $\ri\epi f\subseteq \R$. Together with the previous result, this allows us to conclude that $\ri\epi f\subseteq \R$.
        \item Show that any $(x,z)\in \R$ is also in $\ri\epi f$, and hence, in view of the previous results, that $\ri\epi f= \R$. This may be done by showing that for any $(x',z')\in\epi f$, there exists $\delta>0$ such that $(x,z)+\delta((x,z)-(x',z'))\in\epi f$. Hint: you should need to use another fact we proved earlier, that a convex function is continuous relative to $\dom f$ at all points of $\ri\dom f$, that is, if $x\in \ri\dom f$, then for any $\tau>0$, there exists an $\epsilon>0$ such that $x'\in\dom f$ and $\norm{x'-x}<\epsilon$ together imply $\abs{f(x')-f(x)}<\tau$. For example, it should be possible to show that for small enough $\delta$, one has $z+\delta(z-z')>(z+f(x))/2$ but $f(x+\delta(x-x'))<(z+f(x))/2$.
    \end{enumerate}
\end{problem}

\begin{solution}
    {Solution}
    \begin{enumerate}[(a)]
        \item {
                \begin{proof}
                    Assume $(x,f(x))\in\ri\epi f$, then there exists $\epsilon>0$ such that $B((x,f(x)),\epsilon)\subseteq\epi f$. Since $f$ is a proper convex function, we have $f(x)\neq\infty$. Consider the point $(x, f(x)-\frac{\epsilon}{2})$, though it is within the ball, it is clearly not in $\epi f$ since the second component is strictly less than $f(x)$. This is contradicted to the original assumption. Thus, $(x,f(x))\notin\epi f$.
                \end{proof}
        }
        \item {
            \begin{proof}
                If $x\notin \ri\dom f$, then by the prolongation principle, there is a direction $d\in\R^n$ such that $x+\lambda d\notin \dom f$ for all $\lambda>0$. Therefore, for any $z'$ and arbitrary small  $\lambda>0$, $(x+\lambda d, z')\notin \epi f$, implying $(x,z)$ is not in the relative interior of $\epi f$.
            \end{proof}
        }
        \item {
            \begin{proof}
                Let $(x, z) \in R$, by definition, we know $x \in \ri\dom f$ and $z > f(x)$.

                To prove $(x,z) \in \ri\epi f$, we must show that for every $(x', z') \in \epi f$, there exists a $\delta > 0$ such that
                \[(x, z) + \delta((x, z) - (x', z')) \in \epi f.\]

                The point halfway between $(x, z)$ and $(x', z')$ is
                \[ \left( \frac{x + x'}{2}, \frac{z + z'}{2} \right). \]
                Due to the convexity of $f$, this lies strictly above the graph of $f$ at $x$.

                For $x \in \ri\dom f$, by continuity of convex functions, for any $\tau > 0$, there exists $\epsilon > 0$ such that if $\|x' - x\| < \epsilon$ and $x' \in \dom f$, then $|f(x') - f(x)| < \tau$.

                Choose $\tau = \frac{z - f(x)}{2}$. By continuity, there exists $\epsilon > 0$ ensuring that
                \[ f(x') < f(x) + \tau \]
                whenever $\|x' - x\| < \epsilon$. Given our choice of $\tau$, this means
                \[ f(x') < \frac{z + f(x)}{2} \]
                for $\|x' - x\| < \epsilon$.

                Choose $\delta$ small enough that the point
                \[ (x, z) + \delta((x, z) - (x', z')) \]
                is within an $\epsilon$-distance from $x$ in its first coordinate, and lies below the midway point of $(x, z)$ and $(x', z')$ in its second coordinate. This ensures that this point lies strictly above the graph of $f$.

                Thus, for any $(x', z') \in \epi f$, there exists a $\delta > 0$ such that $(x, z) + \delta((x, z) - (x', z')) \in \epi f$, proving that any $(x, z) \in R$ is also in $\ri\epi f$. \hfill $\square$
            \end{proof}
        }
    \end{enumerate}
\end{solution}

\begin{problem}
    {Q3}
    In this problem, we will prove the following "almost industrial strength" generalization of Proposition 4.2.5(a): let $\R^m\rightarrow\R\cup\set{+\infty}$ be a proper convex function and let $A$ be an $m\times n$ matrix. Define $g(x)=f(Ax)$, which is also a convex function. Then, for all $x\in \R^n$,
    \begin{equation}
        \partial g(x)\supseteq A\trn\partial f(Ax).\label{eq:1}
    \end{equation}
    Furthermore, if $\ri\dom f\cap\im A\neq\emptyset$, that is, there exists some point in $\bar{z}\in\ri\dom f$ that may be expressed as $\bar{z}=A\bar{x}$ for some $\bar{x}\in\R^n$, then for any $x\in\R^n$,
    \begin{equation}
        \partial g(x)=A\trn\partial f(Ax).\label{eq:2}
    \end{equation}
    \begin{enumerate}[(a)]
        \item Prove~\cref{eq:1}.
        \item Define $U=\set{(x,z)\in\R^n\times\R^m\mid z=Ax}$, which is a linear subspace of $\R^n\times\R^m$, along with the following functions $\R^n\times\R^m\rightarrow\R\cup\set{+\infty}$: {
            \begin{align*}
                F_1(x,z)&=f(z) \\
                F_2(x,z)&=\begin{cases}
                    0, & z=Ax \\
                    +\infty, & z\neq Ax
                \end{cases} \\
                F(x,z)&=F_1(x,z)+F_2(x,z)=\begin{cases}
                    f(z), & z=Ax \\
                    +\infty, & z\neq Ax
                \end{cases}
            \end{align*}
        }
        Show that $F_1$, $F_2$ and $F$ defined in this manner are convex and that $d\in\partial g(x)$ implies $(d,0)\in\partial F(x,Ax)$.
        \item Show that {
            \begin{align*}
                \partial F_1(x,z) &= \set{0}\times\partial f(z) \\
                \partial F_2(x,z) &= \begin{cases}
                    \set{(A\trn w,-w)\mid w\in\R^m}, & z=Ax \\
                    \emptyset, & z\neq Ax
                \end{cases}
            \end{align*}
        }
        You may use the elementary linear-algebra fact that for any $p\times q$ matrix $M$, the subspace orthogonal to the subspace $\set{y\in\R^q\mid My=0}$ is $\set{M\trn w\mid w\in\R^q}$.
        \item For the reminder of this problem, assume $\ri\dom f \cap \im A\neq\emptyset$. Show that, in this case, $\ri\dom F_1$ and $\ri \dom F_2$ must intersect.
        \item Find an expression for $\partial F(x,z)=\partial(F_1+F_2)(x,z).$ You may use version of the Moreau-Rockafellar theorem, which asserts that if $\ri\dom f_1\cap\ri\dom f_2\neq\emptyset$, then $\partial(f_1+f_2)(x)=\partial f_1(x)+\partial f_2(x)$ for all $x\in\R^n$.
        \item Combine the above results to show that $\partial g(x)=A\trn\partial f(Ax)$.
    \end{enumerate}
\end{problem}

\begin{solution}
    {Solution}
    \begin{enumerate}[(a)]
        \item {
            \begin{proof}
                Let's take any $v\in\partial f(Ax)$. By the definition of subgradients, we have
                \[
                    f(y)\geq f(Ax)+v\trn(y-Ax) \quad \forall y\in\R^m.
                \]
                Let $y=Ax+Az$, then we have
                \[
                    f(Ax+Az)\geq f(Ax)+v\trn(Ax+Az-Ax)=f(Ax)+v\trn(Az).
                \]
                Notice that the left side is $g(x+z)$, and the function on the right involves $z$ which is the perturbation in $x$.
                \[
                    g(x+z)\geq g(x)+v\trn Az \quad \forall z\in\R^n.
                \]
                This is the definition of the subgradients of $g$ at $x$, thus, we have $A\trn v\in\partial g(x)$. Since $v$ is arbitrary, we have $\partial g(x)\supseteq A\trn\partial f(Ax)$.
            \end{proof}
        }
        \item {
            \begin{proof}
                The proof is as follow:
                \begin{itemize}
                    \item \textbf{\( F_1 \) is convex:}
                    \( F_1(x,z) \) is convex since \( f \) is given to be a proper convex function.

                    \item \textbf{\( F_2 \) is convex:}
                    Consider any two points \( (x_1, z_1) \) and \( (x_2, z_2) \) in \( \mathbb{R}^n \times \mathbb{R}^m \) and any \( \lambda \in (0,1) \).
                    \begin{itemize}
                        \item If \( z_1 = Ax_1 \) and \( z_2 = Ax_2 \), then the line segment between \( (x_1, Ax_1) \) and \( (x_2, Ax_2) \) is entirely contained in the set \( \{ (x,z) \mid z = Ax \} \), and hence \( F_2 \) is zero along this segment.

                        \item If either \( z_1 \neq Ax_1 \) or \( z_2 \neq Ax_2 \), then \( F_2 \) takes the value \( +\infty \) at one or both of these points, and it is trivially convex as \( \infty \leq \infty \).
                    \end{itemize}

                    \item \textbf{\( F \) is convex:}
                    \( F \) is the sum of \( F_1 \) and \( F_2 \), and the sum of two convex functions is also convex.

                    \item \textbf{\( d \in \partial g(x) \) implies \( (d, 0) \in \partial F(x, Ax) \):}
                    By the definition of subgradients and function \( g \), we have:
                    \[ g(x+h) \geq g(x) + d^\top h \text{ for all } h \in \mathbb{R}^n. \]
                    Given \( g(x) = f(Ax) \), this can be rewritten as:
                    \[ f(A(x+h)) \geq f(Ax) + d^\top h. \]
                    Considering the definition of \( F \), we can express this inequality as:
                    \[ F(x+h, A(x+h)) \geq F(x, Ax) + d^\top h. \]
                    Given the definition of the subgradients for functions of two variables, this means:
                    \[ (d, 0) \in \partial F(x, Ax) \]
                \end{itemize}
            \end{proof}
        }
        \item {
            \begin{proof}
                The proof is as follow:
                \begin{itemize}
                    \item Since $F_1$ is only dependent on $z$, its subgradients with respect to $x$ will simply be $0$. With respect to $z$, the subgradients will be the same as the subgradients of $f$ at $z$. Thus, we have $\partial F_1(x,z)=\set{0}\times\partial f(z)$.
                    \item {
                        \begin{itemize}
                            \item When $z=Ax$: To find the subgradients of $F_2$, we want to find all vectors $(d,w)$ such that: \[F_2(x+h,z+k)\geq F_2(x,z)+\ip{d,h}+\ip{w,k}\] for all $(h,k)$. Given that $F_2(x,z)=0$ for $z=Ax$, the inequality becomes \[F_2(x+h,z+k)\geq \ip{d,h}+\ip{w,k}.\] Considering perturbing $z$ slightly by some $k$ such that $z+k\neq A(x+h)$. In this case $F_2(x+h,z+k)=+\infty$, thus, the inequality holds for all $(d,w)$. Thus, only need to deal with $z+k = A(x+h)$. Now the inequality becomes \[0\geq \ip{d,h}+\ip{w,k}.\] Given $k=A(x+h)-Ax=Ah$, the inequalitycan be written as: \[0\geq \ip{d,h}+\ip{w,Ah}.\] For this to hold for all $h$, $d$ must be orthogonal to $A$ and $w$ must be orthogonal to the nullspace of $A\trn$. Using the hint, we have \[d=A\trn w\] for some $w\in\R^m$. Next for any $h$: \[k=Ah\Rightarrow -k=-Ah.\] Thus $w$ should be the negative of any vector in $\R^m$ to ensure the orthogonality condition. Thus, we have $\partial F_2(x,z)=\set{(A\trn w,-w)\mid w\in\R^m}$.
                            \item When $z\neq Ax$: In this case, $F_2(x,z)=+\infty$. Thus, the subgradients are empty.
                        \end{itemize}
                    }
                \end{itemize}
            \end{proof}
        }
        \item {
            \begin{proof}
                $\ri\dom F_1$ is the set of all $z$ such that $f(z)<+\infty$, which means it is the relative interior of the domain of $f$, \ie $ri\dom f$. $\ri\dom F_2$ is the set of all $z$ such that $z=Ax$ for some $x$, which means it is the image of $A$, \ie $im A$. Given that $\ri\dom f\cap\im A\neq\emptyset$, there exists some $\bar{z}\in\ri\dom f$ such that $\bar{z}=A\bar{x}$ for some $\bar{x}\in\R^n$. Therefore, $\bar{z}$ belongs to both $\ri\dom F_1$ and $\ri\dom F_2$, which means $\ri\dom F_1$ and $\ri\dom F_2$ must intersect.
            \end{proof}
        }
        \item {
            \begin{proof}
                The Moreau-Rockafellar theorem states that if $\ri\dom f_1\cap\ri\dom f_2\neq\emptyset$, then $\partial(f_1+f_2)(x)=\partial f_1(x)+\partial f_2(x)$ for all $x\in\R^n$. Since $\ri\dom F_1$ and $\ri\dom F_2$ intersect, we have $\partial F(x,z)=\partial(F_1+F_2)(x,z)=\partial F_1(x,z)+\partial F_2(x,z)$. Thus, we have
                \begin{itemize}
                    \item When $z=Ax$: {
                        \begin{align*}
                            \partial F(x,z)&=\set{0}\times\partial f(z)+\set{(A\trn w,-w)\mid w\in\R^m} \\
                            &=\set{(A\trn w,v-w)\mid w\in\R^m, v\in\partial f(z)}.\\
                        \end{align*}
                    }
                    \item When $z\neq Ax$: {
                        \begin{align*}
                            \partial F(x,z)&=\emptyset+\emptyset \\
                            &=\emptyset.
                        \end{align*}
                    }
                \end{itemize}
            \end{proof}
        }
        \item {
            \begin{proof}
                To find $\partial g(x)$, we use the property that any $d$ in $\partial g(x)$ must satisfy $(d,0)\in\partial F(x,Ax)$. Given $F(x,z)=F_1(x,z)+F_2(x,z)$, $F(x,Ax)=F_1(x,Ax)+F_2(x,Ax)=f(Ax)=g(x)$. Thus, for any $d\in\partial g(x)$, the corresponding $(d,0)\in\partial F(x,Ax)$ must have the form $(A\trn w, v-w)$ where $w\in\R^m$ and $v\in\partial f(Ax)$. But the second coordinate is $0$, which implies $v=w\Rightarrow (d,0)=(A\trn w,0)$. This means $d=A\trn w$ for some $w\in\partial f(Ax)$. In other words: \[d\in A\trn\partial f(Ax).\] This is $\partial g(x)\subseteq A\trn\partial f(Ax)$. Combining the results from (a), we have $\partial g(x)=A\trn\partial f(Ax)$.
            \end{proof}
        }
    \end{enumerate}
\end{solution}

\end{document}