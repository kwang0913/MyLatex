\documentclass[11pt]{../formats/RU}
%%%% delimiters
\DeclarePairedDelimiter\parens{\lparen}{\rparen}  
\DeclarePairedDelimiter\bracks{\lbrack}{\rbrack}
\DeclarePairedDelimiter\braces{\lbrace}{\rbrace}
\DeclarePairedDelimiter\abs{\lvert}{\rvert}
\DeclarePairedDelimiter\norm{\lVert}{\rVert}
\DeclarePairedDelimiter\angles{\langle}{\rangle}
\DeclarePairedDelimiter\ceil{\lceil}{\rceil}
\DeclarePairedDelimiter\floor{\lfloor}{\rfloor}

%%%% math operators naming
\DeclareMathOperator*{\argmax}{\mathrm{argmax}}
\DeclareMathOperator*{\argmin}{\mathrm{argmin}}
\DeclareMathOperator{\tr}{\mathrm{tr}}
\DeclareMathOperator{\eig}{\mathrm{eig}}
\DeclareMathOperator{\sgn}{\mathrm{sgn}}
% \let\det\relax % "Undefine" \det
% \DeclareMathOperator{\det}{\mathrm{det}} % already defined in mathtools
\DeclareMathOperator{\diag}{\mathrm{diag}}
\DeclareMathOperator{\rank}{\mathrm{rank}}
\DeclareMathOperator{\Vol}{\mathrm{Vol}}   % volume
\DeclareMathOperator{\Surf}{\mathrm{Surf}} % surface area

%%%% Transforms! -- requires mathtools package
\newcommand*{\LapTrans}{\xleftrightarrow{\mathcal{Z}}}
\newcommand*{\ZTrans}{\xleftrightarrow{\mathcal{L}}}
\newcommand*{\CTFS}{\xleftrightarrow{\mathrm{CTFS}}}
\newcommand*{\CTFT}{\xleftrightarrow{\mathrm{CTFT}}}
\newcommand*{\DTFS}{\xleftrightarrow{\mathrm{DTFS}}}
\newcommand*{\DTFT}{\xleftrightarrow{\mathrm{DTFT}}}

%%%% vector font
\let\oldvec\vec
\renewcommand*{\vec}[1]{\mathbf{#1}}
\newcommand*{\trn}{{}^{\mkern-4mu\intercal}}
\newcommand*{\inv}{^{-1}} 

%%%% number systems
\DeclareMathOperator{\R}{\mathbb{R}}
\DeclareMathOperator{\C}{\mathbb{C}}
\DeclareMathOperator{\N}{\mathbb{N}}
\DeclareMathOperator{\Z}{\mathbb{Z}}
\DeclareMathOperator{\F}{\mathbb{F}}
\DeclareMathOperator{\Q}{\mathbb{Q}}

%%%% STATISTICS AND PROBABILITY
\newcommand*{\Var}{\mathop{\mathrm{Var}}}
\newcommand*{\Cov}{\mathop{\mathrm{Cov}}}
\newcommand*{\Corr}{\mathop{\mathrm{Corr}}}
\newcommand*{\MSE}{\mathop{\mathrm{MSE}}}

\newcommand*{\E}[1]{\mathbb{E}\bracks*{#1}}
\newcommand*{\condE}[2]{\mathbb{E}\bracks*{#1 \mid #2}}
\renewcommand*{\P}[1]{\mathbb{P}\parens*{#1}}
\newcommand*{\condP}[2]{\mathbb{P}\parens*{#1 \mid #2}}

\DeclareMathOperator{\Bern}{\mathsf{Bern}}
\DeclareMathOperator{\Unif}{\mathsf{Unif}}
\DeclareMathOperator{\Expv}{\mathsf{Exp}}
\DeclareMathOperator{\Poi}{\mathsf{Poi}}
\DeclareMathOperator{\Gamv}{\mathsf{Gamma}}
\DeclareMathOperator{\Dirv}{\mathsf{Dir}}
\DeclareMathOperator{\Mult}{\mathsf{Mult}}
\DeclareMathOperator{\Beta}{\mathsf{Beta}}
\DeclareMathOperator{\Geomv}{\mathsf{Geom}}
\DeclareMathOperator{\Binomv}{\mathsf{Binom}}
\DeclareMathOperator{\NegBinomv}{\mathsf{NB}}
\DeclareMathOperator{\Lap}{\mathsf{Lap}}
\DeclareMathOperator{\Gaus}{\mathsf{N}}
\DeclareMathOperator{\Weibull}{\mathsf{Weibull}}


\DeclareMathOperator{\iidsim}{\stackrel{\mathrm{i.i.d.}}{\sim}}
\DeclareMathOperator{\diff}{\mathop{}\!\mathrm{d}}

%%%% Special norms and linear algebra stuff
\newcommand*{\subgnorm}[1]{\norm*{#1}_{\psi_2}}
\newcommand*{\subexpnorm}[1]{\norm*{#1}_{\psi_1}}
\newcommand*{\frobnorm}[1]{\norm*{#1}_{\mathrm{F}}}
\newcommand*{\opnorm}[1]{\norm*{#1}_{\mathrm{op}}}
\newcommand*{\Lipnorm}[1]{\norm*{#1}_{\mathrm{Lip}}}

%%%%
\newcommand*{\set}[1]{\braces*{\,#1\,}}
\newcommand*{\ie}{\textnormal{i.e.\ }}
\newcommand*{\eg}{\textnormal{e.g.\ }}
\newcommand*{\etc}{\textnormal{etc.\ }}

%Theorem style
\declaretheorem[numbered=no, style=definition]{axiom}
\declaretheorem[numberwithin=section,style=definition]{definition}
\declaretheorem[sibling=definition]{theorem, lemma, corollary, proposition, conjecture}
\declaretheorem[numbered=no,style=remark]{remark, claim}

%Information to be included in the title page:
\title[About Beamer]{About the Beamer class in presentation making}
\subtitle{A short story}
\author[Kai] % (optional, for multiple authors)
{Kailong Wang\inst{1} \and Someone Else\inst{2}}
\institute[Rutgers] % (optional)
{
  \inst{1}%
  Faculty of ECE\\
  Very Famous University
  \and
  \inst{2}%
  Faculty of Statistics\\
  Very Famous University
}
\date[\today] % (optional)
{Very Large Conference, \today}

\begin{document}
\frame{\titlepage}

%---------------------------------------------------------
%This block of code is for the table of contents after
%the title page
\begin{frame}
\frametitle{Table of Contents}
\tableofcontents
\end{frame}
%---------------------------------------------------------


\section{First section}

%---------------------------------------------------------
%Changing visivility of the text
\begin{frame}
\frametitle{Sample frame title}
This is a text in second frame. For the sake of showing an example.

\begin{itemize}
    \item<1-> Text visible on slide 1
    \item<2-> Text visible on slide 2
    \item<3> Text visible on slides 3
    \item<4-> Text visible on slide 4
\end{itemize}
\end{frame}

%---------------------------------------------------------


%---------------------------------------------------------
%Example of the \pause command
\begin{frame}
In this slide \pause

the text will be partially visible \pause

And finally everything will be there
\end{frame}
%---------------------------------------------------------

\section{Second section}

%---------------------------------------------------------
%Highlighting text
\begin{frame}
\frametitle{Sample frame title}

In this slide, some important text will be
\alert{highlighted} because it's important.
Please, don't abuse it.

\begin{block}{Remark}
Sample text
\end{block}

\begin{alertblock}{Important theorem}
Sample text in red box
\end{alertblock}

\begin{exampleblock}{Examples}
Sample text in green box. The title of the block is "Examples".
\end{exampleblock}
\end{frame}
%---------------------------------------------------------


%---------------------------------------------------------
%Two columns
\begin{frame}
\frametitle{Two-column slide}
\framesubtitle{This is a subtitle}

\begin{columns}

\column{0.5\textwidth}
This is a text in first column.
$$E=mc^2$$
\begin{itemize}
\item First item $\ell$
\item Second item
\end{itemize}

\column{0.5\textwidth}
This text will be in the second column
and on a second tought this is a nice looking
layout in some cases.
\end{columns}
\end{frame}
%---------------------------------------------------------


\end{document}