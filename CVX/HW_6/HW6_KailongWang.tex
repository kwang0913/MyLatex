\documentclass{article}
\usepackage[
    letterpaper,
    top=1in,
    bottom=1in,
    inner=0.75in,
    outer=0.75in,
    % margin=1in,
]{geometry}

%%%% Page Header and Foot %%%%
\usepackage{fancyhdr}
\fancypagestyle{plain}{
\fancyhf{}
\fancyhead[L]{266:711:685}
\fancyhead[C]{\textbf{Convex Analysis and Optimization}}
\fancyhead[R]{Rutgers University}
\fancyfoot[L]{}
\fancyfoot[C]{\thepage}
\fancyfoot[R]{}
}
\pagestyle{plain}

%%%% Load Format %%%%
%%%%%%%%%%%%%%%%%%%%%%%%%%%%%%%%%%%%%%%%%%%%%%%%%%%%%%%%%%%%%%%%%%%%%%%%%%%%%%%%%%%% Packages to Load
%% Font Packages
\RequirePackage[english]{babel} %Multilingual support
\RequirePackage[utf8]{inputenc} %utf8 support.
\RequirePackage[T1]{fontenc} %Better Font Encoding for printing
\RequirePackage{libertinus} %Popular Math Font
\RequirePackage{microtype} %Better Font Spacing.
\RequirePackage{lmodern} %Popular Font

%% Math Packages
\RequirePackage{amsfonts,amsmath,amssymb} %Math Fonts and Symbols
\RequirePackage{mathtools} %Define math symbols
\RequirePackage{mleftright}  %Fixes some annoying spacing issues
\RequirePackage{bm} %Boldface math, load after other math font packages

%% Graphics Packages
\RequirePackage{graphicx} %Graphics backend with key-value arguments
\RequirePackage[export]{adjustbox} %More arguments for includegraphics
\RequirePackage[svgnames]{xcolor} %Beamer has this by default
\RequirePackage[most]{tcolorbox} %Customized Box
\RequirePackage{transparent} %Transparent Images
\RequirePackage{subcaption} %Package to use subfigure

%% Table Packages
\RequirePackage{tabularx} %Including tabular with more space option
\RequirePackage{multirow} %Package to use multirow in tables
\RequirePackage{colortbl} %Package to use color in tables
\RequirePackage{array} %Math and more spacing support in tables

%% Theorems etc (also  problems)
\RequirePackage[shortlabels]{enumitem} %Better List
\RequirePackage{hyperref} %Beamer has this by default
\RequirePackage{cleveref} %Must load before amsthm. Conflict With Beamer
\RequirePackage{amsthm, thmtools, thm-restate} %Better Theorems

%% Code Packages
\RequirePackage{listings} %Package to insert code. Better than minted
\RequirePackage{algorithm} %Package to insert algorithm
\RequirePackage{algpseudocode} %Package to insert algorithm
% \RequirePackage{keyval}

%% Display Packages
% \RequirePackage[document]{ragged2e} %Replaced by microtype
\RequirePackage{csquotes} %Package to facilities quotations
\RequirePackage{multicol} %Package to use multicols

%% Other Packages
\RequirePackage{xspace} %Package to add space after macros
\RequirePackage{calc} %Package to calculate in Latex
\RequirePackage{rotating} %Package to rotate objects
% \RequirePackage{pgf, tikz} %Package to draw figures.
\RequirePackage{epstopdf} %Package to convert eps to pdf
\RequirePackage[backend=biber, style=numeric, sorting=none]{biblatex}
%%%% delimiters
\DeclarePairedDelimiter\parens{\lparen}{\rparen}  
\DeclarePairedDelimiter\bracks{\lbrack}{\rbrack}
\DeclarePairedDelimiter\braces{\lbrace}{\rbrace}
\DeclarePairedDelimiter\abs{\lvert}{\rvert}
\DeclarePairedDelimiter\norm{\lVert}{\rVert}
\DeclarePairedDelimiter\angles{\langle}{\rangle}
\DeclarePairedDelimiter\ceil{\lceil}{\rceil}
\DeclarePairedDelimiter\floor{\lfloor}{\rfloor}

%%%% math operators naming
\DeclareMathOperator*{\argmax}{\mathrm{argmax}}
\DeclareMathOperator*{\argmin}{\mathrm{argmin}}
\DeclareMathOperator{\tr}{\mathrm{tr}}
\DeclareMathOperator{\eig}{\mathrm{eig}}
\DeclareMathOperator{\sgn}{\mathrm{sgn}}
% \let\det\relax % "Undefine" \det
% \DeclareMathOperator{\det}{\mathrm{det}} % already defined in mathtools
\DeclareMathOperator{\diag}{\mathrm{diag}}
\DeclareMathOperator{\rank}{\mathrm{rank}}
\DeclareMathOperator{\Vol}{\mathrm{Vol}}   % volume
\DeclareMathOperator{\Surf}{\mathrm{Surf}} % surface area

%%%% Transforms! -- requires mathtools package
\newcommand*{\LapTrans}{\xleftrightarrow{\mathcal{Z}}}
\newcommand*{\ZTrans}{\xleftrightarrow{\mathcal{L}}}
\newcommand*{\CTFS}{\xleftrightarrow{\mathrm{CTFS}}}
\newcommand*{\CTFT}{\xleftrightarrow{\mathrm{CTFT}}}
\newcommand*{\DTFS}{\xleftrightarrow{\mathrm{DTFS}}}
\newcommand*{\DTFT}{\xleftrightarrow{\mathrm{DTFT}}}

%%%% vector font
\let\oldvec\vec
\renewcommand*{\vec}[1]{\mathbf{#1}}
\newcommand*{\trn}{{}^{\mkern-4mu\intercal}}
\newcommand*{\inv}{^{-1}} 

%%%% number systems
\DeclareMathOperator{\R}{\mathbb{R}}
\DeclareMathOperator{\C}{\mathbb{C}}
\DeclareMathOperator{\N}{\mathbb{N}}
\DeclareMathOperator{\Z}{\mathbb{Z}}
\DeclareMathOperator{\F}{\mathbb{F}}
\DeclareMathOperator{\Q}{\mathbb{Q}}

%%%% STATISTICS AND PROBABILITY
\newcommand*{\Var}{\mathop{\mathrm{Var}}}
\newcommand*{\Cov}{\mathop{\mathrm{Cov}}}
\newcommand*{\Corr}{\mathop{\mathrm{Corr}}}
\newcommand*{\MSE}{\mathop{\mathrm{MSE}}}

\newcommand*{\E}[1]{\mathbb{E}\bracks*{#1}}
\newcommand*{\condE}[2]{\mathbb{E}\bracks*{#1 \mid #2}}
\renewcommand*{\P}[1]{\mathbb{P}\parens*{#1}}
\newcommand*{\condP}[2]{\mathbb{P}\parens*{#1 \mid #2}}

\DeclareMathOperator{\Bern}{\mathsf{Bern}}
\DeclareMathOperator{\Unif}{\mathsf{Unif}}
\DeclareMathOperator{\Expv}{\mathsf{Exp}}
\DeclareMathOperator{\Poi}{\mathsf{Poi}}
\DeclareMathOperator{\Gamv}{\mathsf{Gamma}}
\DeclareMathOperator{\Dirv}{\mathsf{Dir}}
\DeclareMathOperator{\Mult}{\mathsf{Mult}}
\DeclareMathOperator{\Beta}{\mathsf{Beta}}
\DeclareMathOperator{\Geomv}{\mathsf{Geom}}
\DeclareMathOperator{\Binomv}{\mathsf{Binom}}
\DeclareMathOperator{\NegBinomv}{\mathsf{NB}}
\DeclareMathOperator{\Lap}{\mathsf{Lap}}
\DeclareMathOperator{\Gaus}{\mathsf{N}}
\DeclareMathOperator{\Weibull}{\mathsf{Weibull}}


\DeclareMathOperator{\iidsim}{\stackrel{\mathrm{i.i.d.}}{\sim}}
\DeclareMathOperator{\diff}{\mathop{}\!\mathrm{d}}

%%%% Special norms and linear algebra stuff
\newcommand*{\subgnorm}[1]{\norm*{#1}_{\psi_2}}
\newcommand*{\subexpnorm}[1]{\norm*{#1}_{\psi_1}}
\newcommand*{\frobnorm}[1]{\norm*{#1}_{\mathrm{F}}}
\newcommand*{\opnorm}[1]{\norm*{#1}_{\mathrm{op}}}
\newcommand*{\Lipnorm}[1]{\norm*{#1}_{\mathrm{Lip}}}

%%%%
\newcommand*{\set}[1]{\braces*{\,#1\,}}
\newcommand*{\ie}{\textnormal{i.e.\ }}
\newcommand*{\eg}{\textnormal{e.g.\ }}
\newcommand*{\etc}{\textnormal{etc.\ }}
\allowdisplaybreaks

%%%% Theorem Style %%%%
\declaretheorem[numbered=no, style=plain]{axiom, lemma}
\declaretheorem[numberwithin=section,style=definition]{definition}
\declaretheorem[sibling=definition]{theorem, corollary, proposition, conjecture}
\declaretheorem[numbered=no,style=remark]{remark, claim}

%%%% Document Information %%%%
\title{HW $6$}
\author{Kailong Wang}
\date{\today}

%%%% Start Document %%%%
\begin{document}
\maketitle

\begin{problem}
    {Q1}
    For each of the following choices of $f:\R\rightarrow \R \cup \set{+\infty}$, compute the convex conjugate function $f^*$:
    \begin{enumerate}[(a)]
        \item $f(x)=\frac{1}{2}x^2$.
        \item For $a ,b\in\R$, $a<b$, ${
            f(x)=\delta_{[a, b]}= \begin{cases}
                0 & x\in [a, b],\\
                +\infty & \text{otherwise}.
            \end{cases}
            }$
        \item $f(x)=e^x$.
    \end{enumerate}
\end{problem}

\begin{solution}
    {Solution}
    We want to find \[f^* = g(\lambda) = \sup_{x\in\dom(f)}\set{\ip*{\lambda, x}-f(x)}.\]
    \begin{enumerate}[(a)]
        \item {
            \begin{align*}
                g(\lambda) 
                &= \sup_{x\in\dom(f)}\set{\ip*{\lambda, x}-\frac{1}{2}x^2} \\
                \frac{\diff g}{\diff x} &= \lambda - x \tag{\text{Let $x=\lambda$ to maximize $g(\lambda)$}}\\
                g(\lambda) &= \frac{1}{2}\lambda^2
            \end{align*}
        }
        \item {
            \begin{align*}
                g(\lambda) 
                &= \sup_{x\in\dom(f)}\set{\ip*{\lambda, x}-\delta_{[a, b]}(x)} \\
                &= \begin{cases}
                    \sup_{x\in\dom(f)}\set{\ip*{\lambda, x}} & x\in[a, b] \\
                    \sup_{x\in\dom(f)} \set{-\infty} & \text{otherwise}
                \end{cases} \\
                \frac{dg}{dx} &= \begin{cases}
                    \lambda & x\in[a, b] \\
                    0 & \text{otherwise}
                \end{cases} \\
                g(\lambda) &= \begin{cases}
                    b\lambda & \lambda\geq 0 \\
                    a\lambda & \lambda\leq 0 \\
                    0 & \lambda=0
                \end{cases}
            \end{align*}
        }
        \item {
            \begin{align*}
                g(\lambda)
                &= \sup_{x\in\dom(f)}\set{\ip*{\lambda, x}-e^x} \\
                \frac{\diff g}{\diff x} &= \lambda - e^x \tag{\text{Let $x=\ln\lambda$ to maximize $g(\lambda)$}}\\
                g(\lambda) &= \lambda\ln\lambda - \lambda
            \end{align*}
        }
    \end{enumerate}
\end{solution}

\begin{problem}
    {Q2}
    A function $f:\R^n\rightarrow\R\cup\set{+\infty}$ is said to be positively homogeneous if
    \begin{align*}
        f(0) &= 0 \\
        f(\alpha x) &= \alpha f(x) \quad \forall \alpha > 0, x\in\R^n.
    \end{align*}
    (Note that some definitions omit the condition $f(0) = 0$, which we include here to accord with our notion of a cone as always containing the point $0$.)
    \begin{enumerate}[(a)]
        \item For any proper function $f:\R^n\rightarrow\R\cup\set{+\infty}$, show that $\epi f$ is a cone in $\R^{n+1}$ \iff $f$ is positively homogeneous.
        \item Consider any nonempty set $X\subseteq\R^n$. The \textit{support function} of $X$ is the convex conjugate ($\delta_X^*$) of the indicator function \[\delta_X = \begin{cases}
            0 & x\in X,\\
            +\infty & \text{otherwise}.
        \end{cases}\]
        Show that \[\delta_X^*(y)=\sup_{x\in X}\set{\ip*{x, y}},\] and this function is positively homogeneous.
        \item Show conversely that, given any positively homogeneous function $f$, its convex conjugate $f^*$ is the indicator function of some closed convex set $C$.
        \item Given a cone $K$, show that $\delta_K^* = \delta_{K^*}$. That is, the conjugate of the indicator function of a $K$ is the indicator function of its polar.
    \end{enumerate}
\end{problem}

\begin{solution}
    {Solution}
    \begin{enumerate}[(a)]
        \item {
            \begin{proof}
                We finish the proof by showing sufficiency and necessity.
                \begin{enumerate}[(i)]
                    \item {
                        If $\epi f$ is a cone in $\R^{n+1}$, then $\forall (x, f(x))\in\epi f$, we have $(\alpha x, \alpha f(x))\in\epi f$ for all $\alpha>0$. \\
                        With the definition of $\epi f$, we have $f(\alpha x)\leq \alpha f(x)$. \\
                        Since $f$ is proper, we have $f(\alpha x)\geq \alpha f(x)$. \\
                        Therefore, $f(\alpha x) = \alpha f(x)$, which means $\epi f$ is positively homogeneous.
                    }
                    \item {
                        If $f$ is positively homogeneous, we have $f(\alpha x) = \alpha f(x) \leq \alpha f(x)$ for all $\alpha>0$, which means $(\alpha x, \alpha f(x))\in\epi f$. \\
                        Since $\alpha$ is arbitrary, $\epi f$ is a cone by definition.
                    }
                \end{enumerate}
            \end{proof}
        }
        \item {
            \begin{proof}
                We have shown the $\delta_X^*(y)=\sup_{x\in X}\set{\ip*{x, y}}$ in Q1. We only need to prove $\delta_X^*$ is positively homogeneous. \\
                \begin{align*}
                    \delta_X^*(\alpha y) 
                    &= \sup_{x\in X}\set{\ip*{x, \alpha y}} \\
                    &= \sup_{x\in X}\set{\alpha\ip*{x, y}} \\
                    &= \alpha\sup_{x\in X}\set{\ip*{x, y}} \\
                    &= \alpha\delta_X^*(y)
                \end{align*}
            \end{proof}
        }
        \item {
            \begin{proof}
                Based on definition, we have $f^*(y) = \sup_{x\in\R^n}\set{\ip*{x, y}-f(x)}$. Construct a set $C$ which is $C = \set{y\in\R^n\mid \ip*{x, y}\leq f(x), \forall x\in\R^n}$.
                \begin{enumerate}[(i)]
                    \item For $y\in C$, we have $\ip*{x, y}\leq f(x)$ for all $x\in\R^n$ and thus $\ip*{x, y}-f(x)\leq 0$. By positively homogeneous $f(0)=0$, and we know $\ip*{0, y}=0$. So $0$ is attainable in the supremum. Therefore, $f^*(y) = \sup_{x\in\R^n}\set{\ip*{x, y}-f(x)}= 0$.
                    \item For $y\notin C$, there exists $x\in\R^n$ such that $\ip*{x, y}>f(x)$. Since $f$ is positively homogeneous, we have $\ip*{\alpha x, y}>f(\alpha x)=\alpha f(x)$ for all $\alpha>0$. Therefore, $\ip*{\alpha x, y}-f(\alpha x)=\alpha (\ip*{x, y}-f(x))$. This means for a given $y\notin C$, we can scale $x$ by arbitrary $\alpha\geq 0$ and the supremum will be unbounded. Therefore, $f^*(y)=\sup_{x\in\R^n}\set{\ip*{x, y}-f(x)}=+\infty$.
                \end{enumerate}
                The above two terms show the $f^*$ is the indicator function of $C$. Now we need to show $C$ is closed and convex to finish the proof.
                \begin{enumerate}[resume*]
                    \item For any $y_1, y_2\in C$ and $\lambda\in[0,1]$, we have $\ip*{x, \lambda y_1+(1-\lambda)y_2}\leq\lambda f(x)+(1-\lambda)f(x)$ for all $x\in\R^n$. Therefore, $\lambda y_1+(1-\lambda)y_2\in C$ and $C$ is convex. The continuity of $\ip*{\cdot, \cdot}$ implies $C$ is closed.
                \end{enumerate}
            \end{proof}
        }
        \item {
            \begin{proof}
                We have $\delta_K^*=\sup_{x\in K}\set{\ip*{x, y}}$ from Q2(b).\\
                For $y\in K^*$, it satisfies $\ip*{x, y}\leq 0, \forall x\in K$, which matches $\delta_K^*(y)=\sup_{x\in K}\set{\ip*{x, y}}=0$. \\
                For $y\notin K^*$, there exists $x\in K$ such that $\ip*{x, y}>0$. Since $K$ is a cone, we have $\ip*{\alpha x, y}=\alpha\ip*{x, y}$ for all $\alpha>0$. Therefore, $\ip*{\alpha x, y}$ is unbounded and matches $\delta_K^*(y)=\sup_{x\in K}\set{\ip*{x, y}}=+\infty$. \\
            \end{proof}
        }
    \end{enumerate}
\end{solution}

\begin{problem}
    {Q3}
    Consider the standard primal linear programming problem
    \begin{align*}
        \min_{x\in\R^n}\quad & c\trn x \\
        \st\quad & Ax=b \\
        & x\geq 0.
    \end{align*}
    Model this problem as $\min f(x)+g(Mx)$, where
    \[
    f(x)\dot{=}
    \begin{cases}
        c\trn x & x\geq 0\\
        +\infty & \text{otherwise}
    \end{cases}
    \qquad
    M\dot{=}A
    \qquad
    g(z)\dot{=}
    \begin{cases}
        0 & z=b\\
        +\infty & \text{otherwise},
    \end{cases}
    \]
    where $A$ is any $m\times n$ matrix and $b\in\R^m$. Show that the corresponding Fenchel dual is equivalent to the standard dual programming problem
    \begin{align}
        \label{eq:standard_dual}
        \begin{split}
            \max_{u\in\R^m}\quad & b\trn u \\
            \st\quad & A\trn u\leq c \\
        \end{split}
    \end{align}
    in the sense that any solution $y^*$ of the Fenchel dual is equal to $-u^*$, where $u^*$ is some optimal solution to the standard dual linear programming problem.
\end{problem}

\begin{solution}
    {Solution}
    \begin{proof}
        Solving $\sup_{x\geq 0}\set{\ip*{x, y}-f(x)}$ and get \[f^*(y)=\begin{cases}
            0 & y\leq c \\
            +\infty & \text{otherwise.}
        \end{cases}\]
        Since $g(z)$ is an indicator function, using the result of Q2(b), we have \[g^*(w)=w\trn b.\]
        The Fenchel dual problem is
        \begin{align*}
            \max_{w\in\R^m}&\set{-f^*(-w)-g^*(w)} \\
            \max_{w\in\R^m}&\set{-f^*(-w)-w\trn b} \\
            \Rightarrow&
            \begin{cases}
                -w\trn b & -w\leq c \\
                -\infty & \text{otherwise.}
            \end{cases}
        \end{align*}
        Let $-u=w$, we get the standard dual problem~\cref{eq:standard_dual}. Note the $w$ here is corresponding to $y$ of Fenchel dual problem in the question statement.
    \end{proof}
\end{solution}

\end{document}