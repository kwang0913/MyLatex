\documentclass{article}
\usepackage[
    letterpaper,
    top=1in,
    bottom=1in,
    inner=0.75in,
    outer=0.75in,
    % margin=1in,
]{geometry}

%%%% Page Header and Foot %%%%
\usepackage{fancyhdr}
\fancypagestyle{plain}{
\fancyhf{}
\fancyhead[L]{266:711:685}
\fancyhead[C]{\textbf{Convex Analysis and Optimization}}
\fancyhead[R]{Rutgers University}
\fancyfoot[L]{}
\fancyfoot[C]{\thepage}
\fancyfoot[R]{}
}
\pagestyle{plain}

%%%% Load Format %%%%
%%%%%%%%%%%%%%%%%%%%%%%%%%%%%%%%%%%%%%%%%%%%%%%%%%%%%%%%%%%%%%%%%%%%%%%%%%%%%%%%%%%% Packages to Load
%% Font Packages
\RequirePackage[english]{babel} %Multilingual support
\RequirePackage[utf8]{inputenc} %utf8 support.
\RequirePackage[T1]{fontenc} %Better Font Encoding for printing
\RequirePackage{libertinus} %Popular Math Font
\RequirePackage{microtype} %Better Font Spacing.
\RequirePackage{lmodern} %Popular Font

%% Math Packages
\RequirePackage{amsfonts,amsmath,amssymb} %Math Fonts and Symbols
\RequirePackage{mathtools} %Define math symbols
\RequirePackage{mleftright}  %Fixes some annoying spacing issues
\RequirePackage{bm} %Boldface math, load after other math font packages

%% Graphics Packages
\RequirePackage{graphicx} %Graphics backend with key-value arguments
\RequirePackage[export]{adjustbox} %More arguments for includegraphics
\RequirePackage[svgnames]{xcolor} %Beamer has this by default
\RequirePackage[most]{tcolorbox} %Customized Box
\RequirePackage{transparent} %Transparent Images
\RequirePackage{subcaption} %Package to use subfigure

%% Table Packages
\RequirePackage{tabularx} %Including tabular with more space option
\RequirePackage{multirow} %Package to use multirow in tables
\RequirePackage{colortbl} %Package to use color in tables
\RequirePackage{array} %Math and more spacing support in tables

%% Theorems etc (also  problems)
\RequirePackage[shortlabels]{enumitem} %Better List
\RequirePackage{hyperref} %Beamer has this by default
\RequirePackage{cleveref} %Must load before amsthm. Conflict With Beamer
\RequirePackage{amsthm, thmtools, thm-restate} %Better Theorems

%% Code Packages
\RequirePackage{listings} %Package to insert code. Better than minted
\RequirePackage{algorithm} %Package to insert algorithm
\RequirePackage{algpseudocode} %Package to insert algorithm
% \RequirePackage{keyval}

%% Display Packages
% \RequirePackage[document]{ragged2e} %Replaced by microtype
\RequirePackage{csquotes} %Package to facilities quotations
\RequirePackage{multicol} %Package to use multicols

%% Other Packages
\RequirePackage{xspace} %Package to add space after macros
\RequirePackage{calc} %Package to calculate in Latex
\RequirePackage{rotating} %Package to rotate objects
% \RequirePackage{pgf, tikz} %Package to draw figures.
\RequirePackage{epstopdf} %Package to convert eps to pdf
\RequirePackage[backend=biber, style=numeric, sorting=none]{biblatex}
%%%% delimiters
\DeclarePairedDelimiter\parens{\lparen}{\rparen}  
\DeclarePairedDelimiter\bracks{\lbrack}{\rbrack}
\DeclarePairedDelimiter\braces{\lbrace}{\rbrace}
\DeclarePairedDelimiter\abs{\lvert}{\rvert}
\DeclarePairedDelimiter\norm{\lVert}{\rVert}
\DeclarePairedDelimiter\angles{\langle}{\rangle}
\DeclarePairedDelimiter\ceil{\lceil}{\rceil}
\DeclarePairedDelimiter\floor{\lfloor}{\rfloor}

%%%% math operators naming
\DeclareMathOperator*{\argmax}{\mathrm{argmax}}
\DeclareMathOperator*{\argmin}{\mathrm{argmin}}
\DeclareMathOperator{\tr}{\mathrm{tr}}
\DeclareMathOperator{\eig}{\mathrm{eig}}
\DeclareMathOperator{\sgn}{\mathrm{sgn}}
% \let\det\relax % "Undefine" \det
% \DeclareMathOperator{\det}{\mathrm{det}} % already defined in mathtools
\DeclareMathOperator{\diag}{\mathrm{diag}}
\DeclareMathOperator{\rank}{\mathrm{rank}}
\DeclareMathOperator{\Vol}{\mathrm{Vol}}   % volume
\DeclareMathOperator{\Surf}{\mathrm{Surf}} % surface area

%%%% Transforms! -- requires mathtools package
\newcommand*{\LapTrans}{\xleftrightarrow{\mathcal{Z}}}
\newcommand*{\ZTrans}{\xleftrightarrow{\mathcal{L}}}
\newcommand*{\CTFS}{\xleftrightarrow{\mathrm{CTFS}}}
\newcommand*{\CTFT}{\xleftrightarrow{\mathrm{CTFT}}}
\newcommand*{\DTFS}{\xleftrightarrow{\mathrm{DTFS}}}
\newcommand*{\DTFT}{\xleftrightarrow{\mathrm{DTFT}}}

%%%% vector font
\let\oldvec\vec
\renewcommand*{\vec}[1]{\mathbf{#1}}
\newcommand*{\trn}{{}^{\mkern-4mu\intercal}}
\newcommand*{\inv}{^{-1}} 

%%%% number systems
\DeclareMathOperator{\R}{\mathbb{R}}
\DeclareMathOperator{\C}{\mathbb{C}}
\DeclareMathOperator{\N}{\mathbb{N}}
\DeclareMathOperator{\Z}{\mathbb{Z}}
\DeclareMathOperator{\F}{\mathbb{F}}
\DeclareMathOperator{\Q}{\mathbb{Q}}

%%%% STATISTICS AND PROBABILITY
\newcommand*{\Var}{\mathop{\mathrm{Var}}}
\newcommand*{\Cov}{\mathop{\mathrm{Cov}}}
\newcommand*{\Corr}{\mathop{\mathrm{Corr}}}
\newcommand*{\MSE}{\mathop{\mathrm{MSE}}}

\newcommand*{\E}[1]{\mathbb{E}\bracks*{#1}}
\newcommand*{\condE}[2]{\mathbb{E}\bracks*{#1 \mid #2}}
\renewcommand*{\P}[1]{\mathbb{P}\parens*{#1}}
\newcommand*{\condP}[2]{\mathbb{P}\parens*{#1 \mid #2}}

\DeclareMathOperator{\Bern}{\mathsf{Bern}}
\DeclareMathOperator{\Unif}{\mathsf{Unif}}
\DeclareMathOperator{\Expv}{\mathsf{Exp}}
\DeclareMathOperator{\Poi}{\mathsf{Poi}}
\DeclareMathOperator{\Gamv}{\mathsf{Gamma}}
\DeclareMathOperator{\Dirv}{\mathsf{Dir}}
\DeclareMathOperator{\Mult}{\mathsf{Mult}}
\DeclareMathOperator{\Beta}{\mathsf{Beta}}
\DeclareMathOperator{\Geomv}{\mathsf{Geom}}
\DeclareMathOperator{\Binomv}{\mathsf{Binom}}
\DeclareMathOperator{\NegBinomv}{\mathsf{NB}}
\DeclareMathOperator{\Lap}{\mathsf{Lap}}
\DeclareMathOperator{\Gaus}{\mathsf{N}}
\DeclareMathOperator{\Weibull}{\mathsf{Weibull}}


\DeclareMathOperator{\iidsim}{\stackrel{\mathrm{i.i.d.}}{\sim}}
\DeclareMathOperator{\diff}{\mathop{}\!\mathrm{d}}

%%%% Special norms and linear algebra stuff
\newcommand*{\subgnorm}[1]{\norm*{#1}_{\psi_2}}
\newcommand*{\subexpnorm}[1]{\norm*{#1}_{\psi_1}}
\newcommand*{\frobnorm}[1]{\norm*{#1}_{\mathrm{F}}}
\newcommand*{\opnorm}[1]{\norm*{#1}_{\mathrm{op}}}
\newcommand*{\Lipnorm}[1]{\norm*{#1}_{\mathrm{Lip}}}

%%%%
\newcommand*{\set}[1]{\braces*{\,#1\,}}
\newcommand*{\ie}{\textnormal{i.e.\ }}
\newcommand*{\eg}{\textnormal{e.g.\ }}
\newcommand*{\etc}{\textnormal{etc.\ }}
\allowdisplaybreaks

%%%% Theorem Style %%%%
\declaretheorem[numbered=no, style=plain]{axiom, lemma}
\declaretheorem[numberwithin=section,style=definition]{definition}
\declaretheorem[sibling=definition]{theorem, corollary, proposition, conjecture}
\declaretheorem[numbered=no,style=remark]{remark, claim}

%%%% Document Information %%%%
\title{Final Exam}
\author{Kailong Wang}
\date{\today}

%%%% Start Document %%%%
\begin{document}
\maketitle

\begin{problem}
    {A primal-dual identity allowing $c\neq 1$.}
    At the end of the last class of the semester, I mentioned the identity
    \[(\forall r\in\R^n) \qquad \prox_h(r)+\prox_{h^*}(r)=r\]
    for any closed proper convex function $h: \R^n\rightarrow\R\cup\set{+\infty}$. For such a function $h$ and an arbitrary positive scalar $c$, prove the more general identity
    \[(\forall r\in\R^n) \qquad \prox_{ch}(r)+c\prox_{(1/c)h^*}(\frac{1}{c}r)=r,\]
    which reduces to the identity shown in class when $c=1$.
\end{problem}

\begin{solution}
    {Solution}
    \begin{proof}
        With the definition from (3.12) of the class notes, we have
        \begin{align*}
            \prox_{h}(r)&=\argmin_{x\in\R^n}\braces*{h(x)+\frac{1}{2}\norm*{x-r}^2} \\
            \prox_{h^*}(r)&=\argmin_{x\in\R^n}\braces*{h^*(x)+\frac{1}{2}\norm*{x-r}^2}. \\
            \prox_{h}(r)+\prox_{h^*}(r)&=\argmin_{x\in\R^n}\braces*{h(x)+h^*(x)+\norm*{x-r}^2} \\
            &\Rightarrow x=r.
        \end{align*}
        Since $0\in \partial\{h(x)+h^*(x)\}$ and take derivative to solve $\norm{x-r}^2$. For the general case,
        \begin{align*}
            \prox_{ch}(r)
            &= \argmin_{x\in\R^n}\braces*{ch(x)+\frac{1}{2}\norm*{x-r}^2} \\
            &= \argmin_{x\in\R^n}\braces*{h(x)+\frac{1}{2c}\norm*{x-r}^2} \\
            c \prox_{(1/c)h^*}(\frac{1}{c}r)
            &= c \argmin_{x\in\R^n}\braces*{\frac{1}{c}h^*(x)+\frac{1}{2}\norm*{x-\frac{1}{c}r}^2} \\
            &= \argmin_{x\in\R^n}\braces*{h^*(x)+\frac{c}{2}\norm*{x-r}^2} \\
            \prox_{ch}(r)+c \prox_{(1/c)h^*}(\frac{1}{c}r)
            &= \argmin_{x\in\R^n}\braces*{h(x)+\frac{1}{2c}\norm*{x-r}^2+h^*(x)+\frac{c}{2}\norm*{x-r}^2} \\
            &\Rightarrow \frac{\diff}{\diff x}\braces*{\frac{1}{2c}\norm*{x-r}^2+\frac{c}{2}\norm*{x-r}^2} \Rightarrow x=r.
        \end{align*}
    \end{proof}
\end{solution}

\begin{problem}
    {General conic augmented Lagrangian.}
    Consider the convex optimization problem
    \begin{align}
        \label{eq:1}
        \begin{aligned}
            &\min && f(x) \\
            &\st && Ax-b \in K,
        \end{aligned}
    \end{align}
    where $f:\R^n\rightarrow\R\cup\set{+\infty}$ is closed proper convex, $A$ is an $m\times n$ matrix, $b\in\R^m$, and $k\subseteq \R^m$ is a nonempty closed convex cone. Define the dualizing parameterization for this problem to be
    \begin{align}
        \label{eq:2}
        F(x, u) = {
            \begin{cases}
                f(x) & Ax-b+u\in K \\
                +\infty & \text{otherwise}.
            \end{cases}
        }
    \end{align}
    In this problem, you will demonstrate that with $F$ defined in this manner, the augmented Lagrangian method takes the form
    \begin{align}
        x^{k+1} & \in \argmin_{x\in\R^n} \braces*{f(x) + \frac{1}{2c_k}\bracks*{\dist_K\parens*{p^k+c_k\parens*{Ax-b}}}^2} \label{eq:3} \\
        p^{k+1} &= \proj_{K^*}\parens*{p^k+c_k\parens*{Ax^{k+1}-b}}. \label{eq:4}
    \end{align}
    Throughout this assignment, we use the following notation: if $C$ is any nonempty closed convex set, $\proj_C(v)$ denotes projection from $v$ onto $C$
    \[\proj_C(v)=\argmin_{w\in C}\braces*{\norm*{w-v}},\]
    and $\dist_C(v)$ denotes distance from $v$ to $C$
    \[\dist_C(v)=\inf_{w\in C}\braces*{\norm*{w-v}}=\norm*{\proj_C(v)-v}.\]
    And $K^*$ denotes the polar of $K$.
    \begin{enumerate}[(a)]
        \item Derive an expression (in terms of $f$, $A$, $b$, and $K^*$) for the Lagrangian $L(x, p)$ for the form of $F$ given above.
        \item Let $C\subseteq \R^m$ be any nonempty closed convex cone, and $v\in\R^m$. Show that $w=\proj_C(v)$ if and only if $w\in C$, $v-w\in C^*$, and $\ip*{v-w, w} = 0$ (equivalently, $\ip*{v, w}=\norm*{w}^2$). Hint: refer to Homework 5 Problem 3(b).
        \item For $C\subseteq \R^m$ still defined to be any closed convex cone, show that for any vector $v\in\R^m$ and scalar $\alpha \geq 0$, one has $\proj_C(\alpha v)=\alpha \proj_C(v)$.
        \item Show that for any closed convex cone $C\in\R^m$ and vector $v\in\R^m$, one has $\proj_{C^*}(v)=v-\proj_C(v)$.
        \item Show that for any scalar $c>0$ and $l,t\in\R^m$, the solution to the problem $\min\set{\norm{l-cu}^2\mid u+t\in K}$ is $u=\proj_K(t+\frac{1}{c}l)-t$.
        \item Using Proposition 3.8 and formulas (3.23)-(3.24) of the class notes, show that the augmented Lagrangian method corresponding to the $F$ given in~\cref{eq:2} is~\cref{eq:3,eq:4} above. Note: in answering, it may be helpful to refer to the earlier parts of this problem.
        \item Drawing again upon the Proposition 3.8 from the notes, show that when one executes algorithm~\cref{eq:3,eq:4} for an instance of problem~\cref{eq:1} possessing a dual optimal solution, $\braces{p^k}$ converges to a dual solution, $\lim_{k\rightarrow\infty}\braces{\dist_K(Ax-b)}=0$, and $\lim\sup_{k\rightarrow\infty}f(x^k)\leq\inf\set{f(x)\mid x\in\R^n, Ax+b\in K}$.
    \end{enumerate}
\end{problem}

\begin{solution}
    {Solution}
    \begin{enumerate}[(a)]
        \item {
            \begin{align*}
                L(x, p)
                &= \inf_{u\in\R^m} \braces*{F(x,u)-\ip*{u,p}} \\
                &= f(x)+\ip*{p, Ax-b}+\delta_K^*(p) \\
                &= f(x)+\ip*{p, Ax-b}+\delta_{K^*}(p).
            \end{align*}
        }
        \item {We prove this by showing the two directions.
            \begin{proof}
                Suppose $w$ satisfies the conditions: $w\in C$, $v-w\in C^*$, and $\ip{v-w, w}=0$. Then for any $u\in C$, we have $\ip{v-w, u}\leq 0$ by the definition of polar cone. Using the Pythagorean Theorem, $\norm{v-u}^2=\norm{v-w}^2+\norm{w-u}^2+2\ip{v-w, u-w}\Rightarrow\norm{v-u}^2\geq\norm{v-w}^2$ with equality if and only if $u=w$. Therefore, $w$ minimizes the distance to $v$ over all $u\in C$, which means $w=\proj_C(v)$.

                Now suppose $w=\proj_C(v)$. By the definition of projection, $w\in C$. For any $u\in C$, the function $\phi(u)=\norm{v-u}^2$ is minimized at $u=w$, which means the derivative of $\phi$ at $w$ must be $0$ in any direction inside $C$. Expanding $\phi(u)=\norm{v}^2+\norm{u}^2-2\ip{v, u}$ and $-2(v-w)$ must be orthogonal to $C$ at $w$ to satisfy the derivative condition. Therefore, $-2(v-w)\in C^*$, which means $\ip{v-w, w}=0$ and $v-w$ lies in the polar cone $C^*$.

                The two directions are proved.
            \end{proof}
        }
        \item {
            \begin{align*}
                \proj_C(\alpha v)
                &= \argmin_{w\in C}\norm{w-\alpha v} \\
                &= \argmin_{w\in C}\norm{\frac{w}{\alpha}-v}\cdot \alpha \\
                &= \alpha \argmin_{u\in C}\norm{u-v} \\
                &= \alpha \proj_C(v).
            \end{align*}
            The fact $u\in C$ because $C$ is a cone.
        }
        \item {
            Considering $\proj_C(v)=\argmin_{x\in C}\norm{x-v}$ has similar form as $\prox$, $\proj_C(v)+\proj_{C^*}(v)=v$ follows the same logic as Q1. Therefore, $\proj_{C^*}(v)=v-\proj_C(v)$.
        }
        \item {
            Let $v=u+t\in K$, then $\norm{l-cu}^2=\norm{l-c(v-t)}^2=\norm{cv-(ct+l)}^2$. Based on the definition of projection, $\argmin_{v\in K}\norm{cv-(ct+l)}^2 \Leftrightarrow v=\proj_{K}(ct+l)/c$. Then $u=\proj_{K}(ct+l)/c-t=\proj_{K}(t+\frac{1}{c}l)-t$ using Q2c.
        }
        \item {
            Proposition 3.8 of the class notes shows
            \begin{align*}
                (x^{k+1}, u^{k+1})
                &\in\argmin_{x\in\R^n, u\in\R^m}\braces*{F(x, u)+\frac{1}{2c_k}\norm*{p^k-c_ku}^2} \\
                &\Rightarrow \min_{x\in\R^n, u\in\R^m}\braces*{F(x,u)-\ip*{p^k, u}+\frac{c_k}{2}\norm*{u}^2} \\
                &= \min_{x\in\R^n}\braces*{\min_{u\in\R^m}\braces*{F(x,u)-\ip*{p^k, u}+\frac{c_k}{2}\norm*{u}^2}}.
            \end{align*}
            From Q2e, we have $u=\proj_{K}(t+\frac{1}{c}l)-t=\proj_{K}(Ax-b+\frac{p^k}{c_k})-(Ax-b)=\proj_{K}(c_k(Ax-b)+p^k)-(Ax-b)$. The former argument lets $t=Ax-b$ and the later argument uses Q2c. Then by the definition of $\dist$,
            \begin{align*}
                (x^{k+1}, u^{k+1})
                &\in\argmin_{x\in\R^n, u\in\R^m}\braces*{F(x, u)+\frac{1}{2c_k}\norm*{\proj_{K}(c_k(Ax-b)+p^k)-(Ax-b)}^2} \\
                \Rightarrow x^{k+1} & \in \argmin_{x\in\R^n} \braces*{f(x) + \frac{1}{2c_k}\bracks*{\dist_K\parens*{p^k+c_k\parens*{Ax-b}}}^2}.
            \end{align*}
            The inner minimization is achieved.
            The $p^{k+1} = \proj_{K^*}\parens*{p^k+c_k\parens*{Ax^{k+1}-b}}$ is proved with Q2d.
        }
        \item The convergence of $\set{p^k}$ can be proved with Lemma 3.7 by setting $r\leftarrow p^k$ and $c\leftarrow c^k$. The conclusion of the lemma is that $\prox_{c_k\phi^*}(p^k)=p^k-c_ku^{k+1}$ have $p^{k+1}=\prox_{c_k\phi^*}(p^k)$. Then when there exists a minimizer of $\phi^*$, the PMA guarantees that $\set{p^k}$ must converge to it. When the dual optimal solution exists, we have
        \[p^{k+1}-p^k=-c_ku^{k+1}\Rightarrow c_ku^{k+1}=p^k-p^{k+1}.\]
        When $\set{c_k}$ is bounded away from $0$, $\lim_{k\rightarrow\infty}u^{k+1}=\lim_{k\rightarrow\infty}\braces*{\dist_K(Ax-b)}=0$. Now fix the iteration number $k$ and suppose the $(x^{k+1},u^{k+1})$ is a minimizer (optimal solution), then
        \[F(x^{k+1}, u^{k+1})-\ip{p^k, u^{k+1}}+\frac{c_k}{2}\norm{u^{k+1}}^2\leq F(x, 0)-\ip{p^k, 0}+\frac{c_k}{2}\norm{0}^2=F(x,0).\]
        The second and third term of the left-hand side converges to $0$ as explained above, we apply the $\lim\sup$ operator to the inequality and get
        \[\lim\sup_{k\rightarrow\infty}F(x^{k+1},u^{k+1})\leq\inf\set{F(x,0)\mid x\in\R^n, Ax+b\in K},\]
        where $F(x^{k+1},u^{k+1})=f(x^{k+1})$ and $F(x,0)=f(x)$ then finish the proof.

        Note: Since it is asked for dual optimal solution, I was wandering maybe I should work around $F^*(0,p)$ instead of $F(x,0)$.
    \end{enumerate}
\end{solution}

\begin{problem}
    {Another way to embed Fenchel duality into the parametric duality framework.}
    Consider two closed proper convex functions $f:\R^n\rightarrow\R\cup\set{+\infty}$ and $g:\R^m\rightarrow\R\cup\set{+\infty}$, and an $m\times n$ matrix $M$. Page 67 of the class notes shows one way of modeling the problem $\min_{x\in\R^n}\braces{f(x)+g(Mx)}$ within the gramework of parametric duality: making the decision variables $(x,z)\in\R^n\times \R^m$, changing the objective to $f(x)+g(z)$ and including a constraint $Mx-z=0$. This problem will explore a different approach to accomplishing the same thing by defining $F:\R^n\times\R^m\rightarrow\R\cup\set{+\infty}$ through $F(x,u)=f(x)+g(Mx+u)$.
    \begin{enumerate}[(a)]
        \item Prove that this choice of $F$ is closed proper convex (and that this is true even if $f+g\circ M$ is not proper, that is, $f(x)+g(Mx)=+\infty$ for all $x\in\R^n$).
        \item Find an explicit expression for the corresponding Lagrangian $L(x,p)$ in terms of $f$, $M$, and the conjugate function $g^*$ of $g$. Be sure to completely describe all the cases in which $L(x,p)=\pm\infty$.
        \item Prove that the resulting dual function $\phi^*:p\mapsto D(0,p)$ is the same as the dual function obtained on page 67 of the notes, and is also equal to $-Q(p)$, where $Q$ is the Fenchel dual problem obtained in (1.11) on page 12 of the notes.
        \item Show that, with $F(x,u)=f(x)+g(Mx+u)$, the abstract augmented Lagrangian method (3.17)-(3.18) in Proposition 3.8 of the notes (page 59), reduces to the same algorithm as (3.46)-(3.47) on page 67.
    \end{enumerate}
\end{problem}

\begin{solution}
    {Solution}
    \begin{enumerate}[(a)]
        \item {
            \begin{proof}
                Since $f$ and $g$ are closed functions, their epigraphs are closed and the summation of two closed functions are closed.

                \begin{align*}
                    F(\lambda(x_1, u_1)+(1-\lambda)(x_2, u_2))
                    &= F(\lambda x_1+(1-\lambda)x_2, \lambda u_1+(1-\lambda)u_2) \\
                    &= f(\lambda x_1+(1-\lambda)x_2)+g(M(\lambda x_1+(1-\lambda)x_2)+\lambda u_1+(1-\lambda)u_2) \\
                    &\leq \lambda f(x_1)+\lambda g(Mx_1+u_1)+(1-\lambda)f(x_2)+(1-\lambda)g(Mx_2+u_2) \\
                    &= \lambda F(x_1, u_1)+(1-\lambda)F(x_2, u_2),
                \end{align*}
                which shows convexity.

                We can prove the proper function using prove by contradiction.  Suppose $F(x, u)=f(x)+g(Mx+u)=+\infty$ for all $(x, u)\in\R^n\times\R^m$. Since $f$ and $g$ are proper, there exist $x_0\in \R^n$ and $z_0\in\R^m$ such that $f(x_0)$ and $g(z_0)$ are finite. Let $u_0=z_0-Mx_0$, then $f(x_0)+g(Mx_0+u_0)=f(x_0)+g(z_0)$ is finite, which contradicts the assumption. Therefore, $F$ is proper.
            \end{proof}
        }
        \item The Lagrangian associated with the minimization problem is \[L(x, p) = F(x,u)-p\trn (Mx+u)=f(x)+g(Mx+u)-p\trn (Mx+u).\] The conjugate function is \[g^*=\sup_{u\in\R^m}\braces*{p\trn (Mx+u)-g(Mx+u)}.\] Then the Lagrangian can be expressed as $L(x, p) = f(x)+g^*(x)$. $L(x,p)=+\infty$  if either $f(x)$ or $g^*(x)$ is $+\infty$. $L(x,p)=-\infty$ never occurs since $f$ and $g^*$ are proper.
        \item {
            The dual function of the form
            \begin{align*}
                D(0,p)
                &= \inf_{x\in\R^n, u\in\R^m}\braces*{f(x)+g(Mx+u)-p\trn (Mx+u)} \\
                &= \inf_{x\in\R^n} \braces*{f(x)+g(Mx)-p\trn Mx}+\inf_{u\in\R^m}\braces*{g(u)-p\trn u} \\
                &= \inf_{x\in\R^n} \braces*{f(x)+g(Mx)-p\trn Mx}+\inf_{u\in\R^m}\braces*{g(u)-\ip*{p, u}} \\
                &= -(\sup_{x\in\R^n}\braces*{-f(x)-g(Mx)+\ip{M\trn p,x}}+\sup_{u\in\R^m}\braces*{-g(u)+\ip*{p, u}}) \\
                &= -(f^*(-M\trn p)+g^*(p)) \\
                &= -Q(p).
            \end{align*}
        }
        \item {
            \begin{align*}
                (x^{k+1}, u^{k+1}) &\in \argmin_{x\in\R^n, u\in\R^m}\braces*{F(x, u)-\ip{p^k, u}+\frac{c_k}{2}\norm{u}^2} \\
                &\Leftrightarrow \argmin_{x\in\R^n, u\in\R^m}\braces*{f(x)+g(Mx+u)-(p^k)\trn (Mx+u)} \\
                p^{k+1} &= p^k-c_ku^{k+1} \\
                &= p^k-c_k(Mx^{k+1}+u^{k+1}) \\
                \Rightarrow
                x^{k+1} &\in \argmin_{x\in\R^n}\braces*{f(x)+(p^k)\trn Mx} \\
                u^{k+1} &\in \argmin_{u\in\R^m}\braces*{g(Mx+u)-(p^k)\trn (Mx+u)} \\
                p^{k+1} &= p^k-c_k(Mx^{k+1}+u^{k+1}) \\
                &\text{substituting the setup into the augmented Lagrangian method} \\
                x^{k+1} &\in \argmin_{x\in\R^n}\braces*{f(x)+g(Mx+u)-(p^k)\trn (Mx+u)+\frac{c_k}{2}\norm{Mx+u}^2} \\
                p^{k+1} &= p^k-c_k(Mx^{k+1}+u^{k+1}) 
            \end{align*}
        }
    \end{enumerate}
\end{solution}

\begin{problem}
    {The ADMM with indicators of affine sets.}
    A very common application of the ADMM for problems of the form $\min_{x\in\R^n}\braces{f(x)+g(Mx)}$ is when $g$ is the indicator function of an affine set $S+d$, where $S\subset\R^m$ is a linear subspace and $d\in\R^m$ (when $d=0$, the set in question is just the linear subspace $S$). This special case is equivalent to minimizing $f(x)$ subject to the constraint that $Mx\in S+d$, or equivalently $Mx-d\in S$. This problem will explore the special properties of the ADMM in this context.
    \begin{enumerate}[(a)]
        \item Give a formula for the normal cone map $N_{S+d}$ of the set $S+d$, which is also the subgradient map of the indicator function $\delta_{S+d}$, proving that your formula is correct (your answer should involve the subspaces $S^{\bot}\subseteq\R^m$ orthogonal to $S$).
        \item In terms of the projection operator $\proj_S$ from $\R^m$ to $S$ and/or the projection operator $\proj_{S^{\bot}}=\mathrm{Id}-\proj_S$ from $\R^m$ to $S^{\bot}$, state and justify a procedure for computing the projection $\proj_{S+d}(y)$ of a given point $y\in\R^m$ onto $S+d$.
        \item Prove that for any $y\in\R^m$ and $w\in S^{\bot}$,$\proj_{S+d}(y+w)=\proj_{S+d}(y)$.
        \item For an arbitrary $f:\R^n\rightarrow\R\cup\set{+\infty}$ and $m\times n$ matrix $M$, consider using the ADMM to solve minimize $f(x)+\delta_{S+d}(Mx)$. Show that if the starting Lagrange multiplier estimates $p^0$ lie in $S^{\bot}$, then all the subsequent Lagrange multiplier estimates $p^k$ also lie in $S^{\bot}$, and the algorithm takes the form
        \begin{align*}
            x^{k+1} &\in \argmin_{x\in\R^n}\braces*{f(x)+\ip*{p^k,Mx}+\frac{c}{2}\norm*{Mx-z^k}^2} \\
            z^{k+1} &= \proj_{S+d}(Mx^{k+1}) \\
            p^{k+1} &= p^k+c(Mx^{k+1}-z^{k+1}).
        \end{align*}
    \end{enumerate}
\end{problem}

\begin{solution}
    {Solution}
    \begin{enumerate}[(a)]
        \item The normal cone at a point $y$ in a set $C$ in $\R^m$ is defined as \[N_C(y)=\set{v\in\R^m \mid \ip{v, z-y}\leq 0, \forall z\in C}.\]
        For the set $S+d$, where $S$ is a linear subspace of $\R^m$ and $d$ is a vector in $\R^m$, the normal cone is $N_{S+d}(y)$ at a point $y\in S+d$ is the set of all vectors orthogonal to $S$. Thus, $N_{S+d}(y)=S^{\bot}$, the orthogonal complement of $S$. We first see that a vector $v$ belongs to $N_{S+d}(y)$ if and only if $\ip{v, z-y}\leq 0$ for all $z\in S+d$. Since $S$ is a linear subspace, for any $z\in S+d$, $z-y\in S$. Therefore, $v$ is orthogonal to every vector S, which means $v\in S^{\bot}$.
        \item To project a point $y\in\R^m$ onto the set $S+d$, we can do following procedures. First, translate $y$ by $-d$ to align with the subspace $S$, which yield $y'=y-d$. Next, project $y'$ onto $S$, thus $\proj_S(y')=\proj_S(y-d)$. Finally, translate the result of the projection back by d to align with the original set $S+d$. Therefore, $\proj_{S+d}(y)=\proj_S(y-d)+d$. I got this inspired by figure 3.5 on page 75.
        \item From Q4b, we have shown that $\proj_{S+d}(z)=\proj_S(z-d)+d$. We first project $y+w$ onto $S+d$ and get $\proj_{S+d}(y+w)=\proj_S((y+w)-d)+d$. Decompose $y+w-d$, it is readily to see that $y-d\in S$ and $w\in S^{\bot}$. Since the projection of a vector onto $S$ only depends on the component in $S$, then $\proj_S((y+w)-d)=\proj_S(y-d)$. Therefore, $\proj_{S+d}(y+w)=\proj_S(y-d)+d=\proj_{S+d}(y)$.
        \item Observing the given algorithm, we notice that $z^{k+1}=\proj_{S+d}(Mx^{k+1})$ implies $z^{k+1}-d\in S$ and $Mx^{k+1}-z^{k+1}\in S$ since $z^{k+1} \in S+d$. Then $p^{k+1}=p^k+c(Mc^{k+1}-z^{k+1})$ has $p^k\in S^{\bot}$ and $Mx^{k+1}-z^{k+1}\in S$, which update $p^{k+1}$ within $S^{\bot}$. This is because $S^{\bot}$ is closed under addition with elements from $S$ due to orthogonality. Therefore, $p^{k+1}\in S^{\bot}$ if $p^0\in S^{\bot}$.
    \end{enumerate}
\end{solution}

\begin{problem}
    {Multi-block decomposition through the ADMM.}
    The principal use of the ADMM specialization discussed in the previous problem is to induce decomposition for block-separable problems and similar settings. As an example, consider problems of the form (3.4) in the class notes,
    \begin{alignat*}{9}
        &\min \quad&& f_1(x_1)&&+&&f_2(x_2)&&+&&\cdots&&+&&f_l(x_l)& \\
        &\st  && A_1x_1&&+&&A_2x_2&&+&&\cdots&&+&&A_lx_l&=b,
    \end{alignat*}
    where $b\in\R^m$, the overall the decision vector $x$ is a concatenation of subvectors $(x_1,x_2,\ldots,x_l) \in \R^{n_1} \times \R^{n_2} \times \cdots \times \R^{n_l}$ and, for each $i=1,2,\ldots,n$, one has that $f_i:\R^{n_i}\rightarrow\R\cup\set{+\infty}$ is a closed proper convex function and $A_i$ is an $m\times n_i$ matrix. It was shown in class and in the notes that Lagrangian decomposition algorithms achieve decomposition for such problems but augmented Lagrangian do not, due to the cross terms in the quadratic penalty $\norm*{\sum_{i=1}^lA_ix_i-b}^2$. Consider modeling problems in the form (3.4) in Fenchel-Rockafellar form, starting as follows:
    \[f(x)=f(x_1,x_2,\ldots,x_l)=\sum_{i=1}^lf_i(x_i)\qquad M=\begin{bmatrix}
        A_1 &&& \\
        & A_2 && \\
        && \ddots & \\
        &&& A_l
    \end{bmatrix}.\]
    Note that $M$ has $n=\sum_{i=1}^{n} n_i$ columns and $lm$ rows. Consider vectors $z=Mx$ of length $lm$ to be written as consisting of $l$ subvectors $(z_1, z_2, \ldots, z_l)$, each of length $m$. Finally, define $g:\R^{lm}\rightarrow\R\cup\set{+\infty}$ by
    \[g(z)=g(z_1,z_2,\ldots,z_l)=\begin{cases}
        0 & \text{if } z_1+z_2+\cdots+z_l=b \\
        +\infty & \text{otherwise}.
    \end{cases}\]
    \begin{enumerate}[(a)]
        \item Show that $x$ is a minimizer of $f(x)+g(Mx)$ with the setup above if and only if it is a minimizer of the block-separable problem (3.4).
        \item Show that applying the ADMM to this setup is equivalent to the recursion pattern below, where
        \begin{itemize}
            \item $c>0$ is a fixed scalar,
            \item $p^k\in\R^m$ for each iteration $k\geq 0$
            \item $z_i^k\in\R^m$ for each $i=1,2,\ldots,l$ and iteration $k\geq 0$
            \item $x_i^k\in\R^{n_i}$ for each $i=1,2,\ldots,l$ and iteration $k\geq 0$.
        \end{itemize}
        \begin{align*}
            x_i^{k+1} &\in \argmin_{x_i\in\R^{n_i}}\braces*{f_i(x_i)+\ip*{p^k,A_ix_i}+\frac{c}{2}\norm*{A_ix_i-z_i^k}^2} \qquad \forall i=1,2,\ldots,l\\
            z_i^{k+1} &= A_ix_i^{k+1}+\frac{1}{l}\parens*{b-\sum_{j=1}^lA_jx_j^{k+1}} \qquad \forall i=1,2,\ldots,l \\
            p^{k+1} &= p^k+\frac{c}{l}\parens*{\sum_{i=1}^lA_ix_i^{k+1}-b}.
        \end{align*}
        You may find it helpful to use various parts of the preceding problem, along with the basic linear algebra fact that if $B$ is a $p\times q$ matrix, the orthogonal complement of the subspace $\ker B \dot{=}\set{u\in\R^q\mid Bu=0}$ is $\set{B^{\top}v\mid v\in\R^p}$.
    \end{enumerate}
\end{problem}

\begin{solution}
    {Solution}
    To show the proof, let's first state the problem (3.4) in the form of Fenchel-Rockafellar form.

    For the giving optimization problem, 
    
\end{solution}

\end{document}