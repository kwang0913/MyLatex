\documentclass{article}
\usepackage[
    letterpaper,
    top=1in,
    bottom=1in,
    inner=0.75in,
    outer=0.75in,
    % margin=1in,
]{geometry}

%%%% Page Header and Foot %%%%
\usepackage{fancyhdr}
\fancypagestyle{plain}{
\fancyhf{}
\fancyhead[L]{266:711:685}
\fancyhead[C]{\textbf{Convex Analysis and Optimization}}
\fancyhead[R]{Rutgers University}
\fancyfoot[L]{}
\fancyfoot[C]{\thepage}
\fancyfoot[R]{}
}
\pagestyle{plain}

%%%% Load Format %%%%
%%%% Page Margin %%%%
\RequirePackage{geometry}

%%%% Page Header and Foot %%%%
\RequirePackage{fancyhdr}
\pagestyle{fancyplain}
\lhead{\lefthead}
\chead{\centerhead}
\rhead{\righthead}
\lfoot{\leftbottom}
\cfoot{\centerbottom}
\rfoot{\rightbottom}

%%%% Additional Packages to Load %%%%
%%%% Main Font
\RequirePackage{lmodern} %Computer Modern, but with many more glyphs
\RequirePackage{microtype} %Better Font Rendering
\RequirePackage{libertinus} %Very popular math font
\RequirePackage[T1]{fontenc} %Better Printing Font
\RequirePackage[utf8]{inputenc} %enable utf8 encoding
\RequirePackage[english]{babel} %English Language

%%%% Math Env
\RequirePackage{amsfonts,amsmath,amssymb} %Math Font Packages
\RequirePackage{mathtools} %Better Math Environments
\RequirePackage{mleftright}  %Fixes some annoying spacing issues
\RequirePackage{bm} %Boldface math, load after other math font packages

%%%% Beautiful Stuff
\RequirePackage{graphicx} %Better Graphics
\RequirePackage[export]{adjustbox} %Better Graphics
\RequirePackage[dvipsnames,svgnames]{xcolor} %Must load before tcolorbox
\RequirePackage[breakable]{tcolorbox} %Customized Box
\tcbuselibrary{skins}
\RequirePackage{transparent} 

%%%% Table and List 
\RequirePackage{tabularx} %Better Table
\RequirePackage{multirow}
\RequirePackage{colortbl}
\RequirePackage{array}

%%%% Theorems etc (also  problems)
\RequirePackage[shortlabels]{enumitem} %Better List
\RequirePackage{hyperref, cleveref} %Fantastic crossrefer
\RequirePackage{amsthm, thmtools} %Better Theorems
\RequirePackage{thm-restate} %Better Theorems

%%%% Code Packages
\RequirePackage{algorithm2e} % Package to create pseudo-code
\RequirePackage{listings} % Package to insert code
% \RequirePackage{keyval}

%%%% Display Packages
\RequirePackage[document]{ragged2e} % Package to justify text
\RequirePackage{csquotes} % Package to facilities quotations
\RequirePackage{multicol} % Package to use multicols

%%%% Other Packages
\RequirePackage{calc}
\RequirePackage{rotating}
\RequirePackage{pgf, tikz}
\RequirePackage{epstopdf}
\RequirePackage[backend=biber, style=numeric, sorting=none]{biblatex}

%%%% A box fill the rest of the page
\newtcolorbox{stretchbox}[1][]{
    height fill,
    sharp corners,
    colback=white,
    % colframe=yellow!50!black,
    #1
    }

%%%% Box Environment
\newtcolorbox{prob}[1]{
% Set box style
    sidebyside,
    sidebyside align=top,
% Dimensions and layout
    width=\textwidth,
    toptitle=2.5pt,
    bottomtitle=2.5pt,
    righthand width=0.20\textwidth,
% Coloring
    colbacktitle=white,
    coltitle=black,
    colback=white,
    colframe=black,
% Title formatting
    title={
        #1 \hfill Grade:\hspace*{0.15\textwidth}
    },
    fonttitle=\large\bfseries
}

%%%% Problem Environment 
\newenvironment{problem}[1]{
    \begin{prob}{#1}
}
{
    \tcblower
    \centering
    \textit{\scriptsize\bfseries Faculty Comments}
    \end{prob}
}

%%%% Draw a line
\newcommand{\myrule}{\rule{1.5in}{0.1mm}}
%%%% delimiters
\DeclarePairedDelimiter\parens{\lparen}{\rparen}
\DeclarePairedDelimiter\bracks{\lbrack}{\rbrack}
\DeclarePairedDelimiter\braces{\lbrace}{\rbrace}
\DeclarePairedDelimiter\abs{\lvert}{\rvert}
\DeclarePairedDelimiter\norm{\lVert}{\rVert}
\DeclarePairedDelimiter\angles{\langle}{\rangle}
\DeclarePairedDelimiter\ceil{\lceil}{\rceil}
\DeclarePairedDelimiter\floor{\lfloor}{\rfloor}

%%%% math operators naming
\DeclareMathOperator*{\argmax}{\textnormal{argmax}}
\DeclareMathOperator*{\argmin}{\textnormal{argmin}}
\DeclareMathOperator{\tr}{\textnormal{tr}}
\DeclareMathOperator{\eig}{\textnormal{eig}}
\DeclareMathOperator{\sgn}{\textnormal{sgn}}
% \let\det\relax % "Undefine" \det
% \DeclareMathOperator{\det}{\textnormal{det}} % already defined in mathtools
\DeclareMathOperator{\diag}{\textnormal{diag}}
\DeclareMathOperator{\rank}{\textnormal{rank}}
\DeclareMathOperator{\Vol}{\textnormal{Vol}}   % volume
\DeclareMathOperator{\Surf}{\textnormal{Surf}} % surface area

%%%% Transforms! -- requires mathtools package
\newcommand*{\LapTrans}{\xleftrightarrow{\mathcal{Z}}}
\newcommand*{\ZTrans}{\xleftrightarrow{\mathcal{L}}}
\newcommand*{\CTFS}{\xleftrightarrow{\textnormal{CTFS}}}
\newcommand*{\CTFT}{\xleftrightarrow{\textnormal{CTFT}}}
\newcommand*{\DTFS}{\xleftrightarrow{\textnormal{DTFS}}}
\newcommand*{\DTFT}{\xleftrightarrow{\textnormal{DTFT}}}

%%%% vector font
\let\oldvec\vec
\renewcommand*{\vec}[1]{\mathbf{#1}}
% \newcommand*{\trn}{\!^{\!\intercal}}
\newcommand*{\trn}{\!^{\mathsf{T}}}
% \newcommand*{\coj}{\!^{\dag}} % Text Mode Symbol, Should not be used
% \newcommand*{\coj}{\!^{\dagger}}
\newcommand*{\coj}{\!^{\mathsf{H}}}
\newcommand*{\inv}{^{-1}}

%%%% number systems
\DeclareMathOperator{\R}{\mathbb{R}}
\DeclareMathOperator{\C}{\mathbb{C}}
\DeclareMathOperator{\N}{\mathbb{N}}
\DeclareMathOperator{\Z}{\mathbb{Z}}
\DeclareMathOperator{\F}{\mathbb{F}}
\DeclareMathOperator{\Q}{\mathbb{Q}}

%%%% STATISTICS AND PROBABILITY
\newcommand*{\Var}{\mathop{\textnormal{Var}}}
\newcommand*{\Cov}{\mathop{\textnormal{Cov}}}
\newcommand*{\Corr}{\mathop{\textnormal{Corr}}}
\newcommand*{\MSE}{\mathop{\textnormal{MSE}}}
\newcommand*{\MSD}{\mathop{\textnormal{MSD}}}
\newcommand*{\NSD}{\mathop{\textnormal{NSD}}}

\newcommand*{\E}[1]{\mathbb{E}\bracks*{#1}}
\newcommand*{\condE}[2]{\mathbb{E}\bracks*{#1 \mid #2}}
\renewcommand*{\P}[1]{\mathbb{P}\parens*{#1}}
\newcommand*{\condP}[2]{\mathbb{P}\parens*{#1 \mid #2}}

\DeclareMathOperator{\Bern}{\mathsf{Bern}}
\DeclareMathOperator{\Unif}{\mathsf{Unif}}
\DeclareMathOperator{\Expv}{\mathsf{Exp}}
\DeclareMathOperator{\Poi}{\mathsf{Poi}}
\DeclareMathOperator{\Gamv}{\mathsf{Gamma}}
\DeclareMathOperator{\Dirv}{\mathsf{Dir}}
\DeclareMathOperator{\Mult}{\mathsf{Mult}}
\DeclareMathOperator{\Beta}{\mathsf{Beta}}
\DeclareMathOperator{\Geomv}{\mathsf{Geom}}
\DeclareMathOperator{\Binomv}{\mathsf{Binom}}
\DeclareMathOperator{\NegBinomv}{\mathsf{NB}}
\DeclareMathOperator{\Lap}{\mathsf{Lap}}
\DeclareMathOperator{\Gaus}{\mathsf{N}}
\DeclareMathOperator{\Weibull}{\mathsf{Weibull}}


\DeclareMathOperator{\iidsim}{\stackrel{\textnormal{i.i.d.}}{\sim}}
\DeclareMathOperator{\diff}{\mathop{}\!\textnormal{d}}

%%%% Special norms and linear algebra stuff
\newcommand*{\subgnorm}[1]{\norm*{#1}_{\psi_2}}
\newcommand*{\subexpnorm}[1]{\norm*{#1}_{\psi_1}}
\newcommand*{\frobnorm}[1]{\norm*{#1}_{\textnormal{F}}}
\newcommand*{\opnorm}[1]{\norm*{#1}_{\textnormal{op}}}
\newcommand*{\Lipnorm}[1]{\norm*{#1}_{\textnormal{Lip}}}

%%%%
\newcommand*{\set}[1]{\braces*{\,#1\,}}
\newcommand*{\ie}{\textnormal{i.e.\ }}
\newcommand*{\eg}{\textnormal{e.g.\ }}
\newcommand*{\etc}{\textnormal{etc.\ }}
\newcommand*{\iid}{\textnormal{i.i.d.\ }}
\allowdisplaybreaks

%%%% Theorem Style %%%%
\declaretheorem[numbered=no, style=plain]{axiom, lemma}
\declaretheorem[numberwithin=section,style=definition]{definition}
\declaretheorem[sibling=definition]{theorem, corollary, proposition, conjecture}
\declaretheorem[numbered=no,style=remark]{remark, claim}

%%%% Document Information %%%%
\title{Homework 1}
\author{Kailong Wang}
\date{\today}

%%%% Start Document %%%%
\begin{document}
\maketitle
% \tableofcontents
\begin{problem}
    {Q1: Affine images and preimages of convex sets.}
    Let $A\in \R^{m\times n}, b\in\R^m, C\subset\R^n, D\subset\R^m$ be convex sets. Show that following sets are convex.
    \begin{enumerate}[(a)]
        \item The image of $C$ under the affine map $x\mapsto Ax+b$. That is {
            \[
                \{Ax+b\mid x\in C\}\subset\R^m.
            \]
        }
        \item The preimage of $D$ under the affine map $x\mapsto Ax+b$. That is {
            \[
                \{x\mid Ax+b\in D\}\subset\R^n.
            \]
        }
    \end{enumerate}
\end{problem}

\begin{solution}
    {Solution}
    \begin{enumerate}[(a)]
        \item  {
            \begin{proof}
                Let $x_1,x_2\in C$ and $\lambda\in[0,1]$, then we have {
                \begin{align*}
                    \lambda(Ax_1+b)+(1-\lambda)(Ax_2+b) &= A(\lambda x_1+(1-\lambda)x_2)+b\\
                    &\in A(C)+b.
                \end{align*}
                }
            Thus, the image of $C$, $A(C)+b$ is convex.
            \end{proof}
        }
        \item {
            \begin{proof}
                Let $y_1,y_2\in A^{-1}(D-b)$ so that $Ay_1+b\in D, Ay_2+b\in D$ and $\lambda\in[0,1]$, then we have {
                \begin{align*}
                    A(\lambda y_1+(1-\lambda)y_2)+b &= \lambda(Ay_1+b)+(1-\lambda)(Ay_2+b)\\
                    &\in \lambda D+(1-\lambda)D\\
                    &= D.
                \end{align*}
            }
            Thus, The preimages of $D$, $A^{-1}(D-b)$ is convex.
            \end{proof}
        }
    \end{enumerate}
\end{solution}

\begin{problem}
    {Q2: Affine functions.}
    Suppose that $f:\R^n\rightarrow\R\backslash\{-\infty,\infty\}$ always obey the convex function relation at equality, that is,
    \begin{equation}
        \label{eq:convex}
        f(\lambda x+(1-\lambda)y) \leq \lambda f(x)+(1-\lambda)f(y),\quad \forall x,y\in\R^n,\lambda\in[0,1].
    \end{equation}
    Show that
    \begin{enumerate}[(a)]
        \item If \cref{eq:convex} holds as stated for all $\lambda\in[0,1]$, it in fact holds for all $\lambda\in\R$.\label{item:convex}
        \item Any $f$ for which \cref{eq:convex} holds must be of the form $f(x)=\angles*{a,x}+b$ for $\lambda\in\R^n,b\in\R$ (that is, $f$ is an inner product with some fixed vector plus a constant).
        \item Any function of this form has the property \cref{eq:convex}.
    \end{enumerate}
    \textit{Hint:} given $f$ satisfying the condition above, show that $g:x\mapsto f(x)\rightarrow f(0)$ is linear. You may then use (without proof, although the proof is very easy) that a linear function $g: \R^n \rightarrow \R$ must be of the form $x \mapsto \angles*{a,x}$ for some $a \in \R^n$.
\end{problem}

\begin{solution}
    {Solution}
    \begin{enumerate}[(a)]
        \item {
            \begin{proof}
                To extend \cref{eq:convex} to $\lambda\in\R^n$, we need to show that \cref{eq:convex} holds for $\lambda\in(-\infty,0)\cup(1,\infty)$.

                First, let $x,y\in\R^n$, and for $\lambda\in(-1,0)$ let $\alpha=-\lambda\in(0,1)$. Then we have
                \[
                    f(\alpha x+(1-\alpha)y) \leq \alpha f(x)+(1-\alpha)f(y),
                \]
                which shows convexity for $\lambda\in(-1,0)$.
                Similarly, for $\lambda\in(1,\infty)$ let $\alpha=\frac{1}{\lambda}\in(0,1)$, and for $\lambda\in(-\infty,-1)$ let $\alpha=-\frac{1}{\lambda}\in(0,1)$, we can prove \cref{item:convex} holds for $\lambda\in(1,\infty)$ and $\lambda\in(-\infty,-1)$ respectively.
            \end{proof}
        }
        \item {
            \begin{proof}
                Let's define $g:\R^n\rightarrow\R$ as $g(x)=f(x)-f(0)$, then we have $g(0)=f(0)-f(0)=0$. For any $x,y\in\R^n$ and $\lambda\in\R$ (as proved above), we have
                \begin{align*}
                    g(\lambda x+(1-\lambda)y) &= f(\lambda x+(1-\lambda)y)-f(0)\\
                    &\leq \lambda f(x)+(1-\lambda)f(y)-f(0)\\
                    &= \lambda(f(x)-f(0))+(1-\lambda)(f(y)-f(0))\\
                    &= \lambda g(x)+(1-\lambda)g(y).
                \end{align*}
                This shows $g$ is a linear function. From the hint, we can represent $g$ as $g(x)=\angles*{a,x}$ for some $a\in\R^n$. Thus, $f(x)=\angles*{a,x}+b$ where $b=f(0)$.
            \end{proof}
        }
        \item {
            \begin{proof}
                If $f(x)=\angles*{a,x}+b$, then for any $x,y\in\R^n$ and $\lambda\in[0,1]$
                \begin{align*}
                    f(\lambda x+(1-\lambda)y) &= \angles*{a,\lambda x+(1-\lambda)y}+b\\
                    &= \lambda\angles*{a,x}+(1-\lambda)\angles*{a,y}+b\\
                    &= \lambda(\angles*{a,x}+b)+(1-\lambda)(\angles*{a,y}+b)\\
                    &\leq \lambda f(x)+(1-\lambda)f(y).
                \end{align*}
            \end{proof}
        }
    \end{enumerate}
\end{solution}

\begin{problem}
    {Q3: Convex hulls.}
    Show that for any set $X\subseteq\R^n$, the convex hull $conv(X)$ of $X$ (the intersection of all convex sets containing $X$) is equal to the set of all convex combinations of points in $X$.

    \textit{Hint:} Define $Y$ to be the set of all convex combinations of points from $X$, that is,
    \[
        Y=\braces*{\sum_{i=1}^m\lambda_ix_i\mid m\geq1, \lambda_i>0,\sum_{i=1}^m\lambda_i=1},
    \]
    and then prove that both $Y\subseteq conv(X)$ (which may be accomplished by showing that it is convex and contains $X$), and $conv(X)\subseteq Y$ (which may be accomplished by showing that every convex set containing $X$ also contains $Y$).
\end{problem}

\begin{solution}
    {Solution}
    \begin{enumerate}
        \item $Y\subseteq conv(X)$ {
            \begin{proof}
                Let $y_1, y_2\in Y$. By definition of $Y$,
                \begin{align*}
                    y_1 &= \sum_{i=1}^{m_1}\alpha_ix_i, \qquad \sum_{i=1}^{m_1}\alpha_i=1,\\
                    y_2 &= \sum_{j=1}^{m_2}\beta_jx_j, \qquad \sum_{j=1}^{m_2}\beta_j=1.
                \end{align*}
                For any $\lambda\in[0,1]$, consider the point $y=\lambda y_1+(1-\lambda)y_2$. Then
                \begin{align*}
                    y&=\lambda\sum_{i=1}^{m_1}\alpha_ix_i+(1-\lambda)\sum_{j=1}^{m_2}\beta_jx_j\\
                    &=\sum_{i=1}^{m_1}(\lambda\alpha_i)x_i+\sum_{j=1}^{m_2}((1-\lambda)\beta_j)x_j\\
                    \text{where}\\
                    &\sum_{i=1}^{m_1}\lambda\alpha_i+\sum_{j=1}^{m_2}(1-\lambda)\beta_j\\
                    &=\lambda\sum_{i=1}^{m_1}\alpha_i+(1-\lambda)\sum_{j=1}^{m_2}\beta_j\\
                    &=\lambda+(1-\lambda)\\
                    &=1.
                \end{align*}
                Clearly, $y\in Y$, which shows $Y$ is convex. Also, every point $x_i\in X$ is in $Y$ with $\lambda_i=1$, which shows $X\subseteq Y$. Since $Y$ is convex and contains $X$, then it must contain $conv(X)$ as $conv(X)$ is the intersection of all convex sets containing $X$.
            \end{proof}
        }
        \item $conv(X)\subseteq Y$ {
            \begin{proof}
                Let $Z$ be any convex set containing $X$. We want to show that $Z$ also contains $Y$. Take any $y\in Y$, since $Z$ is convex and contains $X$, $Z$ must contain $y$, the convex combination of points in $X$. Thus, $Z$ contains $Y$. Since arbitrary $Z$ contains $Y$, $conv(X)$ must contain $Y$ as $conv(X)$ is the intersection of all convex sets containing $X$. Therefore, $conv(X)\subseteq Y$.
            \end{proof}
        }
    \end{enumerate}
\end{solution}

\begin{problem}
    {Q4: Affine sets and hulls.}
    The scalars $\lambda_i$ in this problem may take negative values.
    \begin{enumerate}[(a)]
        \item The textbook defines a set $X\subseteq\R^n$ as being affine if it is of the form $S+x=\braces*{s+x\mid s\in S}$ for some $x\in\R^n$ and linear subspace $S$ of $\R^n$. Show that $X$ is affine according to this definition if and only if $X$ is
        \begin{equation*}
            \begin{rcases}
                & x_1, x_2, \ldots, x_m \in X \\
                & \lambda_1, \lambda_2, \ldots, \lambda_m \in \R \\
                & \sum_{i=1}^m \lambda_i = 1
            \end{rcases}
            \Rightarrow \sum_{i=1}^{m}\lambda_i x_i\in X
        \end{equation*}
        \textit{Hint:} For the ``if'', take any $x\in X$ and show that the set $S=X-x=\braces*{x'-x\mid x'\in X}$ is a linear subspace of $\R^n$.
        \item In the text, the \textit{affine hull} $aff(Y)$ of a set $Y$ is defined to be the intersection of all affine sets containing $Y$. Show that
        \[
            aff(Y)=\braces*{\sum_{i=1}^m\lambda_iy_i\mid m\geq1, \lambda_i\in\R, \sum_{i=1}^m\lambda_i=1, y_i\in Y},
        \]
        that is, the affine hull of $Y$ is the set of all affine combinations of points in $Y$.
    \end{enumerate}
\end{problem}

\begin{solution}
    {Solution}
    \begin{enumerate}[(a)]
        \item {
            \begin{itemize}
                \item ``if'' {
                    \begin{proof}
                        Assume that the condition holds for $X$. We want to show that $X$ is affine by showing $S$ is a linear subspace of $\R^n$. Take any $x\in X$ and let $S=X-x$, we have
                        \begin{enumerate}[(1)]
                            \item $0\in S$ because $x'-x=0$ for $x'=x$ and $x'\in X$.
                            \item For $s_1, s_2,\ldots,s_m\in S$, $\sum_{i=1}^{m}\lambda_i s_i=\sum_{i=1}^{m}\lambda_i(x_i-x)=(\sum_{i=1}^{m}\lambda_i x_i)-x\in S$ for $\sum_{i=1}^{m}\lambda_i=1$.
                            \item For any $s\in S$ and any scalar $\lambda$, $\lambda s=\lambda(x'-x)=[\lambda(x')+(1-\lambda)x]-x\in S$.
                        \end{enumerate}
                        Thus, $S$ is a linear subspace of $\R^n$, which shows $X$ is affine.
                    \end{proof}
                }
                \item ``only if'' {
                    \begin{proof}
                        Suppose $X$ is affine as $X=S+x$ for some $x\in\R^n$ and some linear subspace $S$ of $\R^n$, we want to show that when the conditions hold, $\sum_{i=1}^{m}\lambda_i x_i\in X$. Since $X=S+x$, for any point $x_i=s_i+x$, we have
                        \[
                            \sum_{i=1}^{m}\lambda_i x_i=\sum_{i=1}^{m}\lambda_i(s_i+x)=\sum_{i=1}^{m}\lambda_i s_i+\sum_{i=1}^{m}\lambda_i x=\sum_{i=1}^{m}\lambda_i s_i+x\in X.
                        \]
                    \end{proof}
                }
            \end{itemize}
        }
        \item The proof of this is similar to Q3 without condition $\sum_{i=1}^{m}\lambda_i=1$.
    \end{enumerate}
\end{solution}

\begin{problem}
    {Q5: Arithmetic-Geometric Mean Inequality.}
    Show that if $\lambda_1, \lambda_2, \ldots, \lambda_m$ are positive scalars with $\sum_{i=1}^{m}\lambda_i=1$, then for every set of positive scalars $x_1, x_2, \ldots, x_m$, we have
    \[
        \prod_{i=1}^{m}x_i^{\lambda_i}\leq\sum_{i=1}^{m}\lambda_ix_i,
    \]
    with equality if and only if $x_1=x_2=\cdots=x_m$.

    \textit{Hint:} Show that $-\ln x$ is a strictly convex function on $(0,\infty)$.
\end{problem}

\begin{solution}
    {Solution}
    Consider the function $f(x)=-\ln x$, then $f''(x)=\frac{1}{x^2}>0$ for $x>0$. Thus, $f(x)$ is strictly convex on $(0,\infty)$. By Jensen's inequality, we have {
        \begin{align*}
            -\ln\parens*{\sum_{i=1}^{m}\lambda_ix_i}
            &= f\parens*{\sum_{i=1}^{m}\lambda_ix_i}\\
            &\leq \sum_{i=1}^{m}\lambda_i f(x_i)\\
            &= \sum_{i=1}^{m}\lambda_i(-\ln x_i)\\
            &= -\ln\parens*{\prod_{i=1}^{m}x_i^{\lambda_i}} \\
            \Rightarrow
            \exp\braces*{-\ln\parens*{\sum_{i=1}^{m}\lambda_ix_i}}
            &\leq \exp\braces*{-\ln\parens*{\prod_{i=1}^{m}x_i^{\lambda_i}}}\\
            \Rightarrow
            \sum_{i=1}^{m}\lambda_ix_i &\geq \prod_{i=1}^{m}x_i^{\lambda_i}.
        \end{align*}
    }
    Since $f(x)$ is strictly convex, the equality holds if and only if $x_1=x_2=\cdots=x_m$.
\end{solution}

\end{document}