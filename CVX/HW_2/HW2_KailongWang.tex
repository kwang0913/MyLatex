\documentclass{article}
\usepackage[
    letterpaper,
    top=1in,
    bottom=1in,
    inner=0.75in,
    outer=0.75in,
    % margin=1in,
]{geometry}

%%%% Page Header and Foot %%%%
\usepackage{fancyhdr}
\fancypagestyle{plain}{
\fancyhf{}
\fancyhead[L]{266:711:685}
\fancyhead[C]{\textbf{Convex Analysis and Optimization}}
\fancyhead[R]{Rutgers University}
\fancyfoot[L]{}
\fancyfoot[C]{\thepage}
\fancyfoot[R]{}
}
\pagestyle{plain}

%%%% Load Format %%%%
%%%%%%%%%%%%%%%%%%%%%%%%%%%%%%%%%%%%%%%%%%%%%%%%%%%%%%%%%%%%%%%%%%%%%%%%%%%%%%%%%%%% Packages to Load
%% Font Packages
\RequirePackage[english]{babel} %Multilingual support
\RequirePackage[utf8]{inputenc} %utf8 support.
\RequirePackage[T1]{fontenc} %Better Font Encoding for printing
\RequirePackage{libertinus} %Popular Math Font
\RequirePackage{microtype} %Better Font Spacing.
\RequirePackage{lmodern} %Popular Font

%% Math Packages
\RequirePackage{amsfonts,amsmath,amssymb} %Math Fonts and Symbols
\RequirePackage{mathtools} %Define math symbols
\RequirePackage{mleftright}  %Fixes some annoying spacing issues
\RequirePackage{bm} %Boldface math, load after other math font packages

%% Graphics Packages
\RequirePackage{graphicx} %Graphics backend with key-value arguments
\RequirePackage[export]{adjustbox} %More arguments for includegraphics
\RequirePackage[svgnames]{xcolor} %Beamer has this by default
\RequirePackage[most]{tcolorbox} %Customized Box
\RequirePackage{transparent} %Transparent Images
\RequirePackage{subcaption} %Package to use subfigure

%% Table Packages
\RequirePackage{tabularx} %Including tabular with more space option
\RequirePackage{multirow} %Package to use multirow in tables
\RequirePackage{colortbl} %Package to use color in tables
\RequirePackage{array} %Math and more spacing support in tables

%% Theorems etc (also  problems)
\RequirePackage[shortlabels]{enumitem} %Better List
\RequirePackage{hyperref} %Beamer has this by default
\RequirePackage{cleveref} %Must load before amsthm. Conflict With Beamer
\RequirePackage{amsthm, thmtools, thm-restate} %Better Theorems

%% Code Packages
\RequirePackage{listings} %Package to insert code. Better than minted
\RequirePackage{algorithm} %Package to insert algorithm
\RequirePackage{algpseudocode} %Package to insert algorithm
% \RequirePackage{keyval}

%% Display Packages
% \RequirePackage[document]{ragged2e} %Replaced by microtype
\RequirePackage{csquotes} %Package to facilities quotations
\RequirePackage{multicol} %Package to use multicols

%% Other Packages
\RequirePackage{xspace} %Package to add space after macros
\RequirePackage{calc} %Package to calculate in Latex
\RequirePackage{rotating} %Package to rotate objects
% \RequirePackage{pgf, tikz} %Package to draw figures.
\RequirePackage{epstopdf} %Package to convert eps to pdf
\RequirePackage[backend=biber, style=numeric, sorting=none]{biblatex}
%%%% delimiters
\DeclarePairedDelimiter\parens{\lparen}{\rparen}  
\DeclarePairedDelimiter\bracks{\lbrack}{\rbrack}
\DeclarePairedDelimiter\braces{\lbrace}{\rbrace}
\DeclarePairedDelimiter\abs{\lvert}{\rvert}
\DeclarePairedDelimiter\norm{\lVert}{\rVert}
\DeclarePairedDelimiter\angles{\langle}{\rangle}
\DeclarePairedDelimiter\ceil{\lceil}{\rceil}
\DeclarePairedDelimiter\floor{\lfloor}{\rfloor}

%%%% math operators naming
\DeclareMathOperator*{\argmax}{\mathrm{argmax}}
\DeclareMathOperator*{\argmin}{\mathrm{argmin}}
\DeclareMathOperator{\tr}{\mathrm{tr}}
\DeclareMathOperator{\eig}{\mathrm{eig}}
\DeclareMathOperator{\sgn}{\mathrm{sgn}}
% \let\det\relax % "Undefine" \det
% \DeclareMathOperator{\det}{\mathrm{det}} % already defined in mathtools
\DeclareMathOperator{\diag}{\mathrm{diag}}
\DeclareMathOperator{\rank}{\mathrm{rank}}
\DeclareMathOperator{\Vol}{\mathrm{Vol}}   % volume
\DeclareMathOperator{\Surf}{\mathrm{Surf}} % surface area

%%%% Transforms! -- requires mathtools package
\newcommand*{\LapTrans}{\xleftrightarrow{\mathcal{Z}}}
\newcommand*{\ZTrans}{\xleftrightarrow{\mathcal{L}}}
\newcommand*{\CTFS}{\xleftrightarrow{\mathrm{CTFS}}}
\newcommand*{\CTFT}{\xleftrightarrow{\mathrm{CTFT}}}
\newcommand*{\DTFS}{\xleftrightarrow{\mathrm{DTFS}}}
\newcommand*{\DTFT}{\xleftrightarrow{\mathrm{DTFT}}}

%%%% vector font
\let\oldvec\vec
\renewcommand*{\vec}[1]{\mathbf{#1}}
\newcommand*{\trn}{{}^{\mkern-4mu\intercal}}
\newcommand*{\inv}{^{-1}} 

%%%% number systems
\DeclareMathOperator{\R}{\mathbb{R}}
\DeclareMathOperator{\C}{\mathbb{C}}
\DeclareMathOperator{\N}{\mathbb{N}}
\DeclareMathOperator{\Z}{\mathbb{Z}}
\DeclareMathOperator{\F}{\mathbb{F}}
\DeclareMathOperator{\Q}{\mathbb{Q}}

%%%% STATISTICS AND PROBABILITY
\newcommand*{\Var}{\mathop{\mathrm{Var}}}
\newcommand*{\Cov}{\mathop{\mathrm{Cov}}}
\newcommand*{\Corr}{\mathop{\mathrm{Corr}}}
\newcommand*{\MSE}{\mathop{\mathrm{MSE}}}

\newcommand*{\E}[1]{\mathbb{E}\bracks*{#1}}
\newcommand*{\condE}[2]{\mathbb{E}\bracks*{#1 \mid #2}}
\renewcommand*{\P}[1]{\mathbb{P}\parens*{#1}}
\newcommand*{\condP}[2]{\mathbb{P}\parens*{#1 \mid #2}}

\DeclareMathOperator{\Bern}{\mathsf{Bern}}
\DeclareMathOperator{\Unif}{\mathsf{Unif}}
\DeclareMathOperator{\Expv}{\mathsf{Exp}}
\DeclareMathOperator{\Poi}{\mathsf{Poi}}
\DeclareMathOperator{\Gamv}{\mathsf{Gamma}}
\DeclareMathOperator{\Dirv}{\mathsf{Dir}}
\DeclareMathOperator{\Mult}{\mathsf{Mult}}
\DeclareMathOperator{\Beta}{\mathsf{Beta}}
\DeclareMathOperator{\Geomv}{\mathsf{Geom}}
\DeclareMathOperator{\Binomv}{\mathsf{Binom}}
\DeclareMathOperator{\NegBinomv}{\mathsf{NB}}
\DeclareMathOperator{\Lap}{\mathsf{Lap}}
\DeclareMathOperator{\Gaus}{\mathsf{N}}
\DeclareMathOperator{\Weibull}{\mathsf{Weibull}}


\DeclareMathOperator{\iidsim}{\stackrel{\mathrm{i.i.d.}}{\sim}}
\DeclareMathOperator{\diff}{\mathop{}\!\mathrm{d}}

%%%% Special norms and linear algebra stuff
\newcommand*{\subgnorm}[1]{\norm*{#1}_{\psi_2}}
\newcommand*{\subexpnorm}[1]{\norm*{#1}_{\psi_1}}
\newcommand*{\frobnorm}[1]{\norm*{#1}_{\mathrm{F}}}
\newcommand*{\opnorm}[1]{\norm*{#1}_{\mathrm{op}}}
\newcommand*{\Lipnorm}[1]{\norm*{#1}_{\mathrm{Lip}}}

%%%%
\newcommand*{\set}[1]{\braces*{\,#1\,}}
\newcommand*{\ie}{\textnormal{i.e.\ }}
\newcommand*{\eg}{\textnormal{e.g.\ }}
\newcommand*{\etc}{\textnormal{etc.\ }}
\allowdisplaybreaks

%%%% Theorem Style %%%%
\declaretheorem[numbered=no, style=plain]{axiom, lemma}
\declaretheorem[numberwithin=section,style=definition]{definition}
\declaretheorem[sibling=definition]{theorem, corollary, proposition, conjecture}
\declaretheorem[numbered=no,style=remark]{remark, claim}

%%%% Document Information %%%%
\title{Homework 2}
\author{Kailong Wang}
\date{\today}

%%%% Start Document %%%%
\begin{document}
\maketitle

\begin{problem}
    {Q1}
    Suppose that $f:\R^n\rightarrow(-\infty, +\infty]$ is a convex function and $x \in \dom f$. Show that for any $d\in\R^n$ the function $g_d:(0,\infty)\rightarrow(-\infty,+\infty]$ defined by
    \[
        g_d(\alpha) = \frac{f(x+\alpha d)-f(x)}{\alpha}
    \]
    is non-decreasing.
\end{problem}

\begin{solution}
    {Solution}
    Since $f$ is convex, we have
    \begin{align}
        f(\lambda x + (1-\lambda)y) \leq \lambda f(x) + (1-\lambda)f(y). \quad \forall x, y \in \R^n, \lambda \in [0, 1] \label{eq:convex}
    \end{align}
    To show that $g_d(\alpha)$ is non-decreasing, we need to show
    \[\alpha_1 \leq \alpha_2 \Rightarrow g_d(\alpha_1) \leq g_d(\alpha_2). \quad \forall \alpha_1, \alpha_2 \in (0, \infty)\]
    Let $\lambda=\frac{\alpha_1}{\alpha_2}\in[0,1]$ (because $\alpha_1 \leq \alpha_2$), then
    \begin{align*}
        f(x+\alpha_1d) = f(\frac{\alpha_1}{\alpha_2}(x+\alpha_2d) + (1-\frac{\alpha_1}{\alpha_2})x)
        &\leq \frac{\alpha_1}{\alpha_2}f(x+\alpha_2d) + (1-\frac{\alpha_1}{\alpha_2})f(x) &&\text{by \cref{eq:convex}}\\
        \Rightarrow \alpha_2f(x+\alpha_1d)
        &\leq \alpha_1f(x+\alpha_2d) + (\alpha_2-\alpha_1)f(x) &&\text{multiply $\alpha_2$ on both sides}\\
        \Rightarrow \alpha_2f(x+\alpha_1d) - \alpha_2f(x)
        &\leq \alpha_1f(x+\alpha_2d) - \alpha_1f(x) &&\text{subtract $\alpha_2f(x)$ on both sides}\\
        \Rightarrow \frac{f(x+\alpha_1d)-f(x)}{\alpha_1}
        &\leq \frac{f(x+\alpha_2d)-f(x)}{\alpha_2} &&\text{some algebra}\\
        \Rightarrow g_d(\alpha_1) &\leq g_d(\alpha_2).
    \end{align*}
    Therefore, $g_d(\alpha)$ is non-decreasing.
\end{solution}

\begin{problem}
    {Q2: Non-convex Projections (similar to exercise $2.11$ in the text).}
    Let $C\subset\R^n$ be a non-empty closed set (but possibly not convex), and consider any point $x\in\R^n$.
    \begin{enumerate}[(a)]
        \item Show that the function $g(w)\dot{=}\norm{w-x}$ must have a nonempty, compact set of minima over $C$. Denote this set by $P_C(x)$.
        \item Show that $\dist_C(x)\dot{=}\inf_{w\in C}\norm{w-x}$ is an everywhere finite-valued and continuous function of $x\in\R^n$. (If you like, you can show that it is Lipschitz continuous with modulus $1$, which implies continuity.)
        \item Give an example showing that if $C$ is not convex, $\dist_C$ need not be convex.
    \end{enumerate}
\end{problem}

\begin{solution}
    {Solution}
    \begin{enumerate}[(a)]
        \item {
            \begin{proof}
            To show that the function $g(w)=\norm{w-x}$ must have a nonempty, compact set of minima over the closed set $C$, we can use the fact that $C$ is nonempty and closed.
            \begin{enumerate}[(i)]
                \item If $x\in C$, then {
                    \begin{align*}
                        \min_{w\in C} g(w)
                        & =\norm{w-x} \\
                        &=0. \qquad \forall w=x\in C
                    \end{align*}
                }
                Therefore, $P_C(x)=\set{x}$, which is nonempty and compact.
                \item If $x\notin C$, then $g(w)=\norm{w-x}>0$ for all $w\in C$. We can then prove by contradiction. Assume that $g(w)=\norm{w-x}$ does not have any minimum points within $C$. This means that for any point $w\in C$, there exists a sequence of points $\set{w_n}$ such that
                \[g(w_n)\leq g(w)\]
                for all $n$ (\ie $w_n$ gets arbitrary close to $x$). Since $C$ is closed, the limit of this sequence, denoted as \[w^*=\lim_{n\rightarrow\infty}w_n,\] must also be in $C$ because the limit of a sequence in a closed set belongs to that set. Moreover, since $g(w)$ is continuous, we have \[\lim_{n\rightarrow\infty}g(w_n) = g(w^*).\]
                But this would imply that $g(w^*)=0$ (because $g(w_n)$ gets arbitrary close to $0$), which means $w^*=x$. However, since $x\notin C$, we have a contradiction. Therefore, $g(w)$ must have a nonempty, compact set of minima over $C$ (at least one minimum point).
            \end{enumerate}
        \end{proof}
        }
        \item {
            \begin{proof}
                We show $\dist_C(x)$ is everywhere finiteness and continuous as follows:
                \begin{enumerate}[(i)]
                    \item \textbf{Finiteness:} For any $x\in \R^n$, we have $\norm{w-x}\geq 0$ for all $w\in C$ since the norm is always non-negative. Given that $C$ is nonempty and closed, there exists some $w'\in C$, for any $x\in \R^n$, such that $\norm{w'-x}\geq 0$. And because $g(w)$ is nonempty and compact, the $\norm{w'-x}$ is finite. Since the infimum of a set of finite non-negative value is also finite non-negative, $\dist_C(x)$ must also be finite non-negative.
                    \item \textbf{Continuity:} Given two points $x, y\in \R^n$, let $w^*$ be the point in $C$ that achieves the infimum for $x$ (\ie $\norm{w^*-x}=\dist_C(x)$). Then
                    \begin{align*}
                        \dist_C(y)
                        &\leq \norm{w^* - y} \\
                        &= \norm{(w^* - x) + (x - y)} \\
                        &\leq \norm{w^* - x} + \norm{x - y} && \text{(by triangle inequality)} \\
                        &= \dist_C(x) + \norm{x - y}
                    \end{align*}
                    By symmetry, we can also show that $\dist_C(x)\leq\dist_C(y)+\norm{x-y}$. Therefore, we have $|\dist_C(x)-\dist_C(y)|\leq \norm{x-y}$. This means that $\dist_C(x)$ is Lipschitz continuous with modulus $1$, which implies continuity.
                \end{enumerate}
            \end{proof}
        }
        \item Consider two disjoint closed balls in $\R^2$, {
            \begin{align*}
                B_1 &= \set{w \mid \norm{w-(0, 0)}\leq 1} \\
                B_2 &= \set{w \mid \norm{w-(4, 0)}\leq 1}.
            \end{align*}
        } Let $C=B_1\cup B_2$. $C$ is not convex since the line segment between any point in $B_1$ and any point in $B_2$ is not entirely contained in $C$. Consider three points: $x_1=(0,0)$, $x_2=(4,0)$, and $x_{mid}=(2,0)$. Clearly, $\dist_C(x_1)=\dist_C(x_2)=0$, and $\dist_C(x_{mid})=1$. Since
        \[
            \dist_C(x_{mid}) = \dist_C(\frac{1}{2}x_1+\frac{1}{2}x_2) =1 \geq 0 =\frac{1}{2}\dist_C(x_1) + \frac{1}{2}\dist_C(x_2),
        \]
        $\dist_C(x)$ is not convex.
    \end{enumerate}
\end{solution}

\begin{problem}
    {Q3}
    Given a set $X\subseteq\R^n$, its \textit{indicator function} is the function $\delta_X:\R^n\rightarrow(-\infty, +\infty]$ given by
    \begin{align*}
        \delta_X(x) = \begin{cases}
            0 & \text{if } x\in X\\
            +\infty & \text{if } x\notin X
        \end{cases}
    \end{align*}
    \begin{enumerate}[(a)]
        \item Show that if $X$ is a closed set, $\delta_X$ is a closed function.
        \item Show that if $X$ is a convex set, $\delta_X$ is a convex function.
    \end{enumerate}
\end{problem}

\begin{solution}
    {Solution}
    \begin{enumerate}[(a)]
        \item {
            \begin{proof}
                To show that $\delta_X$ is a closed function, we need to show that the epigraph of $\delta_X$ is a closed set. The epigraph of $\delta_X$ is defined as
                \[
                    \epi(\delta_X) = \set{(x, \alpha) \mid \alpha \geq \delta_X(x)}.
                \]
                Since $X$ is a closed set, we have $\delta_X(x)=0$ for all $x\in X$ and $\delta_X(x)=+\infty$ for all $x\notin X$. Therefore, the epigraph of $\delta_X$ can be written as
                \[
                    \epi(\delta_X) = \set{(x, \alpha) \mid \alpha \geq 0, x\in X} \cup \set{(x, \alpha) \mid \alpha \geq +\infty, x\notin X}.
                \]
                The first set is the product of a closed set and a closed interval, which is closed. The second set is an empty set $\emptyset$. Therefore, $\epi(\delta_X)$ is a union of a closed set and an empty set, which is closed. This means that $\delta_X$ is a closed function.
            \end{proof}
        }
        \item {
            \begin{proof}
                To show that $\delta_X$ is a convex function, we need to show that \[\delta_X(\lambda x_1+(1-\lambda)x_2)\leq\lambda\delta_X(x_1)+(1-\lambda)\delta_X(x_2) \quad \forall x_1, x_2\in \R^n, \lambda\in[0,1]\]
                \begin{enumerate}[(i)]
                    \item If $x_1, x_2\in X$, then $\delta_X(x_1)=\delta_X(x_2)=0$. Therefore, with $X$ is a convex set, we have \[\delta_X(\lambda x_1+(1-\lambda)x_2)=0\leq 0=\lambda\delta_X(x_1)+(1-\lambda)\delta_X(x_2).\]
                    \item If either $x_1$ or $x_2$ (or both) is not in $X$, then the right side of the inequality becomes infinite. Therefore, the inequality holds trivially.
                \end{enumerate}
                This concludes that $\delta_X$ is a convex function.
            \end{proof}
        }
    \end{enumerate}
\end{solution}

\begin{problem}
    {Q4}
    Suppose $K\subset\R^n$ is a nonempty closed convex cone and $y\notin K$. Using the separating hyperplane theorem, show that there exists a vector $a\in \R^n$ such that $\angles*{a,x}\leq 0$ for all $x\in K$ and $\angles*{a,y}>0$ (this is equivalent to showing that there is a hyperplane separating $y$ from $K$ that passes through the origin).
\end{problem}

\begin{solution}
    {Solution}
    Since $K$ is a nonempty closed convex cone and $y\notin K$, we have $K\cap\set{y}=\emptyset$. Therefore, by the separating hyperplane theorem, there exists a vector $a\in \R^n$ such that $\angles*{a,x}\leq 0$ for all $x\in K$ and $\angles*{a,y}>0$. This is equivalent to showing that there is a hyperplane separating $y$ from $K$ that passes through the origin. This concludes the proof.
\end{solution}

\end{document}