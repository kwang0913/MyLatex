\documentclass{article}
\usepackage[
    letterpaper,
    top=1in,
    bottom=1in,
    inner=0.75in,
    outer=0.75in,
    % margin=1in,
]{geometry}

%%%% Page Header and Foot %%%%
\usepackage{fancyhdr}
\fancypagestyle{plain}{
\fancyhf{}
\fancyhead[L]{266:711:685}
\fancyhead[C]{\textbf{Convex Analysis and Optimization}}
\fancyhead[R]{Rutgers University}
\fancyfoot[L]{}
\fancyfoot[C]{\thepage}
\fancyfoot[R]{}
}
\pagestyle{plain}

%%%% Load Format %%%%
%%%% Page Margin %%%%
\RequirePackage{geometry}

%%%% Page Header and Foot %%%%
\RequirePackage{fancyhdr}
\pagestyle{fancyplain}
\lhead{\lefthead}
\chead{\centerhead}
\rhead{\righthead}
\lfoot{\leftbottom}
\cfoot{\centerbottom}
\rfoot{\rightbottom}

%%%% Additional Packages to Load %%%%
%%%% Main Font
\RequirePackage{lmodern} %Computer Modern, but with many more glyphs
\RequirePackage{microtype} %Better Font Rendering
\RequirePackage{libertinus} %Very popular math font
\RequirePackage[T1]{fontenc} %Better Printing Font
\RequirePackage[utf8]{inputenc} %enable utf8 encoding
\RequirePackage[english]{babel} %English Language

%%%% Math Env
\RequirePackage{amsfonts,amsmath,amssymb} %Math Font Packages
\RequirePackage{mathtools} %Better Math Environments
\RequirePackage{mleftright}  %Fixes some annoying spacing issues
\RequirePackage{bm} %Boldface math, load after other math font packages

%%%% Beautiful Stuff
\RequirePackage{graphicx} %Better Graphics
\RequirePackage[export]{adjustbox} %Better Graphics
\RequirePackage[dvipsnames,svgnames]{xcolor} %Must load before tcolorbox
\RequirePackage[breakable]{tcolorbox} %Customized Box
\tcbuselibrary{skins}
\RequirePackage{transparent} 

%%%% Table and List 
\RequirePackage{tabularx} %Better Table
\RequirePackage{multirow}
\RequirePackage{colortbl}
\RequirePackage{array}

%%%% Theorems etc (also  problems)
\RequirePackage[shortlabels]{enumitem} %Better List
\RequirePackage{hyperref, cleveref} %Fantastic crossrefer
\RequirePackage{amsthm, thmtools} %Better Theorems
\RequirePackage{thm-restate} %Better Theorems

%%%% Code Packages
\RequirePackage{algorithm2e} % Package to create pseudo-code
\RequirePackage{listings} % Package to insert code
% \RequirePackage{keyval}

%%%% Display Packages
\RequirePackage[document]{ragged2e} % Package to justify text
\RequirePackage{csquotes} % Package to facilities quotations
\RequirePackage{multicol} % Package to use multicols

%%%% Other Packages
\RequirePackage{calc}
\RequirePackage{rotating}
\RequirePackage{pgf, tikz}
\RequirePackage{epstopdf}
\RequirePackage[backend=biber, style=numeric, sorting=none]{biblatex}

%%%% A box fill the rest of the page
\newtcolorbox{stretchbox}[1][]{
    height fill,
    sharp corners,
    colback=white,
    % colframe=yellow!50!black,
    #1
    }

%%%% Box Environment
\newtcolorbox{prob}[1]{
% Set box style
    sidebyside,
    sidebyside align=top,
% Dimensions and layout
    width=\textwidth,
    toptitle=2.5pt,
    bottomtitle=2.5pt,
    righthand width=0.20\textwidth,
% Coloring
    colbacktitle=white,
    coltitle=black,
    colback=white,
    colframe=black,
% Title formatting
    title={
        #1 \hfill Grade:\hspace*{0.15\textwidth}
    },
    fonttitle=\large\bfseries
}

%%%% Problem Environment 
\newenvironment{problem}[1]{
    \begin{prob}{#1}
}
{
    \tcblower
    \centering
    \textit{\scriptsize\bfseries Faculty Comments}
    \end{prob}
}

%%%% Draw a line
\newcommand{\myrule}{\rule{1.5in}{0.1mm}}
%%%% delimiters
\DeclarePairedDelimiter\parens{\lparen}{\rparen}
\DeclarePairedDelimiter\bracks{\lbrack}{\rbrack}
\DeclarePairedDelimiter\braces{\lbrace}{\rbrace}
\DeclarePairedDelimiter\abs{\lvert}{\rvert}
\DeclarePairedDelimiter\norm{\lVert}{\rVert}
\DeclarePairedDelimiter\angles{\langle}{\rangle}
\DeclarePairedDelimiter\ceil{\lceil}{\rceil}
\DeclarePairedDelimiter\floor{\lfloor}{\rfloor}

%%%% math operators naming
\DeclareMathOperator*{\argmax}{\textnormal{argmax}}
\DeclareMathOperator*{\argmin}{\textnormal{argmin}}
\DeclareMathOperator{\tr}{\textnormal{tr}}
\DeclareMathOperator{\eig}{\textnormal{eig}}
\DeclareMathOperator{\sgn}{\textnormal{sgn}}
% \let\det\relax % "Undefine" \det
% \DeclareMathOperator{\det}{\textnormal{det}} % already defined in mathtools
\DeclareMathOperator{\diag}{\textnormal{diag}}
\DeclareMathOperator{\rank}{\textnormal{rank}}
\DeclareMathOperator{\Vol}{\textnormal{Vol}}   % volume
\DeclareMathOperator{\Surf}{\textnormal{Surf}} % surface area

%%%% Transforms! -- requires mathtools package
\newcommand*{\LapTrans}{\xleftrightarrow{\mathcal{Z}}}
\newcommand*{\ZTrans}{\xleftrightarrow{\mathcal{L}}}
\newcommand*{\CTFS}{\xleftrightarrow{\textnormal{CTFS}}}
\newcommand*{\CTFT}{\xleftrightarrow{\textnormal{CTFT}}}
\newcommand*{\DTFS}{\xleftrightarrow{\textnormal{DTFS}}}
\newcommand*{\DTFT}{\xleftrightarrow{\textnormal{DTFT}}}

%%%% vector font
\let\oldvec\vec
\renewcommand*{\vec}[1]{\mathbf{#1}}
% \newcommand*{\trn}{\!^{\!\intercal}}
\newcommand*{\trn}{\!^{\mathsf{T}}}
% \newcommand*{\coj}{\!^{\dag}} % Text Mode Symbol, Should not be used
% \newcommand*{\coj}{\!^{\dagger}}
\newcommand*{\coj}{\!^{\mathsf{H}}}
\newcommand*{\inv}{^{-1}}

%%%% number systems
\DeclareMathOperator{\R}{\mathbb{R}}
\DeclareMathOperator{\C}{\mathbb{C}}
\DeclareMathOperator{\N}{\mathbb{N}}
\DeclareMathOperator{\Z}{\mathbb{Z}}
\DeclareMathOperator{\F}{\mathbb{F}}
\DeclareMathOperator{\Q}{\mathbb{Q}}

%%%% STATISTICS AND PROBABILITY
\newcommand*{\Var}{\mathop{\textnormal{Var}}}
\newcommand*{\Cov}{\mathop{\textnormal{Cov}}}
\newcommand*{\Corr}{\mathop{\textnormal{Corr}}}
\newcommand*{\MSE}{\mathop{\textnormal{MSE}}}
\newcommand*{\MSD}{\mathop{\textnormal{MSD}}}
\newcommand*{\NSD}{\mathop{\textnormal{NSD}}}

\newcommand*{\E}[1]{\mathbb{E}\bracks*{#1}}
\newcommand*{\condE}[2]{\mathbb{E}\bracks*{#1 \mid #2}}
\renewcommand*{\P}[1]{\mathbb{P}\parens*{#1}}
\newcommand*{\condP}[2]{\mathbb{P}\parens*{#1 \mid #2}}

\DeclareMathOperator{\Bern}{\mathsf{Bern}}
\DeclareMathOperator{\Unif}{\mathsf{Unif}}
\DeclareMathOperator{\Expv}{\mathsf{Exp}}
\DeclareMathOperator{\Poi}{\mathsf{Poi}}
\DeclareMathOperator{\Gamv}{\mathsf{Gamma}}
\DeclareMathOperator{\Dirv}{\mathsf{Dir}}
\DeclareMathOperator{\Mult}{\mathsf{Mult}}
\DeclareMathOperator{\Beta}{\mathsf{Beta}}
\DeclareMathOperator{\Geomv}{\mathsf{Geom}}
\DeclareMathOperator{\Binomv}{\mathsf{Binom}}
\DeclareMathOperator{\NegBinomv}{\mathsf{NB}}
\DeclareMathOperator{\Lap}{\mathsf{Lap}}
\DeclareMathOperator{\Gaus}{\mathsf{N}}
\DeclareMathOperator{\Weibull}{\mathsf{Weibull}}


\DeclareMathOperator{\iidsim}{\stackrel{\textnormal{i.i.d.}}{\sim}}
\DeclareMathOperator{\diff}{\mathop{}\!\textnormal{d}}

%%%% Special norms and linear algebra stuff
\newcommand*{\subgnorm}[1]{\norm*{#1}_{\psi_2}}
\newcommand*{\subexpnorm}[1]{\norm*{#1}_{\psi_1}}
\newcommand*{\frobnorm}[1]{\norm*{#1}_{\textnormal{F}}}
\newcommand*{\opnorm}[1]{\norm*{#1}_{\textnormal{op}}}
\newcommand*{\Lipnorm}[1]{\norm*{#1}_{\textnormal{Lip}}}

%%%%
\newcommand*{\set}[1]{\braces*{\,#1\,}}
\newcommand*{\ie}{\textnormal{i.e.\ }}
\newcommand*{\eg}{\textnormal{e.g.\ }}
\newcommand*{\etc}{\textnormal{etc.\ }}
\newcommand*{\iid}{\textnormal{i.i.d.\ }}
\allowdisplaybreaks

%%%% Theorem Style %%%%
\declaretheorem[numbered=no, style=plain]{axiom, lemma}
\declaretheorem[numberwithin=section,style=definition]{definition}
\declaretheorem[sibling=definition]{theorem, corollary, proposition, conjecture}
\declaretheorem[numbered=no,style=remark]{remark, claim}

%%%% Document Information %%%%
\title{Midterm 1}
\author{Kailong Wang}
\date{\today}

%%%% Start Document %%%%
\begin{document}
\maketitle

\begin{problem}
    {Q1: Normal cones to level sets.}
    Suppose $h:\R^n\rightarrow\R$ is a continuously differentiable convex function, and consider the level set $L(0,h)=\set{x\in\R^n\mid h(x)<0}$. Assuming that there exists some point $\bar{x}\in\R^n$ with $h(\bar{x})<0$, prove that, for any $x\in L(0,h)$, the normal cone $N_{L(0,h)}(x)$ to $L(0,h)$ at $x$ is given by the formula
    \begin{align*}
        N_{L(0,h)}(x)=\set{\alpha\nabla h(x)\mid \alpha\geq 0, \alpha h(x)=0}
        \begin{cases}
            \emptyset & \text{if } h(x)>0 \\
            \set{\alpha\nabla h(x)\mid \alpha\geq 0} & \text{if } h(x)=0\\
            \set{\bm{0}} & \text{if } h(x)<0
        \end{cases}
    \end{align*}
\end{problem}

\begin{solution}
    {Solution}
    \begin{proof}
        % By definition of Normal Cone, we have \[N_{L(0,h)}(x)=\set{v\mid \ip{v, y-x}\leq 0, \forall y\in L(0,h)}.\]
        We prove the statement by cases.
        \begin{enumerate}
            \item If $h(x)>0$, then $x\notin L(0,h)$, so $N_{L(0,h)}(x)=\emptyset$, which is trivial.
            \item If $h(x)=0$, $x$ lies on the boundary of $L(0, h)$. The function $h$ being continuously differentiable and convex implies that at $x$, the gradient $\nabla h(x)$ points in a direction that is normal to the level set $L(0, h)$ (since the normal cone is the polar cone of tangent cone of level set at $x$). By convexity, we have \[h(y)\geq h(x)+\ip{\nabla h(x), y-x}, \qquad \forall y\in L(0, h).\] With $h(y)<0$ (definition of level set) and $h(x)=0$ (case assumption), we have \[\ip{\nabla h(x), y-x}<0.\] This means the vector $\nabla h(x)$ is an outward normal to the level set at $x$. Since $h$ does not increase in the direction inside the level set, the normal cone at $x$ consists of all non-negative scalar multiples of $\nabla h(x)$, \ie \[N_{L(0,h)}(x)=\set{\alpha\nabla h(x)\mid \alpha\geq 0}.\]
            \item If $h(x)<0$, the $x\in\ri L(0, h)$. The normal cone at a point in the relative interior of a convex set is just the zero vector. Because there are directions in every neighborhood around $x$ that stay within $L(0, h)$, and therefore no ``outside'' direction is associated with a decrease from $x$ within level set. Thus, \[N_{L(0,h)}(x)=\set{\bm{0}}.\]
        \end{enumerate}
    \end{proof}
    Question: Is the necessity of the assumption $h(\bar{x})<0$ for some $\bar{x}$ to guarantee that the level set $L(0,h)$ is nonempty?
\end{solution}

\begin{problem}
    {Q2}
    {\bf (Optimality conditions for convex problems with ``mixed'' constraint sets.)}
    Consider an optimization problem of the form
    \begin{equation}
        \begin{aligned}
            \min & \quad f(x) \\
            \st & \quad Ax=b \\
            & \quad h_j(x)\leq 0 \qquad j=1,2,\dots,r \\
            & \quad x\in X
        \end{aligned}
        \label{eq:optimization}
    \end{equation}
    where
    \begin{itemize}
        \item $f:\R^n\rightarrow\R\cup\set{+\infty}$ is a convex function
        \item $A$ is an $m\times n$ matrix and $b\in\R^m$
        \item For $j=1,2,\dots,r$, $h_j:\R^n\rightarrow\R$ is a differentiable convex function
        \item $X$ is a convex set.
    \end{itemize}
    Let $a_i$ denote row $i$ of $A$, $i=1,2,\dots,m$ represented as a column vector. Suppose that there exists a point $\bar{x}\in\R^n$ with the following properties:
    \begin{itemize}
        \item $\bar{x}\in\ri\dom f$
        \item $A\bar{x}=b$
        \item For $j=1,2,\dots,r$, $h_j(\bar{x})<0$
        \item $\bar{x}\in\ri X$.
    \end{itemize}
    Show that for $x^*\in\R^n$ to be a solution of~\cref{eq:optimization}, it is necessary and sufficient that there exist $\lambda^*\in\R^m$ and $\mu^*\in\R^r$ such that
    \begin{align*}
        \partial f(x^*)+\sum_{i=1}^{m}\lambda^*_i a_i+\sum_{j=1}^{r}\mu^*_j\nabla h_j(x^*)+N_X(x^*) &\ni 0\\
        \sum_{j=1}^{r}\mu^*_j h_j(x^*) &= 0 \\
        Ax^* &= b \\
        h_j(x^*) &\leq 0 \qquad j=1,2,\dots,r \\
        \mu^*_j &\geq 0 \qquad j=1,2,\dots,r.
    \end{align*}
\end{problem}

\begin{solution}
    {Solution}
    \begin{proof}
        Solving~\cref{eq:optimization} is equivalent to solving the following cvx problem:
        \begin{enumerate}
            \item $f_1(x)=f(x)$
            \item $f_2(x)=\delta_L(x)=\begin{cases}
                0, & Ax = b \\
                +\infty, & \text{otherwise}
            \end{cases}$
            \item $f_3(x)=\delta_X(x)=\begin{cases}
                0, & x\in X \\
                +\infty, & \text{otherwise}
            \end{cases}$
            \item $f_4(x)=\delta_{h_j}(x)=\begin{cases}
                0, & h_j(x)\leq 0\\
                +\infty, & \text{otherwise}
            \end{cases}$, j=1,2,\dots,r 
        \end{enumerate}
        Let $x^*$ be an optimal solution to the optimization problem. For $h_j(x^*)=0$, we can use the result from the previous question and have normal cone $N_{L(0,h_j)}(x^*)=\set{\mu_j^*\nabla h_j(x^*)\mid \mu_j^*\geq 0}$. For $h_j(x^*)<0$, we will have $\mu_j^*=\bm{0}$ since we can't have a positive multiplier for a strictly feasible constraint. These contribute to the condition of $f_4(x)$ \[\sum_{j=1}^{r}\mu^*_j h_j(x^*) = 0; \qquad h_j(x^*) \leq 0 \quad j=1,2,\dots,r; \qquad \mu^*_j \geq 0 \quad j=1,2,\dots,r.\]

        % The condition of $Ax^*=b$ is trivial since if the equality does not hold, the problem is infeasible.
        With the proposition that we have proved in the class, we know that solving $\partial(f_1+f_2+f_3)(x)\ni\partial f_1(x)+\partial f_2(x) + \partial f_3(x)$ with condition $Ax=b$ implies \[\exists \lambda^*\in\R^m, x^*\in\R^n \qquad \partial f(x^*) + A\trn\lambda^* + N_X(x^*) \ni 0.\]

        Combine the above results (since $\bar{x}\in\ri\dom f_1\cap\ri\dom f_2\cap\ri\dom f_3\cap\ri\dom f_4\neq\emptyset$), we have (using Rockafellar-Moreau theorem) \[\partial f(x^*)+\sum_{i=1}^{m}\lambda^*_i a_i+\sum_{j=1}^{r}\mu^*_j\nabla h_j(x^*)+N_X(x^*) \ni 0,\] which completes the proof.
    \end{proof}
\end{solution}

\begin{problem}
    {Q3: Optimality conditions for convex cone programming.}
    Below, suppose $K\subseteq\R^m$ be a nonempty closed convex cone,
    \begin{enumerate}[(a)]
        \item Show that for any $x\in K$, \[F_K(x)=\set{z-\alpha x\mid z\in K, \alpha\geq 0}.\]
        \item Show that for any $x\in K$, \[N_K(x)=\set{y\in K^*\mid \ip*{x,y}=0}.\]
        Hint: you may use the results of homework 3, problem 1(c) and 5(c).
        \item  $A$ is an $m\times n$ matrix, and $b\in\R^m$, and let $Z=\set{x\in\R^n\mid Ax-b\in K}$. Assume that $Z$ is nonempty. Show that, for $x\in Z$, \[N_Z(x)=\cl\set{A\trn\lambda\mid\lambda\in K^*,\ip{Ax-b,\lambda}=0}.\]
        Hint: you may use the results of homework 3, problem 5.
        \item Show that $\ri Z \supseteq\set{x\in\R^n\mid Ax-b\in\ri K}.$
        \item Let $f:\R^n\rightarrow\R\cup\set{+\infty}$ be a convex function, suppose that the cone $A\trn K^*=\set{A\trn\lambda\mid\lambda\in K^*}$ is closed, and consider the problem{
            \begin{equation}
                \begin{aligned}
                    \min & \quad f(x) \\
                    \st & \quad Ax-b\in K.
                \end{aligned}
                \label{eq:cone}
            \end{equation}
        }
        Further suppose that there exists some point $\bar{x}\in\ri\dom f$ such that $A\bar{x}-b\in\ri K$. Show that, in order for $x^*\in\R^n$ to solve~\cref{eq:cone}, it is necessary and sufficient that there exists $\lambda^*\in \R^m$ such that
        \begin{align*}
            \partial f(x^*)+A\trn\lambda^* &\ni 0 \\
            \lambda^* & \in K^* \\
            \ip{Ax^*-b,\lambda^*} & =0.
        \end{align*}
    \end{enumerate}
\end{problem}

\begin{solution}
    {Solution}
    \begin{enumerate}[(a)]
        \item {
            \begin{proof}
                By definition 4.6.1, given $x\in K$, we have \[F_K(x)=\set{y\in\R^m\mid x+\alpha y\in K, \forall \alpha\in[0,\bar{\alpha}], \bar{\alpha}>0},\] where $y$ is a feasible direction. Here, to show that \[F_K(x)=\set{z-\alpha x\mid z\in K, \alpha\geq 0},\] we need to show $y=z-\alpha x$ is a feasible direction and every feasible direction can be represented as $y=z-\alpha x$.

                Let $y=z-\alpha x$ where $z\in K$ and $\alpha\geq 0$. Consider $x+ty$ for $t>0$, \[x+ty=x+t(z-\alpha x)=x+tz-t\alpha x=(1-t\alpha)x+tz.\] Since $K$ is a convex cone, it is closed under positive linear combinations. For sufficiently samll $t$, $1-t\alpha$ remains positive, and hence $(1-t\alpha)x+tz$ is a positive linear combination of points in $K$, which means $x+ty\in K$ for all sufficiently small $t>0$. Therefore, $y$ is a feasible direction at $x$.

                Let $y$ be any feasible directions in $F_K(x)$. By definition, for all small $t>0$, $x+ty\in K$. By the convexity of $K$, the line segment connecting $x$ and $x+ty$ must entirely lie in $K$. For a sufficiently small $t$, this implies that $y$ can be represented as \[y=\frac{1}{t}(x+ty)-\frac{1}{t}x\Rightarrow(x+ty)-x=z-x,\] where $z=x+ty\in K$. We can set $\alpha=1$ to match the form required. So, $y=z-\alpha x$ with $x\in K$ and $\alpha=1\geq 0$.
            \end{proof}
        }
        \item {
            \begin{proof}
                The proof contains two parts.
                \begin{enumerate}
                    \item $N_K(x)\subseteq\set{y\in K^*\mid \ip{x,y}=0}$. Take any $y\in N_K(x)$. By definition of the normal cone, for all $z\in K$ \[\ip{y, z-x}\leq 0.\] Since $K$ is a cone, for $\lambda>0$, $\lambda x$ is also in $K$. Replace $z$  by $\lambda x$ and get \[\ip{y, \lambda x-x}\leq 0,\] which simplifies to \[\lambda\ip{y,x}-\ip{y,x}\leq 0 \Rightarrow (\lambda-1)\ip{y, x}\leq 0.\] Since $\lambda$ is arbitrary, we must have $\ip{y, x}=0$. Besides, $\ip{y,x}\leq 0$ implies $y\in K^*$. Therefore, $N_K(x)\subseteq\set{y\in K^*\mid \ip{x,y}=0}$.
                    \item $\set{y\in K^*\mid \ip{x,y}=0}\subseteq N_K(x)$. By the definition of $K^*$, we have that for every $y\in K^*$ and for every $z\in K$, \[\ip{y,z}\leq 0.\] And given $\ip{y, x}=0$, we have \[\ip{y, z-x}=\ip{y, z}-\ip{y, x}\leq 0-0 =0,\] which implies $y\in N_K(x)$. Therefore, $\set{y\in K^*\mid \ip{x,y}=0}\subseteq N_K(x)$.
                \end{enumerate}
                The two parts together prove the statement.
            \end{proof}
        }
        \item {
            \begin{proof}
                For vector $v$ that is in $N_Z(x)$, it must have \[\ip{v, z-x}\leq 0 \qquad \forall z\in Z.\] To use the definition of $Z$, we have \[\ip{v, z-x}=\ip{A\trn v,Az-Ax}=\ip{A\trn v, (Az-b)-(Ax-b)}.\] For $v$ to be in the normal cone $N_Z(x)$, the last inner product should be non-positive for all $Az-b\in K$, which means that $A\trn v$ must be in the normal cone to $K$ at the point $Ax-b$, \ie $A\trn v\in N_K(Ax-b)$. With the previous question, we can related $N_K(Ax-b)$ to $K^*$ by \[N_K(Ax-b)=\set{\lambda\in K^*\mid \ip{Ax-b, \lambda}=0}.\] So, a vector $v\in N_Z(x)$ must correspond to a $\lambda$ in the dual cone $K^*$ such that $A\trn\lambda$ has a zero inner product with $Ax-b$, hence $v=A\trn\lambda$ for some $\lambda$ satisfying $\ip{Ax-b, \lambda}=0$. To account for the fact that the normal cone $N_Z(x)$ is a closed set, we take the closure of the set $\set{A\trn\lambda\mid\lambda\in K^*,\ip{Ax-b,\lambda}=0}$ since the image under a linear transformation of a closed set is not necessarily closed. Therefore, \[N_Z(x)=\cl\set{A\trn\lambda\mid\lambda\in K^*,\ip{Ax-b,\lambda}=0}.\]
            \end{proof}
        }
        \item {
            \begin{proof}
                Let \( x_0 \) be such that \( Ax_0 - b \in \text{ri}(K) \). By the prolongation principle, for every point \( \bar{y} \in K \), there exists \( \delta > 1 \) such that
                \[ Ax_0 - b + (\delta - 1)(Ax_0 - b - \bar{y}) \in K. \]

                Now, let \( x \) be any point in \( Z \), implying \( Ax - b \in K \).
                Taking \( \bar{y} = Ax - b \), the prolongation principle gives us
                \begin{align*}
                    Ax_0 - b + (\delta - 1)((Ax_0 - b) - (Ax - b)) &\in K \\
                    \Rightarrow A(x_0 + (\delta - 1)(x_0 - x))-b &\in K.
                \end{align*}
                showing that
                \[ x_0 + (\delta - 1)(x_0 - x) \in Z. \]
                Since this is true for any \( x \in Z \), \( x_0 \) must be in \( \text{ri}(Z) \).

                We conclude that
                \[ \text{ri}\, Z \supseteq \{x \in \mathbb{R}^n \mid Ax - b \in \text{ri}\, K\}. \]
            \end{proof}
        }
        \item {
            \begin{proof}
                Solving~\cref{eq:cone} is equivalent to minimizing $f_1+f_2$ where:
                \begin{enumerate}
                    \item $f_1(x)=f(x)$
                    \item $f_2(x)=\delta_K(Ax-b)=\begin{cases}
                        0, & Ax-b\in K \\
                        +\infty, & \text{otherwise}
                    \end{cases}$
                \end{enumerate}
                The range of $f_1$ and $f_2$ are
                \begin{enumerate}
                    \item $\ri\dom f_1=\ri\dom f$
                    \item $\ri\dom f_2=\ri Z$ ($Z$ is in the same form in the previous question)
                \end{enumerate}
                The condition says $\bar{x}\in\ri\dom f_1\cap\ri K\neq\emptyset$. So $\partial(f_1+f_2)(x)=\partial f_1(x)+\partial f_2(x)$. Given $x^*$ is optimal, we have
                \begin{align*}
                    0& \in\partial(f_1+f_2)(x^*) \\
                    \Rightarrow
                    0& \in\partial f_1(x^*)+\partial f_2(x^*) \\
                    \Rightarrow
                    0& \in\partial f(x^*)+N_Z(x^*) \\
                    \Rightarrow
                    0& \in\partial f(x^*)+A\trn\lambda^* \qquad \text{for some }\lambda^*\in K^*
                \end{align*}
                With $A\trn\lambda^*$ is closed (from statement), combining $N_Z(x)=\cl\set{A\trn\lambda\mid\lambda\in K^*,\ip{Ax-b,\lambda}=0}$ from previous result, we finish the proof.
            \end{proof}
        }
    \end{enumerate}
\end{solution}

\end{document}