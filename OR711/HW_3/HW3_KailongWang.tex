\documentclass{article}
\usepackage[
    letterpaper,
    top=1in,
    bottom=1in,
    inner=0.75in,
    outer=0.75in,
    % margin=1in,
]{geometry}

%%%% Page Header and Foot %%%%
\usepackage{fancyhdr}
\fancypagestyle{plain}{
\fancyhf{}
\fancyhead[L]{266:711:685}
\fancyhead[C]{\textbf{Convex Analysis and Optimization}}
\fancyhead[R]{Rutgers University}
\fancyfoot[L]{}
\fancyfoot[C]{\thepage}
\fancyfoot[R]{}
}
\pagestyle{plain}

%%%% Load Format %%%%
%%%% Page Margin %%%%
\RequirePackage{geometry}

%%%% Page Header and Foot %%%%
\RequirePackage{fancyhdr}
\pagestyle{fancyplain}
\lhead{\lefthead}
\chead{\centerhead}
\rhead{\righthead}
\lfoot{\leftbottom}
\cfoot{\centerbottom}
\rfoot{\rightbottom}

%%%% Additional Packages to Load %%%%
%%%% Main Font
\RequirePackage{lmodern} %Computer Modern, but with many more glyphs
\RequirePackage{microtype} %Better Font Rendering
\RequirePackage{libertinus} %Very popular math font
\RequirePackage[T1]{fontenc} %Better Printing Font
\RequirePackage[utf8]{inputenc} %enable utf8 encoding
\RequirePackage[english]{babel} %English Language

%%%% Math Env
\RequirePackage{amsfonts,amsmath,amssymb} %Math Font Packages
\RequirePackage{mathtools} %Better Math Environments
\RequirePackage{mleftright}  %Fixes some annoying spacing issues
\RequirePackage{bm} %Boldface math, load after other math font packages

%%%% Beautiful Stuff
\RequirePackage{graphicx} %Better Graphics
\RequirePackage[export]{adjustbox} %Better Graphics
\RequirePackage[dvipsnames,svgnames]{xcolor} %Must load before tcolorbox
\RequirePackage[breakable]{tcolorbox} %Customized Box
\tcbuselibrary{skins}
\RequirePackage{transparent} 

%%%% Table and List 
\RequirePackage{tabularx} %Better Table
\RequirePackage{multirow}
\RequirePackage{colortbl}
\RequirePackage{array}

%%%% Theorems etc (also  problems)
\RequirePackage[shortlabels]{enumitem} %Better List
\RequirePackage{hyperref, cleveref} %Fantastic crossrefer
\RequirePackage{amsthm, thmtools} %Better Theorems
\RequirePackage{thm-restate} %Better Theorems

%%%% Code Packages
\RequirePackage{algorithm2e} % Package to create pseudo-code
\RequirePackage{listings} % Package to insert code
% \RequirePackage{keyval}

%%%% Display Packages
\RequirePackage[document]{ragged2e} % Package to justify text
\RequirePackage{csquotes} % Package to facilities quotations
\RequirePackage{multicol} % Package to use multicols

%%%% Other Packages
\RequirePackage{calc}
\RequirePackage{rotating}
\RequirePackage{pgf, tikz}
\RequirePackage{epstopdf}
\RequirePackage[backend=biber, style=numeric, sorting=none]{biblatex}

%%%% A box fill the rest of the page
\newtcolorbox{stretchbox}[1][]{
    height fill,
    sharp corners,
    colback=white,
    % colframe=yellow!50!black,
    #1
    }

%%%% Box Environment
\newtcolorbox{prob}[1]{
% Set box style
    sidebyside,
    sidebyside align=top,
% Dimensions and layout
    width=\textwidth,
    toptitle=2.5pt,
    bottomtitle=2.5pt,
    righthand width=0.20\textwidth,
% Coloring
    colbacktitle=white,
    coltitle=black,
    colback=white,
    colframe=black,
% Title formatting
    title={
        #1 \hfill Grade:\hspace*{0.15\textwidth}
    },
    fonttitle=\large\bfseries
}

%%%% Problem Environment 
\newenvironment{problem}[1]{
    \begin{prob}{#1}
}
{
    \tcblower
    \centering
    \textit{\scriptsize\bfseries Faculty Comments}
    \end{prob}
}

%%%% Draw a line
\newcommand{\myrule}{\rule{1.5in}{0.1mm}}
%%%% delimiters
\DeclarePairedDelimiter\parens{\lparen}{\rparen}
\DeclarePairedDelimiter\bracks{\lbrack}{\rbrack}
\DeclarePairedDelimiter\braces{\lbrace}{\rbrace}
\DeclarePairedDelimiter\abs{\lvert}{\rvert}
\DeclarePairedDelimiter\norm{\lVert}{\rVert}
\DeclarePairedDelimiter\angles{\langle}{\rangle}
\DeclarePairedDelimiter\ceil{\lceil}{\rceil}
\DeclarePairedDelimiter\floor{\lfloor}{\rfloor}

%%%% math operators naming
\DeclareMathOperator*{\argmax}{\textnormal{argmax}}
\DeclareMathOperator*{\argmin}{\textnormal{argmin}}
\DeclareMathOperator{\tr}{\textnormal{tr}}
\DeclareMathOperator{\eig}{\textnormal{eig}}
\DeclareMathOperator{\sgn}{\textnormal{sgn}}
% \let\det\relax % "Undefine" \det
% \DeclareMathOperator{\det}{\textnormal{det}} % already defined in mathtools
\DeclareMathOperator{\diag}{\textnormal{diag}}
\DeclareMathOperator{\rank}{\textnormal{rank}}
\DeclareMathOperator{\Vol}{\textnormal{Vol}}   % volume
\DeclareMathOperator{\Surf}{\textnormal{Surf}} % surface area

%%%% Transforms! -- requires mathtools package
\newcommand*{\LapTrans}{\xleftrightarrow{\mathcal{Z}}}
\newcommand*{\ZTrans}{\xleftrightarrow{\mathcal{L}}}
\newcommand*{\CTFS}{\xleftrightarrow{\textnormal{CTFS}}}
\newcommand*{\CTFT}{\xleftrightarrow{\textnormal{CTFT}}}
\newcommand*{\DTFS}{\xleftrightarrow{\textnormal{DTFS}}}
\newcommand*{\DTFT}{\xleftrightarrow{\textnormal{DTFT}}}

%%%% vector font
\let\oldvec\vec
\renewcommand*{\vec}[1]{\mathbf{#1}}
% \newcommand*{\trn}{\!^{\!\intercal}}
\newcommand*{\trn}{\!^{\mathsf{T}}}
% \newcommand*{\coj}{\!^{\dag}} % Text Mode Symbol, Should not be used
% \newcommand*{\coj}{\!^{\dagger}}
\newcommand*{\coj}{\!^{\mathsf{H}}}
\newcommand*{\inv}{^{-1}}

%%%% number systems
\DeclareMathOperator{\R}{\mathbb{R}}
\DeclareMathOperator{\C}{\mathbb{C}}
\DeclareMathOperator{\N}{\mathbb{N}}
\DeclareMathOperator{\Z}{\mathbb{Z}}
\DeclareMathOperator{\F}{\mathbb{F}}
\DeclareMathOperator{\Q}{\mathbb{Q}}

%%%% STATISTICS AND PROBABILITY
\newcommand*{\Var}{\mathop{\textnormal{Var}}}
\newcommand*{\Cov}{\mathop{\textnormal{Cov}}}
\newcommand*{\Corr}{\mathop{\textnormal{Corr}}}
\newcommand*{\MSE}{\mathop{\textnormal{MSE}}}
\newcommand*{\MSD}{\mathop{\textnormal{MSD}}}
\newcommand*{\NSD}{\mathop{\textnormal{NSD}}}

\newcommand*{\E}[1]{\mathbb{E}\bracks*{#1}}
\newcommand*{\condE}[2]{\mathbb{E}\bracks*{#1 \mid #2}}
\renewcommand*{\P}[1]{\mathbb{P}\parens*{#1}}
\newcommand*{\condP}[2]{\mathbb{P}\parens*{#1 \mid #2}}

\DeclareMathOperator{\Bern}{\mathsf{Bern}}
\DeclareMathOperator{\Unif}{\mathsf{Unif}}
\DeclareMathOperator{\Expv}{\mathsf{Exp}}
\DeclareMathOperator{\Poi}{\mathsf{Poi}}
\DeclareMathOperator{\Gamv}{\mathsf{Gamma}}
\DeclareMathOperator{\Dirv}{\mathsf{Dir}}
\DeclareMathOperator{\Mult}{\mathsf{Mult}}
\DeclareMathOperator{\Beta}{\mathsf{Beta}}
\DeclareMathOperator{\Geomv}{\mathsf{Geom}}
\DeclareMathOperator{\Binomv}{\mathsf{Binom}}
\DeclareMathOperator{\NegBinomv}{\mathsf{NB}}
\DeclareMathOperator{\Lap}{\mathsf{Lap}}
\DeclareMathOperator{\Gaus}{\mathsf{N}}
\DeclareMathOperator{\Weibull}{\mathsf{Weibull}}


\DeclareMathOperator{\iidsim}{\stackrel{\textnormal{i.i.d.}}{\sim}}
\DeclareMathOperator{\diff}{\mathop{}\!\textnormal{d}}

%%%% Special norms and linear algebra stuff
\newcommand*{\subgnorm}[1]{\norm*{#1}_{\psi_2}}
\newcommand*{\subexpnorm}[1]{\norm*{#1}_{\psi_1}}
\newcommand*{\frobnorm}[1]{\norm*{#1}_{\textnormal{F}}}
\newcommand*{\opnorm}[1]{\norm*{#1}_{\textnormal{op}}}
\newcommand*{\Lipnorm}[1]{\norm*{#1}_{\textnormal{Lip}}}

%%%%
\newcommand*{\set}[1]{\braces*{\,#1\,}}
\newcommand*{\ie}{\textnormal{i.e.\ }}
\newcommand*{\eg}{\textnormal{e.g.\ }}
\newcommand*{\etc}{\textnormal{etc.\ }}
\newcommand*{\iid}{\textnormal{i.i.d.\ }}
\allowdisplaybreaks

%%%% Theorem Style %%%%
\declaretheorem[numbered=no, style=plain]{axiom, lemma}
\declaretheorem[numberwithin=section,style=definition]{definition}
\declaretheorem[sibling=definition]{theorem, corollary, proposition, conjecture}
\declaretheorem[numbered=no,style=remark]{remark, claim}

%%%% Document Information %%%%
\title{Homework 3}
\author{Kailong Wang}
\date{\today}

%%%% Start Document %%%%
\begin{document}
\maketitle

\begin{problem}
    {Q1: Polar Cone Operations. Problems 3.4(a) - (c)}
    Show the following:
    \begin{enumerate}[(a)]
        \item For any nonempty cones $C_i\subset \R^{n_i}$, $i=1,2,\ldots,m$, we have \[\parens*{C_1 \times C_2 \times \cdots \times C_m}^*=C_1^*\times C_2^* \times \cdots \times C_m^*.\]
        \item For any collection of nonempty cones $\set{C_i\mid i\in I}$, we have \[\parens*{\cup_{i\in I}C_i}^*=\cap_{i\in I}C_i^*.\]
        \item For any two nonempty cones $C_1$ and $C_2$, we have \[\parens*{C_1+C_2}^*=C_1^*\cap C_2^*.\]
    \end{enumerate}
    Hint: to show $C^*=K$, the simplest general strategy is usually to show that $\ip{x,y}\leq 0$ for all $x\in C$ and $y\in K$, establishing $K\subseteq C^*$, and then show that if $z\notin K$, then there exists some $x\in C$ with $\ip{x,z}>0$, implying that $z\notin C^*$, and thus $C^*=K$ since $z\notin K$ was arbitrary.
\end{problem}

\begin{solution}
    {Solution}
    \begin{enumerate}[(a)]
        \item {
            \begin{proof}
                Let $C=\parens*{C_1 \times C_2 \times \cdots \times C_m}^*$ and $C' = C_1^*\times C_2^* \times \cdots \times C_m^*$. We will show that $C=C'$ by showing that $C\subseteq C'$ and $C'\subseteq C$.
                \begin{enumerate}[(i)]
                    \item {
                        Let $x\in C$. Then for all $y\in C_1\times C_2\times\cdots\times C_m$, we have $\ip{x,y}\leq 0$. Equivalently, that is $\sum_{i=1}^{m}x_i y_i\leq0$ where $y_i\in C_i$ for all $i\in\set{1,2,\ldots,m}$. Since $C_i$ are cones and $0$ belongs to their closure,  then $\ip{x_i,y_i}\leq 0$ for all $i\in\set{1,2,\ldots,m}$ by letting all $y_k\rightarrow 0$, $i\neq k$. Thus $x_i\in C_i^*$ for all $i\in\set{1,2,\ldots,m}$. Therefore, $x\in C'$ and then $C\subseteq C'$.
                    }
                    \item {
                        Let $x\in C'$. Then $x=(x_1,x_2,\ldots,x_m)$ where $x_i\in C_i^*$ for all $i\in\set{1,2,\ldots,m}$. Let $y\in C_1\times C_2\times\cdots\times C_m$. Then $y=(y_1,y_2,\ldots,y_m)$ where $y_i\in C_i$ for all $i\in\set{1,2,\ldots,m}$. Then $\ip{x_i,y_i}\leq 0$ for all $i\in\set{1,2,\ldots,m}$. Thus $\ip{x,y}\leq 0$. Therefore, $x\in C$ and then $C'\subseteq C$.
                    }
                \end{enumerate}
                % Therefore $C=C'$. Thus $C=\parens*{C_1 \times C_2 \times \cdots \times C_m}^*=C_1^*\times C_2^* \times \cdots \times C_m^*$.
            \end{proof}
        }
        \item {
            \begin{proof}
                Let $C=\parens*{\cup_{i\in I}C_i}^*$ and $C'=\cap_{i\in I}C_i^*$. We will show that $C=C'$ by showing that $C\subseteq C'$ and $C'\subseteq C$.
                \begin{enumerate}[(i)]
                    \item {
                        Let $x\in C$. Then for all $y\in \cup_{i\in I}C_i$, we have $\ip{x,y}\leq 0$. Equivalently, that is $\ip{x,y_i}\leq0$ where $y\in C_i$ for all $i\in I$. Thus, $x\in C_i^*$ for all $i\in I$. Therefore, $x\in C'$ and then $C\subseteq C'$.
                    }
                    \item {
                        Let $x\in C'$. Then $x\in C_i^*$ for all $i\in I$. Let $y\in \cup_{i\in I}C_i$. Then $y_i\in C_i$ for $i\in I$. Then $\ip{x,y_i}\leq 0$. Thus $\ip{x,y}\leq 0$. Therefore, $x\in C$ and then $C'\subseteq C$.
                    }
                \end{enumerate}
                % Therefore $C=C'$. Thus $C=\parens*{\cup_{i\in I}C_i}^*=\cap_{i\in I}C_i^*$.
            \end{proof}
        }
        \item {
            \begin{proof}
                Let $C=\parens*{C_1+C_2}^*$ and $C'=C_1^*\cap C_2^*$. We will show that $C=C'$ by showing that $C\subseteq C'$ and $C'\subseteq C$.
                \begin{enumerate}[(i)]
                    \item {
                        Let $x\in C$. Then for all $y\in C_1+C_2$, we have $\ip{x,y}\leq 0$. Equivalently, that is $\ip{x,y_1+y_2}\leq0$ where $y_1\in C_1$ and $y_2\in C_2$. Thus, $\ip{x,y_1}+\ip{x,y_2}\leq 0$. Since $C_1$ and $C_2$ are cones and $0$ belongs to their closure, following the same logic in (a), $\ip{x,y_1}\leq 0$ and $\ip{x,y_2}\leq 0$. Thus $x\in C_1^*$ and $x\in C_2^*$. Therefore, $x\in C'$ and then $C\subseteq C'$.
                    }
                    \item {
                        Let $x\in C'$. Then $x\in C_1^*$ and $x\in C_2^*$. Let $y\in C_1+C_2$. Then $y=y_1+y_2$ where $y_1\in C_1$ and $y_2\in C_2$. Then $\ip{x,y_1}\leq 0$ and $\ip{x,y_2}\leq 0$. Thus $\ip{x,y_1+y_2}\leq 0$. Therefore, $\ip{x,y}\leq 0$. Therefore, $x\in C$ and then $C'\subseteq C$.
                    }
                \end{enumerate}
                % Therefore $C=C'$. Thus $C=\parens*{C_1+C_2}^*=C_1^*\cap C_2^*$.
            \end{proof}
        }
    \end{enumerate}
    I didn't follow the hint. Please let me know my mistakes if this proof doesn't work. Thanks!
\end{solution}

\begin{problem}
    {Q2: Cone Separation}
    Suppose $K\in\R^n$ is a nonempty closed convex cone. Show that if $z\in \R^n$ and $z\notin K$, then there exists $a\in K^*$ with $\ip*{a,z}>0$.
\end{problem}

\begin{solution}
    {Solution}
    \begin{proof}
        Using \textbf{Separating Hyperplane Theorem}, if $K$ is a nonempty closed convex cone and $z\notin K$, then by the theorem, there exists a hyperplane that can separate $z$ from $K$. This means there exists $a\neq 0$ such that $\ip*{a, x}\leq 0$ for all $x\in K$ and $\ip*{a, z}>0$. By definition of polar cone, $a\in K^*$. Thus, there exists $a\in K^*$ with $\ip*{a,z}>0$.
    \end{proof}
\end{solution}

\begin{problem}
    {Q3: Sums of Convex Cones}
    Show that if $C_1, C_2\subseteq \R^n$ are convex cones, then $C_1+C_2$ is a convex cone.
\end{problem}

\begin{solution}
    {Solution}
    \begin{proof}
    We will show that $C_1+C_2$ is a convex cone by showing that $C_1+C_2$ is a cone and $C_1+C_2$ is convex.
    \begin{enumerate}[(a)]
        \item Convexity. {
                Let $x,y\in C_1+C_2$ and $\theta\in[0,1]$. Then $x=x_1+x_2$ and $y=y_1+y_2$ where $x_1,y_1\in C_1$ and $x_2,y_2\in C_2$. Then $\theta x+(1-\theta)y=\theta x_1+(1-\theta)y_1+\theta x_2+(1-\theta)y_2$. Since $C_1$ and $C_2$ are convex, $\theta x_1+(1-\theta)y_1\in C_1$ and $\theta x_2+(1-\theta)y_2\in C_2$, which implies $\theta x+(1-\theta)y\in C_1+C_2$. Therefore, $C_1+C_2$ is convex.
        }
        \item Coneness. {
            % \begin{proof}
                Let $x\in C_1+C_2$ and $\theta\geq 0$. Then $x=x_1+x_2$ where $x_1\in C_1$ and $x_2\in C_2$. Then $\theta x=\theta x_1+\theta x_2$. Since $C_1$ and $C_2$ are cones, $\theta x_1\in C_1$ and $\theta x_2\in C_2$, which implies $\theta x\in C_1+C_2$. Therefore, $C_1+C_2$ is a cone.
            % \end{proof}
        }
    \end{enumerate}
\end{proof}
\end{solution}

\begin{problem}
    {Q4}
    Show that if $C_1, C_2 \R^m$ are closed convex cones, then $\parens*{C_1\cap C_2}^*=\cl \parens*{C_1^*+C_2^*}$.

    Note: this is the main result of problem 3.4(d).

    Hint: to show that $z\notin\cl(C_1^*+C_2^*)$ implies $z\notin \parens*{C_1\cap C_2}^*$, use problem 2, problem 1(c), and the polar cone theorem.
\end{problem}

\begin{solution}
    {Solution}
    \begin{proof}
        We need to show that $\cl \parens*{C_1^*+C_2^*} \subseteq \parens*{C_1\cap C_2}^*$ and $z\notin\cl(C_1^*+C_2^*)$ implies $z\notin \parens*{C_1\cap C_2}^*$.
        \begin{enumerate}[(i)]
            \item {
                For any $y\in \cl(C_1^*+C_2^*)$, there exists $y_1\in C_1^*$ and $y_2\in C_2^*$ such that for any $\epsilon>0$, $\norm{y-(y_1+y_2)}<\epsilon$. Let $x\in C_1\cap C_2$, we have $\ip*{y, x}=\ip*{y_1+y_2, x}+\ip*{y-(y_1+y_2), x}$. Because $y_1$ is in the polar of $C_1$ and $y_2$ is in the polar of $C_2$, $\ip*{y_1, x}\leq 0$ and $\ip*{y_2, x}\leq 0$, which implies $\ip*{y_1+y_2, x}\leq 0$. Using Cauchy-Schwarz inequality, $\ip*{y-(y_1+y_2), x}\leq \norm{y-(y_1+y_2)}\norm{x}<\epsilon\norm{x}$. Since $\epsilon$ is arbitrary, $\ip*{y-(y_1+y_2), x}\leq 0$. Thus, $\ip*{y, x}\leq 0$. Therefore, $y\in \parens*{C_1\cap C_2}^*$ and then $\cl(C_1^*+C_2^*)\subseteq \parens*{C_1\cap C_2}^*$.
            }
            \item {
                If $z\notin\cl(C_1^*+C_2^*)$, by the cone separation theorem, there exists an $x$ such that \[\ip{x,z}>0, \qquad \ip*{x,p}\leq 0\] for all $p\in \cl(C_1^*+C_2^*)$. The second condition implies that $x\in(\cl(C_1^*+C_2^*))^*=(C_1^*+C_2^*)^*=C_1^{**}\cap C_2^{**}=C_1\cap C_2$. Then the first condition implies, $z\in C_1\cap C_2$ which means $z\notin\cl(C_1\cap C_2)^*$.
            }
        \end{enumerate}
        Thus we complete the proof.
        % Therefore $C=C'$. Thus $C=\parens*{C_1\cap C_2}^*=\cl \parens*{C_1^*+C_2^*}$.
    \end{proof}
\end{solution}

\begin{problem}
    {Q5}
    Let $A$ be an $m\times n$ real matrix, and $C\subseteq \R^m$ be a closed convex cone. Define \[K=\set{x\in\R^n\mid Ax\in C} \qquad P=\set{A\trn y\mid y\in C^*}.\]
    \begin{enumerate}[(a)]
        \item Show that $K$ is a closed convex cone.
        \item Show that $K^*=\cl P$. Hint: to show that $z\notin \cl P$ implies $z\notin K^*$, use problem 2.
        \item Show that $P^*=K$ (by using the polar cone theorem).
    \end{enumerate}
\end{problem}

\begin{solution}
    {Solution}
    \begin{enumerate}[(a)]
        \item {
            \begin{proof}
                To show that $K$ is a closed convex cone, we need to prove the convexity, closedness and coneness of $K$.
                \begin{enumerate}[(i)]
                    \item Coneness. Given any $x\in K$, $Ax\in C$, and $\lambda\geq0$, since $C$ is a cone, $A\lambda x=\lambda Ax\in C$. Thus, $\lambda x\in K$ and $K$ is a cone.
                    \item Convexity. Given $x_1, x_2 \in K$, $Ax_1, Ax_2 \in C$, since $C$ is a convex set, for any $\lambda\in[0, 1]$, we have \[\lambda Ax_1+(1-\lambda)Ax_2=A(\lambda x_1+(1-\lambda)x_2)=A\cdot\lambda x_1+A\cdot(1-\lambda)x_2.\] Thus, $\lambda x_1+(1-\lambda)x_2\in K$ and $K$ is a convex set.
                    \item Closedness. Since $C$ is a closed set, $Ax$ is a continuous function. Thus, $K=A^{-1}Ax$ is a closed set.
                \end{enumerate}
            \end{proof}
        }
        \item {
            \begin{proof}
                % We will show that $K^*=\cl P$ by showing that $K^*\subseteq \cl P$ and $\cl P\subseteq K^*$.
                Given $z\in K^*$, for all $x\in K$, we have \[\ip*{z,x}\leq 0, \qquad \forall x, s.t. Ax\in C.\] By definition of polar cone $C^*$, $y\in C^*$ if and only if $\ip*{y, Ax}\leq 0$. Hence, for the same $x$, \[\ip*{A\trn y, x}\leq 0.\] If $z$ is an element in $A\trn y\in P$. Since $P$ is not necessarily closed, we have $z\in \cl P$, which is $K^*\subseteq \cl P$. If $z\notin \cl P$, by the cone separation theorem, there exists $x\in (\cl P)^*=P^*$ such that $\ip*{x, z}>0$, and $\ip*{x, p}\leq 0$ for all $p\in \cl P$. Given that $P=\set{A\trn y \mid y\in C^*}$, we can write the second condition as $\ip*{x, A\trn y} \leq 0$ for all $y\in C^*$. This is equivalent to $\ip*{y, Ax}\leq 0$. This implies $Ax\in C$ and $x\in K$. Since $\ip*{x, z}>0$ and $x\in K$, we have $z\in K$ and so the $z\notin K^*$. Then the proof is completed by the hint.
            \end{proof}
        }
        \item {
            \begin{proof}
                The polar cone theorem states that for a convex cone $C$, $C^{**}=C$ and $(\cl C)^*=C^*$. From (b), $K^*=\cl P\Rightarrow (K^*)^*=P^*\Rightarrow K=P^*$. Thus, $P^*=K$.
            \end{proof}
        }
    \end{enumerate}
\end{solution}

\begin{problem}
    {Q6}
    A cone $K$ is called \textit{self-dual} if $K^*=-K$. Show that the following cones are self-dual:
    \begin{enumerate}[(a)]
        \item The non-negative orthant $\set{x\in\R^n\mid x\geq 0}$.
        \item The \textit{Lorentz cone} (also called the ``ice cream cone'') in $\R^{n+1}$, defined as follows: \[K=\set{(x,w)\in \R^n\times\R\mid w\geq\norm{x}}.\]
    \end{enumerate}
\end{problem}

\begin{solution}
    {Solution}
    \begin{enumerate}[(a)]
        \item {
            \begin{proof}
                Let $K=\set{x\in\R^n\mid x\geq 0}$. Then $K^*=\set{y\in\R^n\mid \ip*{x,y}\leq 0, \forall x\in K}$. Since $x\geq 0$, $\ip*{x,y}\leq 0$ for all $y\leq 0$. Thus, $K^*=\set{y\in\R^n\mid y\leq 0}$. Therefore, $K^*=-K$ and $K$ is self-dual.
            \end{proof}
        }
        \item {
            \begin{proof}
                Let $K=\set{(x,w)\in \R^n\times\R\mid w\geq\norm{x}}$. Then $K^*=\set{(y,z)\in \R^n\times\R\mid \ip*{(x,w),(y,z)}\leq 0, \forall (x,w)\in K}$. If $x=0$, then $w\geq \norm{x}=0$ and $\ip*{(x,w),(y,z)} = x\trn y+wz \leq 0$ implies $z\leq 0$. If $x\neq 0$, let $u=\frac{x}{\norm{x}}$ be unit vector, $\ip*{(x,w),(y,z)} = x\trn y+wz \leq 0 \Rightarrow u\trn y+z\leq 0$. To make the inequality hold, there must be $y=0$ and $z\leq 0$ since $u$ represents all directions. Thus, $K^*=\set{(0,z)\in \R^n\times\R\mid z\leq 0}$. Therefore, $K^*=-K$ and $K$ is self-dual.
            \end{proof}
        }
    \end{enumerate}
\end{solution}

\end{document}